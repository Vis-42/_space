\documentclass{report}

\input{latex-templates/preamble}
\newcommand{\eps}{\epsilon}
\newcommand{\veps}{\varepsilon}
\newcommand{\Qed}{\begin{flushright}\qed\end{flushright}}

\newcommand{\parinn}{\setlength{\parindent}{1cm}}
\newcommand{\parinf}{\setlength{\parindent}{0cm}}

% \newcommand{\norm}{\|\cdot\|}
\newcommand{\inorm}{\norm_{\infty}}
\newcommand{\opensets}{\{V_{\alpha}\}_{\alpha\in I}}
\newcommand{\oset}{V_{\alpha}}
\newcommand{\opset}[1]{V_{\alpha_{#1}}}
\newcommand{\lub}{\text{lub}}
\newcommand{\del}[2]{\frac{\partial #1}{\partial #2}}
\newcommand{\Del}[3]{\frac{\partial^{#1} #2}{\partial^{#1} #3}}
\newcommand{\deld}[2]{\dfrac{\partial #1}{\partial #2}}
\newcommand{\Deld}[3]{\dfrac{\partial^{#1} #2}{\partial^{#1} #3}}
\newcommand{\der}[2]{\frac{\mathrm{d} #1}{\mathrm{d} #2}}
% \newcommand{\ddd}[3]{\frac{\mathrm{d}^{#3} #1}{\mathrm{d}^{#3} #2}}
\newcommand{\lm}{\lambda}
\newcommand{\uin}{\mathbin{\rotatebox[origin=c]{90}{$\in$}}}
\newcommand{\usubset}{\mathbin{\rotatebox[origin=c]{90}{$\subset$}}}
\newcommand{\lt}{\left}
\newcommand{\rt}{\right}
\newcommand{\bs}[1]{\boldsymbol{#1}}
\newcommand{\exs}{\exists}
\newcommand{\st}{\strut}
\newcommand{\dps}[1]{\displaystyle{#1}}
\newcommand{\id}{\text{id}}
\newcommand{\imps}{\quad \Rightarrow \quad}
\newcommand{\cimps}{\quad \Leftrightarrow \quad}
\newcommand{\kyuki}[1]{\quad \quad \bqty{\because \eqref{#1}}}
\newcommand{\kyukifir}[2]{\quad \quad \bqty{\because \eqref{#1} \& \eqref{#2}}}
\newcommand{\boxdia}[2]{\begin{wrapfigure}{r}{#1\textwidth}
		\fbox{\includegraphics[width=\linewidth]{Figures/#2.png}}
	\end{wrapfigure}}
\newcommand{\dia}[2]{\begin{wrapfigure}{r}{#1\textwidth}
		\includegraphics[width=\linewidth]{Figures/#2.png}
	\end{wrapfigure}}
\newcommand{\boxudia}[2]{\begin{figure}[H]
		\centering
		\fbox{\includegraphics[width=#1\textwidth]{Figures/#2.png}}
		\end{figure}}
\newcommand{\udia}[2]{\begin{figure}[H]
		\centering
		\includegraphics[width=#1\textwidth]{Figures/#2.png}
	\end{figure}}
\newcommand{\su}[2]{\textcolor{my#1}{#2}}
\newcommand{\shs}[1]{\\ \textbf{{\Large #1}}\\}
\newcommand{\sss}[1]{\vspace*{-1cm} \subsubsection*{#1}}
\newcommand{\unt}[1]{\text{#1}}
\newcommand{\wa}{
	\noindent\rule{\textwidth}{0.4pt} 
	\vspace{0.5cm}}
\newcommand{\wb}{\noindent\rule{\textwidth}{0.4pt}}
\newcommand{\qmi}{\int_{-\infty}^{\infty}}
\newcommand{\qmk}{|\psi(x,0)|^{2}}
\newcommand{\qml}{\exp{-\frac{(x - x_0)^2}{4\sigma_0^2} + \frac{i}{\hbar}p_0 x}}
\newcommand{\qmls}{\exp{-\frac{(x - x_0)^2}{4\sigma_0^2} - \frac{i}{\hbar}p_0 x}}
\newcommand{\e}[1]{\exp\lt(#1\rt)}
\newcommand\prm[2][^n]{\prescript{#1\mkern-2.5mu}{}P_{#2}}
\newcommand\cmb[2][^n]{\prescript{#1\mkern-0.5mu}{}C_{#2}}
\newcommand{\ki}[1]{\lt[\therefore #1\rt]}
\newcommand{\h}{\underset{\rotatebox{135}{\#}}{}}
\newcommand{\f}{\frac{1}{2}}


%\newcommand{\sol}[1]{\vspace{0.5cm} 
%\setlength{\parindent}{0cm} \textcolor{mytheoremfr}{\textbf{\underline{Solution:}}} \textcolor{mytheoremfr}{#1}}
\newcommand{\solve}[1]{\setlength{\parindent}{0cm}\textbf{\textit{Solution: }}\setlength{\parindent}{1cm}#1 \Qed}

\input{latex-templates/letterfonts}
\usepackage{multicol}
\usepackage{physics}
\usepackage{float}
\usepackage{hyperref}
\usepackage{wrapfig}
\usepackage{pgfplots}

\setlength{\fboxsep}{1pt} % Space between image and border
\setlength{\fboxrule}{0.5pt} % Border thickness

\setlength{\columnsep}{20pt} % Adjust space between columns
\setlength{\columnseprule}{1pt}% Thickness of vertical line

\title{\Huge{PC2032 Classical Mechanics 1}\\Homework Assignment 9}
\author{\huge{Parth Bhargava}\\ AO310667E}
\date{}

\begin{document}
	\maketitle
	
	\pbm{}{
	A block of mass $m$ is revolving with linear speed $v_1$ in a circle of radius $r_1$ on a frictionless horizontal surface. The string is slowly pulled from below until the radius of the circle in which the block is revolving is reduced to $r_2$.
	\begin{enumerate}
		\item[a.] Calculate the tension $T$ in the string as a function of $r$, the distance of the block from the hole. Your answer will be in terms of the initial velocity $v_1$ and the radius $r_1$.
		\item[b.] Use
		$$W=\int_{r_1}^{r_2}\vec{T}(r)\dd\vec{r}$$
		to calculate the work done by $T$ when $r$ changes from $r_1$ to $r_2$.
		\item[c.] Compare the results of part (b) to the change in the kinetic energy of the block.
	\end{enumerate}
	}
	\sol{}{
	\udia{0.4}{rdf80}
	\sss{a.}
	For circular motion,
	\begin{align*}
		T&=\frac{mv^2}{r}=\frac{L^2}{mr^3}=\frac{m^2v_1^2r_1^2}{mr^3} \imps T=\frac{mv_1^2r_1^2}{r^3} \imps \vec{T}(r)= -\frac{mv_1^2r_1^2}{r^3} \hat{r} \h
	\end{align*}
	\sss{b.}
	The work done by tension is given by,
	\begin{align*}
		W&=\int_{r_1}^{r_2}\vec{T}(r)\dd\vec{r}= \int_{r_1}^{r_2}-\frac{mv_1^2r_1^2}{r^3}\dd r \imps W= mv_1^2r_1^2 \lt[\frac{1}{2r^2}\rt]_{r_1}^{r_2} \imps W=\frac{mv_1^2r_1^2}{2} \lt[\frac{1}{r_2^2} - \frac{1}{r_1^2}\rt]\h
	\end{align*}
	\sss{c.}
	The change in kinetic energy is given by,
	\begin{align*}
		\Delta \sE_K &= \f mv_2^2 -\f mv_1^2 = \f \frac{L^2}{mr_2^2} - \f \frac{L^2}{mr_1^2} = \f \frac{mv_1^2r_1^2}{r_2^2} - \f \frac{mv_1^2r_1^2}{r_1^2} \imps \Delta \sE_K = \frac{mv_1^2r_1^2}{2} \lt[\frac{1}{r_2^2} - \frac{1}{r_1^2}\rt] = W \h
	\end{align*}
	}
	\wa
	
	\pagebreak
	\pbm{}{
	A satellite initially moves around the Earth in a circular orbit of radius $r$. Suddenly, an explosion breaks the satellite into two pieces, with mass $m$ and $4m$. Immediately after the explosion the smaller piece of mass $m$ is stationary with respect to the Earth and falls directly toward the Earth.
	\begin{enumerate}
		\item[a.] What is the speed $v_0$ of the satellite before the explosion?
		\item[b.] What is the speed of the larger piece immediately after the explosion?
		\item[c.] Because of the increase in its speed, this larger piece now moves in a new elliptical orbit. Find the distance away from the centre of the Earth when it reaches the other end of the ellipse.
	\end{enumerate}
	}
	\sol{}{
	\udia{1}{rdf79}
	\sss{a.}
	For circular motion of the satellite,
	\begin{align*}
		\mcF_g&=\frac{(5m)v_0^2}{r} \imps v_0^2=\frac{r}{5m}\frac{G M_e (5m)}{r^2} \imps v_0=\sqrt{\frac{G M_e}{r}} \h
	\end{align*}
	\sss{b.}
	By conservation of angular momentum,
	\begin{align*}
		5mv_0r&=4mv_1r+0 \imps v_1=\frac{5}{4}v_0 \imps v_1=\frac{5}{4}\sqrt{\frac{GM_e}{r}}\h
	\end{align*}
	\sss{c.}
	The total energy of the body of mass $4m$ is given by,
	\begin{align*}
		\mfE&=\f (4m) v_1^2 - \frac{GM_e(4m)}{r}\\
		&=\f (4m) \lt(\frac{25}{16}\frac{GM_e}{r}\rt) - \frac{GM_e(4m)}{r}\\
		&=-\frac{7}{8}\frac{GM_em}{r}
	\end{align*}
	When the body is at the opposite side of the elliptical orbit, at a distance of $\mcR$ from Earth,
	\begin{align*}
		\mfE&=\f (4m) v_2^2 - \frac{GM_e(4m)}{\mcR}\\
		&=\f (4m) \lt(\frac{v_1r}{\mcR}
		\rt)^2 - \frac{GM_e(4m)}{\mcR} && \lt[\because (4m)v_1r=(4m)v_2\mcR\rt]\\
		&=\f (4m) \lt(\frac{r}{\mcR}
		\rt)^2\lt(\frac{25}{16}\frac{GM_e}{r}\rt) - \frac{GM_e(4m)}{\mcR} \\
		-\frac{7}{8}\frac{GM_em}{r}&=\frac{GM_em}{8\mcR}\lt(25\frac{r}{\mcR}-32\rt)\\
		-7\frac{\mcR}{r}&=\lt(25\frac{r}{\mcR}-32\rt)\\
	\end{align*}
	Let $\dfrac{r}{\mcR}=x$,
	\begin{align*}
		-\frac{7}{x}=\lt(25x-32\rt)\imps 25x^2-32x+7=0 \imps (x-1)(25x-7)=0
	\end{align*}
	Since the orbit is elliptical and not circular, 
	\begin{align*}
		x=\frac{7}{25} \imps \frac{r}{\mcR}= \frac{7}{25} \imps \mcR= \frac{25r}{7}\h
	\end{align*}
	}
	\wa
	
	\pagebreak
	\pbm{}{
	A sphere of matter, mass $M$ , radius $a$, has a concentric cavity, radius $b$ as shown in cross section below.
	\udia{0.4}{rdf68}
	\begin{enumerate}
		\item[a.] Sketch the gravitational force $F$ exerted by the sphere on a particle of mass $m$, located a distance $r$ from the centre of the sphere, as a function of $r$ in the range $0\leq r\leq \infty$. Consider points $r = 0, b, a, \text{and} \infty$ in particular.
		\item[b.] Sketch the corresponding curve for the potential energy $U(r)$ of the system.
	\end{enumerate}
	}
	\sol{}{
	\udia{0.4}{rdf76}
	\sss{a.}
	Consider the conservative vector field $g$ that gives us the gravitational strength per unit mass. For this field, we can apply the Gauss Law to give us,
	\begin{align*}
		\div \va{g} &= -4\pi G \rho \quad \quad \text{where, }\rho \text{  is the mass density}\\
		OR\\
		\oint \va{g}\cdot\dd s &= -4\pi G M_{enc}
	\end{align*}
	For Gaussian surfaces with radii \( r_1, r_2, r_3 \) respectively,  
	\begin{align*}
		g_1 (4\pi r_1^2) &= 0 \imps g_1 = 0 \\
		g_2 (4\pi r_2^2) &= - 4\pi G \left( \frac{4\pi}{3} \right) (r_2^3 - b^3) \left( \frac{M}{(4\pi/3)(a^3 - b^3)} \right) \imps g_2 = \frac{GM}{a^3 - b^3} \left( \frac{b^3 - r_2^3}{r_2^2} \right) \\
		g_3 (4\pi r_3^2) &= -4\pi GM \imps g_3 = -\frac{GM}{r_3^2}
	\end{align*}
	Specifically,
	\begin{enumerate}
		\item $r =0 \imps g_1 = 0 \imps \mcF_g(0) = mg =0$
		\item $r_1=b \text{ and } r_2=b \imps g_2 = g_1= 0 \imps \mcF_g(b) = mg =0$
		\item $r_2=a \text{ and } r_3=a \imps g_2 = g_3= -\dfrac{GM}{a^2} \imps \mcF_g(a) = mg =-\dfrac{GMm}{a^2}$
		\item $r \to \infty \imps g_3 = 0 \imps \mcF_g(r)\eval_{r\to\infty} = mg =0$
	\end{enumerate}
	Hence,
	$$
	\mcF_g(r)= mg = 
		\begin{cases}
		0 &\qc 0< r \leq b\\
		\dfrac{GMm}{a^3 - b^3} \lt( \dfrac{b^3 - r^3}{r^2} \rt) &\qc b<r \leq a\\
		-\dfrac{GMm}{r^2} &\qc a<r< \infty \\
	\end{cases}
	$$
	Here the '-' sign represents that the force is attractive and points opposit to the position vector $\va{r}$.\\
	Thus the graph looks like:
	\udia{0.8}{rdf78}
	\sss{b.}
	The scalar potential field \( \mcU (r) \) corresponding to the force field \( \mcF_g (r) \) is given by,
	\begin{align*}
		\mcU (r) &= \int_{\infty}^{r} \mcF_g (r) \cdot \dd r
	\end{align*}
	Subsequently, we can write,
	$$
	\mcU (r) =
	\begin{cases}
		C_1 & \qc 0 < r \leq b \\[10pt]
		-\dfrac{GMm}{a^3 - b^3} \left( \dfrac{b^3}{r} - \dfrac{r^2}{2} \right) + C_2 & \qc b < r \leq a \\[10pt]
		-\dfrac{GMm}{r} + C_3 & \qc a < r \leq \infty
	\end{cases}
	$$
	Applying the conditions that $\mcU (a)$ and $\mcU (b)$ must approach the same values from both sides and $\mcU (r)$ is equal to 0 as $r \to \infty$,
	$$
	\mcU (r) =
	\begin{cases}
		- \dfrac{3 GMm (a + b)}{2(a^2 + ab + b^2)} & \text{if } 0 < r \leq b, \\
		\dfrac{GMm}{a^3 - b^3} \left( \dfrac{b^3}{r} + \dfrac{1}{2} r^2 - \dfrac{b^3}{a} - \dfrac{1}{2} a^2 \right) - \dfrac{GMm}{a} & \text{if } b < r \leq a, \\
		- \dfrac{GMm}{r} & \text{if } a < r < \infty.
	\end{cases}
	$$
	Thus the graph looks like:
	\udia{0.8}{rdf77}
	}
	\wa
	
	\pagebreak
	\pbm{}{
	A thin, uniform rod has length $L$ and mass $M$ . A small uniform sphere of mass $m$ is placed a distance $x$ from one end of the rod.
	\begin{enumerate}
		\item[a.] Calculate the gravitational potential energy of the rod-sphere system. Take the potential energy to be zero when the rod and sphere are infinitely far apart.
		Show that your answer reduces to the expected results when x is much larger than L. \textit{Hint: Use the power series expansion,}
		$$\ln(1+x)=x-\frac{x^2}{2} +\frac{x^3}{3} -\frac{x^4}{4} + \cdots \quad (\text{for }|x| < 1)$$
		\item[b.] Use $F_x = -\dv{U}{x}$ to find the magnitude and direction of the gravitational force exerted on the sphere by the rod. Show that your answer reduces to the expected results when $x$ is much larger than $L$.
	\end{enumerate}
	\udia{0.7}{rdf69}
	}
	\sol{}{
	\sss{a.}
	Consider a small mass element \( dm \) of the rod at a distance \( s \) from the end closest to the sphere. The gravitational potential energy due to this element is:
	\[
	dU = -\frac{G m \, dm}{r}, \quad \text{where } r = x + s.
	\]
	The linear mass density of the rod is \( \lambda = \dfrac{M}{L} \), so \( dm = \lambda \, ds = \dfrac{M}{L} \, ds \). \\ 
	Integrating over the entire rod,
	\begin{align*}
		U &= -\frac{G M m}{L} \int_0^L \frac{ds}{x + s}\\
		U &= -\frac{G M m}{L} \ln\left(\frac{x+L}{x}\right)\h
	\end{align*}
	For \( x \gg L \),
	\[
	\ln\left(\frac{x+L}{x}\right) = \ln\left(1 + \frac{L}{x}\right) \approx \frac{L}{x} + \frac{L^2}{2x^2} + \cdots.
	\]
	Thus,
	\[
	U \approx -\frac{G M m}{L} \left(\frac{L}{x}\right) = -\frac{G M m}{x},
	\]
	which matches the point-mass result.
	\sss{b.}
	The gravitational force is obtained as follows,
	\begin{align*}
		F_x &= -\frac{dU}{dx} = -\frac{d}{dx} \left(-\frac{G M m}{L} \ln\left(\frac{x+L}{x}\right)\right) = \frac{G M m}{L} \left(\frac{1}{x} - \frac{1}{x + L}\right)\imps F_x = -\frac{G M m}{x(x + L)}\h
	\end{align*} 
	For \( x \gg L \), this reduces to:
	\[
	F_x \approx -\frac{G M m}{x^2}
	\]
	consistent with Newton's law of gravitation for a point mass.\\
	}
	\wa
\end{document}