\documentclass{report}

\input{latex-templates/preamble}
\newcommand{\eps}{\epsilon}
\newcommand{\veps}{\varepsilon}
\newcommand{\Qed}{\begin{flushright}\qed\end{flushright}}

\newcommand{\parinn}{\setlength{\parindent}{1cm}}
\newcommand{\parinf}{\setlength{\parindent}{0cm}}

% \newcommand{\norm}{\|\cdot\|}
\newcommand{\inorm}{\norm_{\infty}}
\newcommand{\opensets}{\{V_{\alpha}\}_{\alpha\in I}}
\newcommand{\oset}{V_{\alpha}}
\newcommand{\opset}[1]{V_{\alpha_{#1}}}
\newcommand{\lub}{\text{lub}}
\newcommand{\del}[2]{\frac{\partial #1}{\partial #2}}
\newcommand{\Del}[3]{\frac{\partial^{#1} #2}{\partial^{#1} #3}}
\newcommand{\deld}[2]{\dfrac{\partial #1}{\partial #2}}
\newcommand{\Deld}[3]{\dfrac{\partial^{#1} #2}{\partial^{#1} #3}}
\newcommand{\der}[2]{\frac{\mathrm{d} #1}{\mathrm{d} #2}}
% \newcommand{\ddd}[3]{\frac{\mathrm{d}^{#3} #1}{\mathrm{d}^{#3} #2}}
\newcommand{\lm}{\lambda}
\newcommand{\uin}{\mathbin{\rotatebox[origin=c]{90}{$\in$}}}
\newcommand{\usubset}{\mathbin{\rotatebox[origin=c]{90}{$\subset$}}}
\newcommand{\lt}{\left}
\newcommand{\rt}{\right}
\newcommand{\bs}[1]{\boldsymbol{#1}}
\newcommand{\exs}{\exists}
\newcommand{\st}{\strut}
\newcommand{\dps}[1]{\displaystyle{#1}}
\newcommand{\id}{\text{id}}
\newcommand{\imps}{\quad \Rightarrow \quad}
\newcommand{\cimps}{\quad \Leftrightarrow \quad}
\newcommand{\kyuki}[1]{\quad \quad \bqty{\because \eqref{#1}}}
\newcommand{\kyukifir}[2]{\quad \quad \bqty{\because \eqref{#1} \& \eqref{#2}}}
\newcommand{\boxdia}[2]{\begin{wrapfigure}{r}{#1\textwidth}
		\fbox{\includegraphics[width=\linewidth]{Figures/#2.png}}
	\end{wrapfigure}}
\newcommand{\dia}[2]{\begin{wrapfigure}{r}{#1\textwidth}
		\includegraphics[width=\linewidth]{Figures/#2.png}
	\end{wrapfigure}}
\newcommand{\boxudia}[2]{\begin{figure}[H]
		\centering
		\fbox{\includegraphics[width=#1\textwidth]{Figures/#2.png}}
		\end{figure}}
\newcommand{\udia}[2]{\begin{figure}[H]
		\centering
		\includegraphics[width=#1\textwidth]{Figures/#2.png}
	\end{figure}}
\newcommand{\su}[2]{\textcolor{my#1}{#2}}
\newcommand{\shs}[1]{\\ \textbf{{\Large #1}}\\}
\newcommand{\sss}[1]{\vspace*{-1cm} \subsubsection*{#1}}
\newcommand{\unt}[1]{\text{#1}}
\newcommand{\wa}{
	\noindent\rule{\textwidth}{0.4pt} 
	\vspace{0.5cm}}
\newcommand{\wb}{\noindent\rule{\textwidth}{0.4pt}}
\newcommand{\qmi}{\int_{-\infty}^{\infty}}
\newcommand{\qmk}{|\psi(x,0)|^{2}}
\newcommand{\qml}{\exp{-\frac{(x - x_0)^2}{4\sigma_0^2} + \frac{i}{\hbar}p_0 x}}
\newcommand{\qmls}{\exp{-\frac{(x - x_0)^2}{4\sigma_0^2} - \frac{i}{\hbar}p_0 x}}
\newcommand{\e}[1]{\exp\lt(#1\rt)}
\newcommand\prm[2][^n]{\prescript{#1\mkern-2.5mu}{}P_{#2}}
\newcommand\cmb[2][^n]{\prescript{#1\mkern-0.5mu}{}C_{#2}}
\newcommand{\ki}[1]{\lt[\therefore #1\rt]}
\newcommand{\h}{\underset{\rotatebox{135}{\#}}{}}
\newcommand{\f}{\frac{1}{2}}


%\newcommand{\sol}[1]{\vspace{0.5cm} 
%\setlength{\parindent}{0cm} \textcolor{mytheoremfr}{\textbf{\underline{Solution:}}} \textcolor{mytheoremfr}{#1}}
\newcommand{\solve}[1]{\setlength{\parindent}{0cm}\textbf{\textit{Solution: }}\setlength{\parindent}{1cm}#1 \Qed}

\input{latex-templates/letterfonts}
\setlength{\parindent}{0pt}
\usepackage{physics, siunitx}
\usepackage{float}
\usepackage{hyperref}
\usepackage{wrapfig}
\usepackage{pgfplots}
\usepackage{longtable}
\setlength{\fboxsep}{4pt} % Space between image and border
\setlength{\fboxrule}{0.5pt} % Border thickness

\title{\Huge{\textbf{PC2130}}\\ \su{g}{Quantum Mechanics I} \\ {\huge \su{r}{Assignment 2}}}
\author{\huge{Parth Bhargava}\\ } %A0310667E
\date{\today}

\begin{document}
	\maketitle
	
  \pbm{}{
	Consider a quantum harmonic oscillator of mass $m$ moving in a one - dimensional world, say the $x$-axis. The quantum dynamics of the particle is governed by the \su{r}{\textit{Schrödinger Equation}},
	\begin{align*}
	    i\hbar \pdv{\psi}{t} &= -\frac{\hbar^2}{2m} \pdv[2]{\psi}{x} + \frac{1}{2} m\omega^2 x^2 \psi
	\end{align*}
	where $\psi = \psi(x, t)$ is its normalised wavefunction. Suppose at time $t = 0$,
	\begin{align*}
	    \psi(x, 0) = \frac{1}{\sqrt{\sqrt{2\pi}\sigma_0}} \exp\left(\frac{ip_0 x}{\hbar}\right) \exp\left(-\frac{(x - x_0)^2}{4\sigma_0^2}\right)
	\end{align*}
	determine $\psi(x, t)$ for time $t > 0$. Here, $\omega, p_0, \sigma_0, x_0 \in \mathbb{R}$.
	\begin{enumerate}
		\item[a.] Using Ehrenfest's theorem, determine
		\begin{enumerate}
		    \item[(i)] $\ev{x}_t$, the expectation value of the position,
		    \item[(ii)] $\ev{p}_t$, the expectation value of the linear momentum
		\end{enumerate}
		of the quantum particle at time $t > 0$.
		\item[b.] \textit{Energy eigenvalue problem}. Write down the associated \su{r}{energy eigenvalue equation} and solve it, i.e., determine the  \su{b}{energy eigenvalues} $E$ and corresponding \su{b}{energy eigenfunctions} $u_E = u_E(x)$. Note also that $E \geq 0$. (Why?)
		\item[c.] Using \su{r}{MATHEMATICA}, or otherwise, determine $\psi(x, t)$ for time $t > 0$. Note that you need to pick appropriate values for $\omega, p_0, \sigma_0, x_0, m$ and $\hbar$.
		\item[d.] Hence, simulate and describe the dependence of $|\psi(x, t)|^2$ on time $t$. Reconcile your simulation with the results in (a).
		\item[e.] Express the \su{r}{\textit{Schrödinger Equation}} for the quantum harmonic oscillator in the \su{r}{linear momentum representation}, and solve it to obtain $\phi(p, t)$ for time $t > 0$. Hence, simulate and describe the dependence of $|\phi(p, t)|^2$ on time $t$. Use the same appropriate values for $\omega, p_0, \sigma_0, x_0, m$ and $\hbar$ in (c). Reconcile your simulation with the results in (a).
		\item[f.] Compare and contrast the behaviour of $|\psi(x, t)|^2$ and $|\phi(p, t)|^2$ over time. What can you conclude?
	\end{enumerate}
	}

\wb
  \sol{}{
  \wa
  # Quantum Harmonic Oscillator in a One-Dimensional World

Consider a quantum harmonic oscillator of mass $m$ moving in a one-dimensional world, specifically along the $x$-axis. The quantum dynamics of the particle is governed by the Schrödinger equation:

\begin{align*}
i\hbar\frac{\partial\psi}{\partial t} = -\frac{\hbar^2}{2m}\frac{\partial^2\psi}{\partial x^2} + \frac{1}{2}m\omega^2 x^2 \psi
\end{align*}

where $\psi = \psi(x,t)$ is its normalized wavefunction. At time $t=0$, the wavefunction is given by:

\begin{align*}
\psi(x,0) = \frac{1}{\sqrt{\sqrt{2\pi}\sigma_0}}\exp\left(\frac{i}{\hbar}p_0 x\right)\exp\left[-\frac{1}{4\sigma_0^2}(x-x_0)^2\right]
\end{align*}

where $\omega, p_0, \sigma_0, x_0 \in \mathbb{R}$.

## (a) Using Ehrenfest's theorem to determine expectation values

### (i) Expectation value of position $\langle x \rangle_t$
### (ii) Expectation value of linear momentum $\langle p \rangle_t$

According to Ehrenfest's theorem, the expectation values of operators follow classical equations of motion. For the quantum harmonic oscillator, we have:

\begin{align*}
\frac{d}{dt}\langle x\rangle_t &= \frac{d}{dt}\langle\psi(t)|\hat{x}|\psi(t)\rangle \\
&= \frac{i}{\hbar}\langle\psi(t)|[\hat{H},\hat{x}]|\psi(t)\rangle \\
&= \frac{1}{m}\langle p\rangle_t
\end{align*}

This follows because:
\begin{align*}
i\hbar\frac{d}{dt}|\psi(t)\rangle &= \hat{H}|\psi(t)\rangle \\
[\hat{x},\hat{p}] &= i\hbar \\
[\hat{H},\hat{x}] &= -\frac{i\hbar}{m}\hat{p}
\end{align*}

Similarly, for momentum:
\begin{align*}
\frac{d}{dt}\langle p\rangle_t &= \frac{d}{dt}\langle\psi(t)|\hat{p}|\psi(t)\rangle \\
&= \frac{i}{\hbar}\langle\psi(t)|[\hat{H},\hat{p}]|\psi(t)\rangle \\
&= -m\omega^2\langle x\rangle_t
\end{align*}

since $[\hat{H},\hat{p}] = i\hbar m\omega^2\hat{x}$.

Combining these equations, we get:
\begin{align*}
\frac{d^2}{dt^2}\langle x\rangle_t + \omega^2\langle x\rangle_t = 0
\end{align*}

This is the differential equation for simple harmonic motion. The general solution is:
\begin{align*}
\langle x\rangle_t &= \langle x\rangle_0\cos\omega t + \frac{1}{m\omega}\langle p\rangle_0\sin\omega t \\
&= x_0\cos\omega t + \frac{p_0}{m\omega}\sin\omega t
\end{align*}

And for the momentum:
\begin{align*}
\langle p\rangle_t &= m\frac{d}{dt}\langle x\rangle_t \\
&= p_0\cos\omega t - m\omega x_0\sin\omega t
\end{align*}

## (b) Energy eigenvalue problem


## (c) Determining $\psi(x,t)$ for time $t > 0$


## (d) Simulation and description of $|\psi(x,t)|^2$ over time

The probability density $|\psi(x,t)|^2$ represents the probability of finding the particle at position $x$ at time $t$. For the quantum harmonic oscillator with the given initial state, several key observations can be made:

1. The center of the probability distribution follows the classical trajectory:
   \begin{align*}
   \langle x \rangle_t = x_0\cos(\omega t) + \frac{p_0}{m\omega}\sin(\omega t)
   \end{align*}

2. The distribution maintains a coherent Gaussian-like form for all $t \geq 0$.

3. The distribution peaks at the stationary points of the classical simple harmonic oscillator (as per Ehrenfest's theorem).

4. The width of the distribution may oscillate over time, reflecting the quantum nature of the system.

These observations are consistent with the results obtained in part (a) using Ehrenfest's theorem.

## (e) Schrödinger equation in momentum representation



## (f) Comparison of $|\psi(x,t)|^2$ and $|\phi(p,t)|^2$ over time


\section*{(e) Schrödinger Equation in Momentum Representation}

The Schrödinger equation in momentum space is derived by representing position as a differential operator. In momentum space:
\[
\hat{x}_{op} = i\hbar\frac{\partial}{\partial p}, \quad \hat{p}_{op} = p
\]
Substituting into the Hamiltonian:
\[
\hat{H} = \frac{p^2}{2m} + \frac{1}{2}m\omega^2\left(i\hbar\frac{\partial}{\partial p}\right)^2
\]
Simplifying the potential term:
\[
\frac{1}{2}m\omega^2\left(-\hbar^2\frac{\partial^2}{\partial p^2}\right) = -\frac{1}{2}m\omega^2\hbar^2\frac{\partial^2}{\partial p^2}
\]
This gives the momentum-space Schrödinger equation:
\begin{align*}
i\hbar\frac{\partial}{\partial t}\phi(p,t) = \left[\frac{p^2}{2m} - \frac{1}{2}m\omega^2\hbar^2\frac{\partial^2}{\partial p^2}\right]\phi(p,t)
\end{align*}

The eigenvalue equation \(\hat{H}v_n(p) = E_nv_n(p)\) is solved using ladder operators adapted for momentum space:
\[
\hat{a}_p = \sqrt{\frac{\hbar m\omega}{2}}\frac{\partial}{\partial p} + \frac{p}{\sqrt{2\hbar m\omega}}, \quad \hat{a}_p^\dagger = -\sqrt{\frac{\hbar m\omega}{2}}\frac{\partial}{\partial p} + \frac{p}{\sqrt{2\hbar m\omega}}
\]
These satisfy \([\hat{a}_p, \hat{a}_p^\dagger] = 1\). The ground state \(v_0(p)\) satisfies:
\[
\hat{a}_pv_0(p) = 0 \implies \left(\sqrt{\frac{\hbar m\omega}{2}}\frac{\partial}{\partial p} + \frac{p}{\sqrt{2\hbar m\omega}}\right)v_0(p) = 0
\]
Solving this differential equation:
\[
\frac{\partial v_0}{\partial p} = -\frac{p}{\hbar m\omega}v_0 \implies v_0(p) = A\exp\left(-\frac{p^2}{2\hbar m\omega}\right)
\]
Normalization gives:
\[
v_0(p) = \left(\frac{1}{\pi\hbar m\omega}\right)^{1/4}\exp\left(-\frac{p^2}{2\hbar m\omega}\right)
\]

Higher states are generated using the raising operator:
\[
v_n(p) = \frac{1}{\sqrt{n!}}(\hat{a}_p^\dagger)^n v_0(p) = \frac{1}{\sqrt{2^n n!}}\left(\frac{1}{\pi\hbar m\omega}\right)^{1/4}H_n\left(\frac{p}{\sqrt{\hbar m\omega}}\right)\exp\left(-\frac{p^2}{2\hbar m\omega}\right)
\]

The general solution combines eigenstates with time-dependent phases:
\[
\phi(p,t) = \sum_{n=0}^\infty \langle v_n|\phi(0)\rangle \exp\left[-i\left(n+\frac{1}{2}\right)\omega t\right]v_n(p)
\]
The expansion coefficients are determined by the initial condition:
\[
\langle v_n|\phi(0)\rangle = \int_{-\infty}^\infty v_n^*(p)\phi(p,0)dp
\]
For our Gaussian initial state \(\psi(x,0)\), the momentum-space initial condition is:
\[
\phi(p,0) = \frac{1}{\sqrt{\sqrt{2\pi}\sigma_p}}\exp\left(-\frac{i}{\hbar}x_0p\right)\exp\left[-\frac{1}{4\sigma_p^2}(p-p_0)^2\right]
\]
where \(\sigma_p = \hbar/(2\sigma_0)\). The time-evolved wavefunction becomes:
\[
\phi(p,t) = e^{-i\omega t/2}\sum_{n=0}^\infty \frac{1}{\sqrt{n!}}\left(\frac{\alpha}{\sqrt{2}}\right)^n H_n\left(\frac{p-p_{cl}(t)}{\sqrt{\hbar m\omega}}\right)\exp\left[-\frac{(p-p_{cl}(t))^2}{2\hbar m\omega}\right]
\]
where \(p_{cl}(t) = p_0\cos\omega t - m\omega x_0\sin\omega t\) follows the classical momentum trajectory, and \(\alpha\) contains initial displacement parameters.

The momentum expectation value evolves as:
\[
\langle p\rangle_t = p_0\cos\omega t - m\omega x_0\sin\omega t
\]
which matches the classical equation from part (a).

  \wa
  }
\end{document}
