\documentclass[12pt]{article}

\title{\Large{The Detrimental Reign of Fast Fashion: An Environmental and Social Menace}}
\author{\large{Parth Bhargava}\\ A0310667E }
\date{}

\begin{document}
	\maketitle
	
	The fast fashion industry, celebrated for its affordability and rapid trend turnover, has entrenched itself as a global economic force. However, its environmental devastation, exploitation of labour, and unsustainable economic practices overwhelmingly justify the stance that fast fashion’s harms far outweigh its fleeting benefits.
	
	Primarily, fast fashion’s environmental toll is catastrophic. The industry is the second-largest polluter globally, responsible for 8\% of carbon emissions and 20\% of wastewater (Bailey et al., 2022). Drennan (2015, as cited in Williams, 2022) underscores that producing a single cotton t-shirt consumes 2,700 litres of water—enough for one person’s needs for 900 days—while toxic chemicals from dyeing processes contaminate waterways, rendering them unsafe for ecosystems and human use. For instance, the United Nations estimates that 93 billion cubic meters of water are used annually by the fashion industry, with textile dyeing alone polluting 17–20\% of global industrial wastewater (Williams, 2022). The Le (2020) further reveals that 35\% of ocean microplastics originate from synthetic textiles like polyester, which degrade into carcinogenic particles ingested by marine life and humans. Such irreversible ecological consequences demonstrate that environmental sustainability is incompatible with fast fashion’s current model.
	
	Moreover, fast fashion relies on exploitative labour practices, particularly in developing nations where workers endure hazardous conditions for poverty wages. For example, the 2013 Rana Plaza factory collapse in Bangladesh, which killed 1,132 garment workers, exposed systemic negligence—factories operated in structurally unsafe buildings to meet aggressive production deadlines (Lin, 2022). Child labour remains rampant, with underage workers facing physical abuse and denied education, perpetuating cycles of poverty (Williams, 2022). In Ethiopia, garment workers are paid as little as \$23.70 monthly, struggling to afford basic necessities like soap and transportation (Williams, 2022). Psychologically, Lin (2022) argues that fast fashion preys on consumer insecurities through social media, fostering overconsumption via “haul culture” and FOMO. This manipulation normalizes a throwaway mentality that devalues human dignity and environmental welfare.
	
	Proponents like Buchanan (2023) argue that fast fashion drives economic growth in developing nations, citing job creation and infrastructure development. However, this growth is illusory. While Crasnitchi (2023) acknowledges that outsourcing boosts GDP in developing countries, she warns that these gains often mask systemic issues of reliance on low-wage labour and insufficient environmental safeguards. For instance, despite promises of economic progress, 35\% of Bangladesh’s population still lives below the poverty line, with garment workers’ wages failing to cover nutritional needs (Lambert, 2014 as cited in Williams, 2022). Moreover, the industry’s shift to African nations for cheaper labour and lax oversight (Williams, 2022) reveals a pattern of exploitation, not progress. Economic “benefits” are fleeting, overshadowed by long-term costs like healthcare crises from pollution and worker exploitation. True development requires equitable wages and sustainable practices—neither of which fast fashion prioritizes.
	
	In conclusion, fast fashion’s environmental recklessness, human rights violations, and unsustainable economic practices render it a net negative force. While it offers temporary affordability, the industry’s true cost—ecosystem collapse, exploited labour, and psychological manipulation—demands urgent systemic change.

	\\
  \\
	\\
	Word Count: 494
\end{document}
