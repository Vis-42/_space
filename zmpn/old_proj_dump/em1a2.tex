\documentclass{report}

\input{latex-templates/preamble}
\newcommand{\eps}{\epsilon}
\newcommand{\veps}{\varepsilon}
\newcommand{\Qed}{\begin{flushright}\qed\end{flushright}}

\newcommand{\parinn}{\setlength{\parindent}{1cm}}
\newcommand{\parinf}{\setlength{\parindent}{0cm}}

% \newcommand{\norm}{\|\cdot\|}
\newcommand{\inorm}{\norm_{\infty}}
\newcommand{\opensets}{\{V_{\alpha}\}_{\alpha\in I}}
\newcommand{\oset}{V_{\alpha}}
\newcommand{\opset}[1]{V_{\alpha_{#1}}}
\newcommand{\lub}{\text{lub}}
\newcommand{\del}[2]{\frac{\partial #1}{\partial #2}}
\newcommand{\Del}[3]{\frac{\partial^{#1} #2}{\partial^{#1} #3}}
\newcommand{\deld}[2]{\dfrac{\partial #1}{\partial #2}}
\newcommand{\Deld}[3]{\dfrac{\partial^{#1} #2}{\partial^{#1} #3}}
\newcommand{\der}[2]{\frac{\mathrm{d} #1}{\mathrm{d} #2}}
% \newcommand{\ddd}[3]{\frac{\mathrm{d}^{#3} #1}{\mathrm{d}^{#3} #2}}
\newcommand{\lm}{\lambda}
\newcommand{\uin}{\mathbin{\rotatebox[origin=c]{90}{$\in$}}}
\newcommand{\usubset}{\mathbin{\rotatebox[origin=c]{90}{$\subset$}}}
\newcommand{\lt}{\left}
\newcommand{\rt}{\right}
\newcommand{\bs}[1]{\boldsymbol{#1}}
\newcommand{\exs}{\exists}
\newcommand{\st}{\strut}
\newcommand{\dps}[1]{\displaystyle{#1}}
\newcommand{\id}{\text{id}}
\newcommand{\imps}{\quad \Rightarrow \quad}
\newcommand{\cimps}{\quad \Leftrightarrow \quad}
\newcommand{\kyuki}[1]{\quad \quad \bqty{\because \eqref{#1}}}
\newcommand{\kyukifir}[2]{\quad \quad \bqty{\because \eqref{#1} \& \eqref{#2}}}
\newcommand{\boxdia}[2]{\begin{wrapfigure}{r}{#1\textwidth}
		\fbox{\includegraphics[width=\linewidth]{Figures/#2.png}}
	\end{wrapfigure}}
\newcommand{\dia}[2]{\begin{wrapfigure}{r}{#1\textwidth}
		\includegraphics[width=\linewidth]{Figures/#2.png}
	\end{wrapfigure}}
\newcommand{\boxudia}[2]{\begin{figure}[H]
		\centering
		\fbox{\includegraphics[width=#1\textwidth]{Figures/#2.png}}
		\end{figure}}
\newcommand{\udia}[2]{\begin{figure}[H]
		\centering
		\includegraphics[width=#1\textwidth]{Figures/#2.png}
	\end{figure}}
\newcommand{\su}[2]{\textcolor{my#1}{#2}}
\newcommand{\shs}[1]{\\ \textbf{{\Large #1}}\\}
\newcommand{\sss}[1]{\vspace*{-1cm} \subsubsection*{#1}}
\newcommand{\unt}[1]{\text{#1}}
\newcommand{\wa}{
	\noindent\rule{\textwidth}{0.4pt} 
	\vspace{0.5cm}}
\newcommand{\wb}{\noindent\rule{\textwidth}{0.4pt}}
\newcommand{\qmi}{\int_{-\infty}^{\infty}}
\newcommand{\qmk}{|\psi(x,0)|^{2}}
\newcommand{\qml}{\exp{-\frac{(x - x_0)^2}{4\sigma_0^2} + \frac{i}{\hbar}p_0 x}}
\newcommand{\qmls}{\exp{-\frac{(x - x_0)^2}{4\sigma_0^2} - \frac{i}{\hbar}p_0 x}}
\newcommand{\e}[1]{\exp\lt(#1\rt)}
\newcommand\prm[2][^n]{\prescript{#1\mkern-2.5mu}{}P_{#2}}
\newcommand\cmb[2][^n]{\prescript{#1\mkern-0.5mu}{}C_{#2}}
\newcommand{\ki}[1]{\lt[\therefore #1\rt]}
\newcommand{\h}{\underset{\rotatebox{135}{\#}}{}}
\newcommand{\f}{\frac{1}{2}}


%\newcommand{\sol}[1]{\vspace{0.5cm} 
%\setlength{\parindent}{0cm} \textcolor{mytheoremfr}{\textbf{\underline{Solution:}}} \textcolor{mytheoremfr}{#1}}
\newcommand{\solve}[1]{\setlength{\parindent}{0cm}\textbf{\textit{Solution: }}\setlength{\parindent}{1cm}#1 \Qed}

\input{latex-templates/letterfonts}
\usepackage{physics, siunitx}
\usepackage{float}
\usepackage{hyperref}
\usepackage{wrapfig}
\usepackage{pgfplots}
\setlength{\fboxsep}{4pt} % Space between image and border
\setlength{\fboxrule}{0.5pt} % Border thickness

\title{\Huge{PC2031 Electricity and Magnetism 1}\\Assignment 2}
\author{\huge{Parth Bhargava}\\ AO310667E}
\date{}

\begin{document}
	\maketitle
	
	\pbm{}{
	\begin{enumerate}
		\item[a.] Consider a circular loop of radius $R$ and uniform line charge density $\lm$ lying in the $xy$-plane.
		\udia{0.25}{rdf38}
		The centre of the loop is the origin. Find the electric field at an arbitrary point along the $z$-axis.
		\item[b.] Repeat the calculations for a circular disk of radius $R$ and uniform
		surface charge density $\sigma$.
		\udia{0.3}{rdf39}
		Check your answers for the limiting cases $R \to \infty$ and $z \gg R$.
	\end{enumerate}
	
	
	}
	\sol{}{
	\sss{(a)}
	\dia{0.4}{rdf41}
	For an elemental segment of the loop,
	$$\dd Q=\lm R \dd \phi$$
	Due to which,
	$$\dd E=\frac{k\dd Q}{R^2\csc^2\theta}$$
	By symmetry, it can be established that the contribution of $\dd E \sin\theta$ will cancel out as $\phi$ goes from 0 to $2\pi$ and the total electric field at P will point along the $z$-axis.\\
	Consequently,
	\begin{align*}
		|\vb{E}|&=\int_0^{2\pi}\frac{\dd E}{\dd \phi}\cos\theta  \dd \phi \\
		&=\frac{k\lm R}{R^2\csc^2\theta}\cos\theta \qty(\int_0^{2\pi}\dd \phi) \\
		&=\frac{2k\pi\lm \sin^2\theta}{R}\cos\theta  \\
		&=\frac{2k\pi\lm \qty(\frac{R^2}{z^2+R^2})}{R}\frac{z}{\sqrt{z^2+R^2}} \\
	\end{align*}
	$$\imps \boxed{\vb{E}=\frac{1}{4\pi\eps}\frac{Qz}{\pqty{z^2+R^2}^{3/2}} \vu*{z}}\h$$
	where $Q=2\pi R \lm$ is the total charge on the loop.\\
	\wa
	\sss{(b)}
	\dia{0.4}{rdf42}
	For an elemental loop in the disc,
	$$\dd Q = \sigma \cdot 2\pi r \dd r$$
	Due to which,\textit{(in reference to the result in part (a))}
	$$\dd E=\frac{1}{4\pi\eps}\frac{z \dd Q}{\pqty{z^2+r^2}^{3/2}}=\frac{kz \dd Q}{\pqty{z^2+r^2}^{3/2}}$$
	By symmetry, it can be established that each contribution of $\dd E$ and subsequently the total electric field at P will point along the $z$-axis.\\
	Hence,
	\begin{align*}
		|\vb{E}|&=\int_0^{R}\frac{\dd E}{\dd r} \dd r \\
		&=\int_0^{R}\frac{kz \sigma \cdot 2\pi r}{\pqty{z^2+r^2}^{3/2}} \dd r \\
		&= \sigma \pi k z \int_0^{R} \frac{2r}{\pqty{z^2+r^2}^{3/2}} \dd r \\
		&= \sigma \pi k z \int_0^{R} \frac{1}{\pqty{z^2+r^2}^{3/2}} \dd{(z^2+r^2)} \\
		&= \sigma \pi k z \qty[-\frac{2}{\pqty{z^2+r^2}^{1/2}}]_0^R \\
		&= \frac{1}{4 \pi \eps } 2\sigma \pi z \qty[-\frac{1}{\pqty{z^2+R^2}^{1/2}}+\frac{1}{z}] 
	\end{align*}
	$$\imps \boxed{\vb{E}=\frac{\sigma}{2\eps} \pqty{1-\frac{z}{\pqty{z^2+R^2}^{1/2}}} \vu*{z}}\h$$
	\sss{For the limiting cases,}
	\begin{enumerate}
		\item For $R \to \infty$,
		$$\frac{z}{\pqty{z^2+R^2}^{1/2}} \approx 0 \imps \vb{E}=\frac{\sigma}{2\eps} \vu*{z}$$
		which is equivalent to the electric field due to a charged infinite sheet. \\Additionally, it is important to note that the electric field attains a constant value at all points in space.
		\item For $z \gg R$,
		\begin{align*}
			&\frac{z}{\pqty{z^2+R^2}^{1/2}}= \frac{1}{\pqty{1+\qty(\dfrac{R}{z})^2}^{1/2}} \approx \frac{1}{\pqty{1+\dfrac{1}{2}\qty(\dfrac{R}{z})^2}}= \frac{2z^2}{\pqty{2z^2+R^2}}\\
			\imps &\vb{E}=\frac{\sigma}{2\eps} \pqty{1-\frac{2z^2}{\pqty{2z^2+R^2}}} \vu*{z}\\
			&\vb{E}=\frac{\sigma}{2\eps} \pqty{\frac{R^2}{2z^2+R^2}} \vu*{z}\\
			&\vb{E}=\frac{1}{4\pi\eps} \pqty{\frac{\pi R^2 \sigma}{z^2}} \vu*{z} && \lt[\because z \gg R \to 2z^2+R^2 \approx 2z^2 \rt]\\
			\imps &\vb{E}=\frac{kQ}{z^2} \vu*{z} && \lt[ \because Q=\pi R^2 \sigma \rt]
		\end{align*}
		which is equivalent to the electric field due to a point charge $Q=\pi R^2 \sigma $ at a distance $z$.
	\end{enumerate}
	\wa
	}
	
	\pbm{}{
	Four charged spheres are connected by strings of equal length.
	\udia{0.4}{rdf40}
	The system is in equilibrium. Determine $\tan\theta$ and express it in terms of $Q$ and $q$.
	}
	\sol{}{
	\sss{On charge Q,}
	By Coulomb's Law,
	\begin{align*}
		F_1=\frac{kQ^2}{4x^2\cos^2\theta}+\frac{2kQq}{x^2}\cos\theta
	\end{align*}
	By Lami's Theorem,
	\begin{align*}
		\frac{F_1}{\sin2\theta}=\frac{T}{\sin(\pi-\theta)}\imps
		F_1\sin\theta=T(2\sin\theta\cos\theta)\imps
		\cos\theta=\frac{F_1}{2T} \tag{1} \label{1}
	\end{align*}
	\udia{0.7}{rdf43}
	\sss{On charge q,}
	By Coulomb's Law,
	\begin{align*}
		F_2=\frac{kq^2}{4x^2\sin^2\theta}+\frac{2kQq}{x^2}\sin\theta
	\end{align*}
	By Lami's Theorem,
	\begin{align*}
		\frac{F_2}{\sin(\pi-2\theta)}=\frac{T}{\sin(\frac{\pi}{2}+\theta)}\imps
		F_2\cos\theta=T(2\sin\theta\cos\theta)\imps
		\sin\theta=\frac{F_2}{2T} \tag{2} \label{2}
	\end{align*}
	\sss{By \eqref{1} and \eqref{2},}
	\begin{align*}
		\frac{F_1}{\cos\theta}&=\frac{F_2}{\sin\theta}\\
		\frac{kQ^2}{4x^2\cos^3\theta}+\frac{2kQq}{x^2}&=\frac{kq^2}{4x^2\sin^3\theta}+\frac{2kQq}{x^2}\\
		\frac{Q^2}{\cos^3\theta}&=\frac{q^2}{\sin^3\theta}\\
		\frac{\sin^3\theta}{\cos^3\theta}&=\frac{q^2}{Q^2}
	\end{align*}
	$$\imps \boxed{\tan\theta=\qty(\frac{q^2}{Q^2})^{1/3}}\h$$
	}
	\wa
	
	\pbm{}{
	A very long cylindrical rod with a radius of 4.50 cm carries a uniformly distributed positive charge density $\lm= 17.00$ nC per centimeter. A Cartesian coordinate system is established such that the center of the rod is at $(x,y)=(0,0)$. A cylindrical cavity with a radius of 2.15 cm is drilled through the rod, with its center positioned at $(x,y)= (0.50, 1.50)$ cm. Determine the electric field (in N/C) at the point $(x,y)= (-2.85, 3.28)$ cm.
	}
	\sol{}{
	
	\udia{0.7}{rdf49}
	\sss{Electric field inside an infinitely long charged cylinder}
	Given an infinitely long cylinder of radius \(R\) with a uniform volume charge density \(\rho\) (in C/m\(^3\)), consider a point inside the cylinder at a radial distance \(r\) (with \(r < R\)).\\
	 Owing to the cylindrical symmetry of the configuration, the electric field is radial, written as \(\vb{E} = E(r) \vu*{r}\), and depends solely on \(r\) without any variation along the axial (\(z\)) or angular (\(\theta\)) directions. \\
	 To analyze the field using Gauss’s law, a cylindrical Gaussian surface coaxial with the charged cylinder is chosen, having a radius \(r\) and a length \(L\).
	The total flux through the Gaussian surface is:
	\begin{align*}
		\oint \vb{E} \cdot \dd{\vb{A}} &= \frac{Q_{\text{enc}}}{\eps_0} \imps
		\underbrace{E(r) \cdot 2\pi r L}_{\text{Curved surface}} + \underbrace{0 + 0}_{\text{End caps}} = \frac{\rho \cdot \pi r^2 L}{\eps_0} && [ \because Q_{\text{enc}} = \rho V = \rho \pi r^2 L ]\\ \\
		\imps E(r) &= \frac{\rho \pi r^2 L}{2\pi r L \eps_0} =\frac{\rho r}{2\eps_0} \imps \boxed{\vb{E} = \frac{\rho r}{2\eps_0} \vu*{r}}
	\end{align*}
	
	\sss{Electric Field outside an infinitely long charged cylinder}
	Given an infinitely long cylinder of radius \(R\) with a uniform volume charge density \(\rho\) (in C/m\(^3\)), consider a point outside the cylinder at a radial distance \(r\) (with \(r > R\)). \\
	Owing to the cylindrical symmetry of the configuration, the electric field is radial, written as \(\vb{E} = E(r) \vu*{r}\), and depends solely on \(r\) without any variation along the axial (\(z\)) or angular (\(\theta\)) directions.\\
	To analyze the field using Gauss’s law, a cylindrical Gaussian surface coaxial with the charged cylinder is chosen, having a radius \(r\) and a length \(L\). \\
	The total flux through the Gaussian surface is: 
	\begin{align*}
		\oint \vb{E} \cdot \dd{\vb{A}} &= \frac{Q_{\text{enc}}}{\eps_0} \imps
		\underbrace{E(r) \cdot 2\pi r L}_{\text{Curved surface}} + \underbrace{0 + 0}_{\text{End caps}} = \frac{\rho \cdot \pi R^2 L}{\eps_0} && [ \because Q_{\text{enc}} = \rho V = \rho \pi R^2 L ]\\ \\
		\imps E(r) &= \frac{\rho \pi R^2 L}{2\pi r L \eps_0} =\frac{\rho r}{2\eps_0} \imps \boxed{\vb{E} = \frac{\rho R^2}{2 \eps_0 r} \vu*{r}}
	\end{align*}
	
	\sss{Solving the cavity problem}
	\udia{0.7}{rdf48}
	Now in the case of the rod under examination, to determine the electric field at point $P$, we use the superposition principle. Instead of directly analyzing the field due to the rod with a cavity, we model the cavity as a superposition of charge distributions:
	\begin{enumerate}
		\item Original Rod: Assume the rod is uniformly charged with volume charge density $\rho$.
		\item Negative Charge Equivalent of the Cavity: The cavity can be treated as if it were filled with a uniform charge density $\rho$, and then we superimpose an equal and opposite (negative) charge density in the same region to effectively create an empty cavity.
	\end{enumerate}
	Without taking the cavity under consideration, $\rho = \dfrac{\lm}{\pi R^2}$ where $\lm$ is the linear charge density.\\
	At point P,
	\begin{align*}
		\vb{E}_{\text{total}} &= \vb{E}_{\text{rod}} + \vb{E}_{\text{cavity}} \\
		\vb{E}_{\text{total}} &= \frac{\rho}{2 \eps_0} \lt( x_p \vu*{i} + y_p \vu*{j} \rt) -\frac{\rho a^2}{2 \eps_0 (l)^2} \lt( l_x \vu*{i} + l_y \vu*{j} \rt)\\
		\vb{E}_{\text{total}} &=\frac{\lm}{2 \eps_0 \pi R^2} \lt( x_p \vu*{i} + y_p \vu*{j} \rt) -\frac{\lm a^2}{2 \eps_0 \pi R^2 (l)^2} \lt( l_x \vu*{i} + l_y \vu*{j} \rt) \\
		&= \frac{\lm}{2 \eps_0 \pi R^2} \Bigg[ \lt( x_p \vu*{i} + y_p \vu*{j} \rt) - \frac{a^2}{l^2} \lt( l_x \vu*{i} + l_y \vu*{j} \rt) \Bigg]\\
		\vb{E}_{\text{total}} &= \frac{\lm}{2 \eps_0 \pi R^2} \qty(\qty[ x_p - \frac{a^2 l_x}{l^2} ] \vu*{i} + \qty[ y_p - \frac{a^2 l_y}{l^2} ] \vu*{j})
	\end{align*}
	Given numerical values:
	\begin{align*}
		R &= \SI{0.045}{m} &&a = \SI{0.0215}{m} && x_p = \SI{-0.0285}{m},\\
		y_p &= \SI{0.0328}{m} &&x_c = \SI{0.005}{m} &&y_c = \SI{0.015}{m}\\
		l_x &= -0.0285 - 0.005 = \SI{-0.0335}{m} &&l_y = 0.0328 - 0.015 = \SI{0.0178}{m} &&\lm = \SI{1.70e-6}{C/m}  
	\end{align*}
	We get,
	\begin{align*}
		\vb{E}_{\text{total}} = \frac{1.70 \times 10^{-6}}{2 \times 8.85 \times 10^{-12} \times \pi \times 0.045^2} 
		\qty(
			\qty[ -0.0285 - \frac{0.0215^2 \times (-0.0335)}{(-0.0335)^2 + 0.0178^2} ] \vu*{i} 
			+ \qty[ 0.0328 - \frac{0.0215^2 \times 0.0178}{(-0.0335)^2 + 0.0178^2} ] \vu*{j}
		)
	\end{align*}
	$$
	\imps \boxed{\vb{E}_{\text{total}}=(-2.68 \times 10^5 \vu*{i} + 4.09 \times 10^5 \vu*{j})~\mathrm{N/C}}\h$$
	\wa
	}
\end{document}