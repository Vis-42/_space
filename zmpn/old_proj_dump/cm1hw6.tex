\documentclass{report}

\input{latex-templates/preamble}
\newcommand{\eps}{\epsilon}
\newcommand{\veps}{\varepsilon}
\newcommand{\Qed}{\begin{flushright}\qed\end{flushright}}

\newcommand{\parinn}{\setlength{\parindent}{1cm}}
\newcommand{\parinf}{\setlength{\parindent}{0cm}}

% \newcommand{\norm}{\|\cdot\|}
\newcommand{\inorm}{\norm_{\infty}}
\newcommand{\opensets}{\{V_{\alpha}\}_{\alpha\in I}}
\newcommand{\oset}{V_{\alpha}}
\newcommand{\opset}[1]{V_{\alpha_{#1}}}
\newcommand{\lub}{\text{lub}}
\newcommand{\del}[2]{\frac{\partial #1}{\partial #2}}
\newcommand{\Del}[3]{\frac{\partial^{#1} #2}{\partial^{#1} #3}}
\newcommand{\deld}[2]{\dfrac{\partial #1}{\partial #2}}
\newcommand{\Deld}[3]{\dfrac{\partial^{#1} #2}{\partial^{#1} #3}}
\newcommand{\der}[2]{\frac{\mathrm{d} #1}{\mathrm{d} #2}}
% \newcommand{\ddd}[3]{\frac{\mathrm{d}^{#3} #1}{\mathrm{d}^{#3} #2}}
\newcommand{\lm}{\lambda}
\newcommand{\uin}{\mathbin{\rotatebox[origin=c]{90}{$\in$}}}
\newcommand{\usubset}{\mathbin{\rotatebox[origin=c]{90}{$\subset$}}}
\newcommand{\lt}{\left}
\newcommand{\rt}{\right}
\newcommand{\bs}[1]{\boldsymbol{#1}}
\newcommand{\exs}{\exists}
\newcommand{\st}{\strut}
\newcommand{\dps}[1]{\displaystyle{#1}}
\newcommand{\id}{\text{id}}
\newcommand{\imps}{\quad \Rightarrow \quad}
\newcommand{\cimps}{\quad \Leftrightarrow \quad}
\newcommand{\kyuki}[1]{\quad \quad \bqty{\because \eqref{#1}}}
\newcommand{\kyukifir}[2]{\quad \quad \bqty{\because \eqref{#1} \& \eqref{#2}}}
\newcommand{\boxdia}[2]{\begin{wrapfigure}{r}{#1\textwidth}
		\fbox{\includegraphics[width=\linewidth]{Figures/#2.png}}
	\end{wrapfigure}}
\newcommand{\dia}[2]{\begin{wrapfigure}{r}{#1\textwidth}
		\includegraphics[width=\linewidth]{Figures/#2.png}
	\end{wrapfigure}}
\newcommand{\boxudia}[2]{\begin{figure}[H]
		\centering
		\fbox{\includegraphics[width=#1\textwidth]{Figures/#2.png}}
		\end{figure}}
\newcommand{\udia}[2]{\begin{figure}[H]
		\centering
		\includegraphics[width=#1\textwidth]{Figures/#2.png}
	\end{figure}}
\newcommand{\su}[2]{\textcolor{my#1}{#2}}
\newcommand{\shs}[1]{\\ \textbf{{\Large #1}}\\}
\newcommand{\sss}[1]{\vspace*{-1cm} \subsubsection*{#1}}
\newcommand{\unt}[1]{\text{#1}}
\newcommand{\wa}{
	\noindent\rule{\textwidth}{0.4pt} 
	\vspace{0.5cm}}
\newcommand{\wb}{\noindent\rule{\textwidth}{0.4pt}}
\newcommand{\qmi}{\int_{-\infty}^{\infty}}
\newcommand{\qmk}{|\psi(x,0)|^{2}}
\newcommand{\qml}{\exp{-\frac{(x - x_0)^2}{4\sigma_0^2} + \frac{i}{\hbar}p_0 x}}
\newcommand{\qmls}{\exp{-\frac{(x - x_0)^2}{4\sigma_0^2} - \frac{i}{\hbar}p_0 x}}
\newcommand{\e}[1]{\exp\lt(#1\rt)}
\newcommand\prm[2][^n]{\prescript{#1\mkern-2.5mu}{}P_{#2}}
\newcommand\cmb[2][^n]{\prescript{#1\mkern-0.5mu}{}C_{#2}}
\newcommand{\ki}[1]{\lt[\therefore #1\rt]}
\newcommand{\h}{\underset{\rotatebox{135}{\#}}{}}
\newcommand{\f}{\frac{1}{2}}


%\newcommand{\sol}[1]{\vspace{0.5cm} 
%\setlength{\parindent}{0cm} \textcolor{mytheoremfr}{\textbf{\underline{Solution:}}} \textcolor{mytheoremfr}{#1}}
\newcommand{\solve}[1]{\setlength{\parindent}{0cm}\textbf{\textit{Solution: }}\setlength{\parindent}{1cm}#1 \Qed}

\input{latex-templates/letterfonts}
\usepackage{multicol}
\usepackage{physics}
\usepackage{float}
\usepackage{hyperref}
\usepackage{wrapfig}
\usepackage{pgfplots}

\setlength{\fboxsep}{1pt} % Space between image and border
\setlength{\fboxrule}{0.5pt} % Border thickness

\setlength{\columnsep}{20pt} % Adjust space between columns
\setlength{\columnseprule}{1pt}% Thickness of vertical line

\title{\Huge{PC2032 Classical Mechanics 1}\\Homework Assignment 6}
\author{\huge{Parth Bhargava}\\ AO310667E}
\date{}

\begin{document}
	\maketitle
	
	\pbm{}{
	There are $N$ men with equal mass $m$ standing on a cart with mass $M$ . The
	cart is initially resting on a frictionless floor. \\
	\begin{enumerate}
		\item[a.] If all $N$ men jump off the cart at the same time with a horizontal speed $v_r$ in the
		same direction \textit{relative} to the cart, what is the final speed of the cart?\\
		\item[b.] If these men jump off the cart one by one with the same horizontal speed $v_r$ in
		the same direction \textit{relative} to the cart, what is the final speed of the cart?\\
		\item[c.] In the above two cases, which cart speed is higher? Why?\\
	\end{enumerate}
	}
	\sol{}{
	\sss{a.}
	By conservation of momentum,
	\begin{align*}
		0 &= Mu - Nm (v_r - u) \\
		u &= \frac{Nm}{M} \lt( v_r - u \rt) \\
		u \lt( 1 + \frac{Nm}{M} \rt) &= \frac{Nm}{M} v_r \\
		\imps u &= \frac{Nm v_r}{\lt( M + Nm \rt)}\h
	\end{align*}
	
	\sss{b.}
	After the $1^{st}$ jump,
	\begin{align*}
		0 &= (M + (N-1)m) u_1 - m (v_r - u_1) \\
		0 &= (M + N m) u_1 - m v_r \\
		u_1 &= \frac{m v_r}{M + N m} \label{01} \tag{1}
	\end{align*}
	If we put ourselves in the reference frame of the cart after the $(k-1)^{th}$ jump and ignore the $(k-1)$ people that have already jumped, we can apply \eqref{01} for the $k^{th}$ jump,
	\begin{align*}
		u_k - u_{k-1}=\frac{mv_r}{M+(N-k+1)m} \label{02} \tag{2}
	\end{align*}
	Using \eqref{02},
	\begin{align*}
		u_k &=\frac{mv_r}{M+(N-k+1)m} + u_{k-1}\\
		u_k &=\frac{mv_r}{M+(N-k+1)m} + \frac{mv_r}{M+(N-(k-1)+1)m} + u_{k-2} &&\kyuki{02}\\
		u_k &=\sum_{p=1}^N \frac{mv_r}{M+(N-(k-p))m} &&\kyuki{02}
	\end{align*}
	For $k=N$,
	$$
	u_N =\sum_{p=1}^N \frac{mv_r}{M+pm} \h
	$$
	\udia{0.7}{rdf59}
	\sss{c.}
	The final speed of the cart is higher in case (b). This is because each subsequent jump occurs when the cart is lighter, leading to a larger velocity increment per jump. The cumulative effect of these increments exceeds the single velocity change when all jump together.\\
	}
	\wa
	
	\pbm{}{
	A raindrop starts falling freely with an initial mass of $m_0$. During its descent,
	water vapor condenses onto it at a constant rate $k$. Neglecting air resistance, determine
	the motion of the raindrop by answering the following questions:\\
	\begin{enumerate}
		\item[a.] Find the velocity of the raindrop as a function of time.\\
		\item[b.] Find the distance fallen by the raindrop after a time $t$.\\
	\end{enumerate}
	}
	\sol{}{
	\sss{a.}
	Applying Newton's $2^{nd}$ Law:
	\begin{align*}
		mg&=\dv{t} (mv) = v \dv{m}{t} + m \dv{v}{t} = vk + m \dv{v}{t}\\
		&\imps \dv{v}{t} + \frac{vk}{m} = g \\
		&\imps \dv{v}{t} + \lt(\frac{k}{m_0+kt}\rt)v = g \\
	\end{align*}
	This gives us a linear ODE, for which:\\
	Integrating Factor, $\gamma = \e{\int \dfrac{k}{m_0+kt}\dd{t}}=\e{\ln(m_0+kt)}=m_0+kt$\\
	Hence,
	\begin{align*}
		v\gamma&=\int\gamma g \dd{t}\\
		v(m_0+kt)&=\int(m_0+kt) g \dd{t}\\
		v(m_0+kt)&=gm_0t+\frac{kgt^2}{2}+c &&\ki{v=0 \text{ at } t=0 \Rightarrow c=0}\\
		\imps v&=\frac{2gm_0t+kgt^2}{2(m_0+kt)}\h
	\end{align*}
	\sss{b.}
	Given the velocity function, we find the displacement \( x(t) \) by integrating \( v \) with respect to \( t \):
	\begin{align*}
		x(t) &= \int v \dd{t} \\
		&= \int \frac{2g m_0 t + k g t^2}{2(m_0 + k t)} \dd{t}\\
		&= \int \frac{g}{2} \lt( \frac{t^2 + \frac{2 m_0}{k} t}{\frac{m_0}{k} + t} \rt) \dd{t}\\
		&= \frac{g}{2} \int \lt( \frac{\lt(t + z\rt)^2 - \lt(z\rt)^2}{z + t} \rt) \dd{t} && \ki{ z = \frac{m_0}{k}}\\
		&= \frac{g}{2} \int \lt( \qty(t + z) - \frac{z^2}{t + z} \rt) \dd{t} \\
		&= \frac{g}{2} \lt( \int (t + z) \dd{t} - \int \frac{z^2}{t + z} \dd{t} \rt) \\
		&= \frac{g}{2} \lt( \frac{t^2}{2} + z t - z^2 \ln |t + z| \rt) + c \\
		\imps x(t) &= \frac{g}{2} \lt( \frac{t^2}{2} + \frac{m_0}{k} t - \lt(\frac{m_0}{k}\rt)^2 \ln \abs{t + \frac{m_0}{k}} \rt) + C\h
	\end{align*}
	}
	\wa
	
	\pbm{}{
	As shown in the figure, two massless rods of equal length $l$ meet at a spherical
	bead of mass $m$. The other ends of the rods are attached to spheres of mass $m$ and
	$2m$, respectively, both of which rest on a frictionless horizontal surface. Initially, the
	rods are held upright, with the top bead at rest. Once released, the lower two spheres
	slide outward and the rods fall in the same vertical plane (friction neglected). Find:\\
	\begin{enumerate}
		\item[a.] The speed of the top bead (mass $m$) the moment it reaches the table.\\
		\item[b.] The speed of the sphere of mass $2m$ at the instant the angle between the two
		rods is $90^{\circ}$.\\
	\end{enumerate}
	\udia{0.4}{rdf55}
	}
	\sol{}{
	\udia{0.7}{rdf58}
	Along the rod joining the bead of mass $m$ and sphere of mass $m$, the components of velocity along the rod for each body must be equal for the distance between them to remain the same (i.e. for the rod to not break),
	\begin{equation}
		u_1 \sin\beta + u_2 \cos\beta = v_1 \cos\beta \imps u_1 \tan\beta = v_1 - u_2 \label{03} \tag{3}
	\end{equation}
	Similarily, 
	\begin{equation}
		u_1 \sin\beta - u_2 \cos\beta = v_2 \cos\beta \imps u_1 \tan\beta = v_2 + u_2 \label{04} \tag{4}
	\end{equation}
	By Energy conservation,
	\begin{equation}
		mgl(1-\sin\beta) = \frac{1}{2}m(u_1^2 + u_2^2) + \frac{1}{2}mv_1^2 + mv_2^2 \imps 2gl(1-\sin\beta) = (u_1^2 + u_2^2) + v_1^2 + 2v_2^2 \label{05} \tag{5}
	\end{equation}
	By momentum conservation along the x-axis,
	\begin{equation}
		m u_2 + m v_1 = 2m v_2 \imps u_2 + v_1 = 2 v_2 \label{06} \tag{6}
	\end{equation}
	\sss{a.}
	At this moment $\beta=0$,
	\begin{align*}
		v_1&=u_2=-v_2=0 \quad \quad \bqty{\because \eqref{03}, \eqref{04}, \eqref{06}}
	\end{align*}
	Using \eqref{05},
	\begin{align*}
		2gl(1-0) &= (u_1^2 + 0) + 0 + 0 \imps u_1^2=2gl
	\end{align*}
	Hence, \\
	The speed of the top bead when it reaches the ground is , $$u=\sqrt{u_1^2+u_2^2} \imps u=\sqrt{2gl}\h$$
	\sss{b.}
	At the instant the angle between the two rods is $90^{\circ}$, $\beta=\dfrac{\pi}{4}$,
	\begin{align*}
		u_1&=v_1-u_2=v_2+u_2 \quad \quad \bqty{\because \eqref{03}, \eqref{04}}
	\end{align*}
	By \eqref{06},
	\begin{align*}
		2v_2=u_2 + v_2+2u_2 \imps v_2= 3u_2
	\end{align*}
	Using \eqref{05},
	\begin{align*}
		2gl\lt(1-\frac{1}{\sqrt{2}}\rt) &= ((v_2+u_2)^2 + u_2^2) + (v_2+2u_2)^2 + 2v_2^2 \\
		2gl\lt(1-\frac{1}{\sqrt{2}}\rt) &= \qty(\qty(\frac{4v_2}{3})^2 + \qty(\frac{v_2}{3})^2) + \qty(\frac{5v_2}{3})^2 + 2v_2^2 \\
		18gl\lt(1-\frac{1}{\sqrt{2}}\rt) &= (16+1+25+18)v_2^2 \\
		\imps v_2&=\sqrt{\frac{18gl}{60}\lt(1-\frac{1}{\sqrt{2}}\rt)}
	\end{align*}
	Hence, \\
	The speed of the sphere of mass $2m$ at the instant the angle between the two rods is $90^{\circ}$ is , 
	$$v_2=\sqrt{\frac{3gl}{10}\lt(1-\frac{1}{\sqrt{2}}\rt)}\h$$
	}
	\wa
\end{document}