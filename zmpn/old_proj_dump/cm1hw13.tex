\documentclass{report}

\input{latex-templates/preamble}
\newcommand{\eps}{\epsilon}
\newcommand{\veps}{\varepsilon}
\newcommand{\Qed}{\begin{flushright}\qed\end{flushright}}

\newcommand{\parinn}{\setlength{\parindent}{1cm}}
\newcommand{\parinf}{\setlength{\parindent}{0cm}}

% \newcommand{\norm}{\|\cdot\|}
\newcommand{\inorm}{\norm_{\infty}}
\newcommand{\opensets}{\{V_{\alpha}\}_{\alpha\in I}}
\newcommand{\oset}{V_{\alpha}}
\newcommand{\opset}[1]{V_{\alpha_{#1}}}
\newcommand{\lub}{\text{lub}}
\newcommand{\del}[2]{\frac{\partial #1}{\partial #2}}
\newcommand{\Del}[3]{\frac{\partial^{#1} #2}{\partial^{#1} #3}}
\newcommand{\deld}[2]{\dfrac{\partial #1}{\partial #2}}
\newcommand{\Deld}[3]{\dfrac{\partial^{#1} #2}{\partial^{#1} #3}}
\newcommand{\der}[2]{\frac{\mathrm{d} #1}{\mathrm{d} #2}}
% \newcommand{\ddd}[3]{\frac{\mathrm{d}^{#3} #1}{\mathrm{d}^{#3} #2}}
\newcommand{\lm}{\lambda}
\newcommand{\uin}{\mathbin{\rotatebox[origin=c]{90}{$\in$}}}
\newcommand{\usubset}{\mathbin{\rotatebox[origin=c]{90}{$\subset$}}}
\newcommand{\lt}{\left}
\newcommand{\rt}{\right}
\newcommand{\bs}[1]{\boldsymbol{#1}}
\newcommand{\exs}{\exists}
\newcommand{\st}{\strut}
\newcommand{\dps}[1]{\displaystyle{#1}}
\newcommand{\id}{\text{id}}
\newcommand{\imps}{\quad \Rightarrow \quad}
\newcommand{\cimps}{\quad \Leftrightarrow \quad}
\newcommand{\kyuki}[1]{\quad \quad \bqty{\because \eqref{#1}}}
\newcommand{\kyukifir}[2]{\quad \quad \bqty{\because \eqref{#1} \& \eqref{#2}}}
\newcommand{\boxdia}[2]{\begin{wrapfigure}{r}{#1\textwidth}
		\fbox{\includegraphics[width=\linewidth]{Figures/#2.png}}
	\end{wrapfigure}}
\newcommand{\dia}[2]{\begin{wrapfigure}{r}{#1\textwidth}
		\includegraphics[width=\linewidth]{Figures/#2.png}
	\end{wrapfigure}}
\newcommand{\boxudia}[2]{\begin{figure}[H]
		\centering
		\fbox{\includegraphics[width=#1\textwidth]{Figures/#2.png}}
		\end{figure}}
\newcommand{\udia}[2]{\begin{figure}[H]
		\centering
		\includegraphics[width=#1\textwidth]{Figures/#2.png}
	\end{figure}}
\newcommand{\su}[2]{\textcolor{my#1}{#2}}
\newcommand{\shs}[1]{\\ \textbf{{\Large #1}}\\}
\newcommand{\sss}[1]{\vspace*{-1cm} \subsubsection*{#1}}
\newcommand{\unt}[1]{\text{#1}}
\newcommand{\wa}{
	\noindent\rule{\textwidth}{0.4pt} 
	\vspace{0.5cm}}
\newcommand{\wb}{\noindent\rule{\textwidth}{0.4pt}}
\newcommand{\qmi}{\int_{-\infty}^{\infty}}
\newcommand{\qmk}{|\psi(x,0)|^{2}}
\newcommand{\qml}{\exp{-\frac{(x - x_0)^2}{4\sigma_0^2} + \frac{i}{\hbar}p_0 x}}
\newcommand{\qmls}{\exp{-\frac{(x - x_0)^2}{4\sigma_0^2} - \frac{i}{\hbar}p_0 x}}
\newcommand{\e}[1]{\exp\lt(#1\rt)}
\newcommand\prm[2][^n]{\prescript{#1\mkern-2.5mu}{}P_{#2}}
\newcommand\cmb[2][^n]{\prescript{#1\mkern-0.5mu}{}C_{#2}}
\newcommand{\ki}[1]{\lt[\therefore #1\rt]}
\newcommand{\h}{\underset{\rotatebox{135}{\#}}{}}
\newcommand{\f}{\frac{1}{2}}


%\newcommand{\sol}[1]{\vspace{0.5cm} 
%\setlength{\parindent}{0cm} \textcolor{mytheoremfr}{\textbf{\underline{Solution:}}} \textcolor{mytheoremfr}{#1}}
\newcommand{\solve}[1]{\setlength{\parindent}{0cm}\textbf{\textit{Solution: }}\setlength{\parindent}{1cm}#1 \Qed}

\input{latex-templates/letterfonts}
\setlength{\parindent}{0pt}
\usepackage{physics, siunitx}
\usepackage{float}
\usepackage{hyperref}
\usepackage{wrapfig}
\usepackage{pgfplots}
\setlength{\fboxsep}{4pt} % Space between image and border
\setlength{\fboxrule}{0.5pt} % Border thickness

\title{\Huge{\textbf{PC2032}}\\ \su{g}{Classical Mechanics I} \\ {\huge \su{r}{Homework 13}}}
\author{\huge{Parth Bhargava}\\ A0310667E}
\date{\today}

\begin{document}

\maketitle

\pbm{}{
The equation of a transverse wave travelling in a string is given by
$$y = 0.20 \sin (0.1\pi x - 400\pi t)$$
where $x$ and $y$ are in centimetres and $t$ in seconds.
\begin{enumerate}
  \item[(a)] Find the amplitude, frequency and wavelength of the wave.
  \item[(b)] Find the maximum transverse speed of a particle in the wave.
  \item[(c)] If the tension on the string is 20 N, find the average power propagating along the string.
\end{enumerate}
}
\wb
\sol{}{
\wa
\sss{(a)}\\
For a wave equation in the standard form 
$y = 0.20 \sin (0.1\pi x - 400\pi t)$
,we can directly identify:
\begin{enumerate}
  \item Amplitude: $A=0.20 \text{ cm}$\\
  \item Wave number: $k=0.1 \pi \text{ cm}^{-1}$\\
  \item Angular frequency: $ω=400 \pi \text{ rad/s}$\\
\end{enumerate}
From the given parameters, we can calculate:
\begin{align*}
\text{Frequency} &= \frac{\omega}{2\pi} = \frac{400\pi}{2\pi} = 200  \text{Hz} \\
\text{Wavelength} &= \frac{2\pi}{k} = \frac{2\pi}{0.1\pi} = 20  \text{cm}
\end{align*}
\wa
\sss{(b)}\\
To find the maximum transverse speed, we need to compute the derivative of displacement with respect to time:

\begin{align*}
v_y &= \frac{\partial y}{\partial t} \\
&= \frac{\partial}{\partial t}[0.20 \sin (0.1 \pi x-400 \pi t)] \\
&= 0.20 \cdot (-400\pi) \cdot \cos (0.1 \pi x-400 \pi t) \\
&= -80\pi \cos (0.1 \pi x-400 \pi t)
\end{align*}

The maximum speed occurs when $\cos (0.1 \pi x -400 \pi t)= \pm 1$ giving:

\begin{align*}
|v_y|_{\text{max}} &= 80\pi , \text{cm/s} \\
&= 80 \cdot 3.14159 , \text{cm/s} \\
&\approx 251.3 , \text{cm/s}
\end{align*}
\wa
\sss{(c)}\\
Given that the tension in the string is $T=20$ N, we can find the average power using:

\begin{align*}
P_{\text{avg}} &= \frac{1}{2}\mu\omega^2A^2v
\end{align*}

First, we need to calculate the wave speed:
\begin{align*}
v &= \frac{\omega}{k} = \frac{400\pi}{0.1\pi} = 4000 , \text{cm/s} = 40 , \text{m/s}
\end{align*}

For a string under tension $T$, the wave speed is related to the linear mass density $\mu$ by:
\begin{align*}
v &= \sqrt{\frac{T}{\mu}} \\
\Rightarrow \mu &= \frac{T}{v^2} = \frac{20 , \text{N}}{(40 , \text{m/s})^2} = \frac{20}{1600} = 0.0125 , \text{kg/m}
\end{align*}

Now, calculating the average power:
\begin{align*}
P_{\text{avg}} &= \frac{1}{2}\mu\omega^2A^2v \\
&= \frac{1}{2} \cdot 0.0125 , \text{kg/m} \cdot (400\pi , \text{rad/s})^2 \cdot (0.0020 , \text{m})^2 \cdot 40 , \text{m/s} \\
&= \frac{1}{2} \cdot 0.0125 \cdot 160000\pi^2 \cdot 4 \times 10^{-6} \cdot 40 \\
&= 0.5 \cdot 0.0125 \cdot 160000 \cdot 4 \times 10^{-6} \cdot 40 \cdot \pi^2 \\
&= 0.5 \cdot 0.0125 \cdot 160000 \cdot 4 \times 10^{-6} \cdot 40 \cdot 9.87 \\
&= 0.5 \cdot 0.0125 \cdot 160000 \cdot 160 \times 10^{-6} \cdot 9.87 \\
&= 0.5 \cdot 0.0125 \cdot 25.6 \cdot 9.87 \\
&= 0.5 \cdot 0.0125 \cdot 253 \\
&\approx 1.58 , \text{W}
\end{align*}
}
\wb
\pagebreak

\pbm{}{
Two loudspeakers , $A$ and $B$, radiate sound uniformly in all directions in air at $20^{\circ}$C. The acoustic power output from $A$ is $8.00 \times 10^{-4}$ W, and from $B$ is $6.00 \times 10^{-5}$ W. Both loudspeakers are vibrating in phase at a frequency of 172 Hz.
\udia{0.5}{rdf108}
\begin{enumerate}
  \item[(a)] Determine the difference in phase of the two signals at a point $C$ along the line joining $A$ and $B$, 4.00 m from $A$ and 3.00 m from $B$.  
  \item[(b)] Determine the intensity and sound intensity level at $C$.
  \begin{enumerate}
  \item[i.] from speaker $A$ if speaker $B$ is turned off, and
  \item[ii.] from speaker $B$ if speaker $A$ is turned off.
  \end{enumerate}
  \item[(c)] With both speakers on, what are the intensity and sound intensity level at $C$?
\end{enumerate}
}
\wb
\sol{}{
\wa
\sss{(a)}\\
The speed of sound in air at $20^{\circ}$C is approximately 343 m/s. We can calculate the wavelength:

\begin{align*}
\lambda &= \frac{v}{f} = \frac{343 , \text{m/s}}{172 , \text{Hz}} = 1.994 , \text{m}
\end{align*}

The path difference between the two sources to point $C$ is:
\begin{align*}
\Delta d &= d_B - d_A = 3.00 , \text{m} - 4.00 , \text{m} = -1.00 , \text{m}
\end{align*}

The phase difference is given by:
\begin{align*}
\Delta \phi &= \frac{2\pi}{\lambda} \times \Delta d \\
&= \frac{2\pi}{1.994 , \text{m}} \times (-1.00 , \text{m}) \\
&= -2\pi \times 0.5015 \\
&= -\pi \times 1.003 \\
&\approx -\pi , \text{rad} = -180°
\end{align*}
\wa
\sss{(b)}\\
\subsection*{i. From Speaker $A$ Only}

The intensity from speaker $A$ at point $C$:
\begin{align*}
I_A &= \frac{P_A}{4\pi d_A^2} \\
&= \frac{8.00 \times 10^{-4} , \text{W}}{4\pi \times (4.00 , \text{m})^2} \\
&= \frac{8.00 \times 10^{-4}}{4\pi \times 16.0} \\
&= \frac{8.00 \times 10^{-4}}{201.06} \\
&\approx 3.98 \times 10^{-6} , \text{W/m}^2
\end{align*}

The sound intensity level:
\begin{align*}
\beta_A &= 10 \log\left(\frac{I_A}{I_0}\right) \\
&= 10 \log\left(\frac{3.98 \times 10^{-6}}{10^{-12}}\right) \\
&= 10 \log(3.98 \times 10^6) \\
&= 10 \times 6.6 \\
&\approx 66 , \text{dB}
\end{align*}

\subsection*{ii. From Speaker $B$ Only}

The intensity from speaker $B$ at point $C$:
\begin{align*}
I_B &= \frac{P_B}{4\pi d_B^2} \\
&= \frac{6.00 \times 10^{-5} , \text{W}}{4\pi \times (3.00 , \text{m})^2} \\
&= \frac{6.00 \times 10^{-5}}{4\pi \times 9.0} \\
&= \frac{6.00 \times 10^{-5}}{113.1} \\
&\approx 5.31 \times 10^{-7} , \text{W/m}^2
\end{align*}

The sound intensity level:
\begin{align*}
\beta_B &= 10 \log\left(\frac{I_B}{I_0}\right) \\
&= 10 \log\left(\frac{5.31 \times 10^{-7}}{10^{-12}}\right) \\
&= 10 \log(5.31 \times 10^5) \\
&= 10 \times 5.73 \\
&\approx 57.3 , \text{dB}
\end{align*}

\wa
\sss{(c)}\\
For two coherent sources with intensities $I_1$ and $I_2$ and phase difference $\Delta \phi$, the combined intensity is:
\begin{align*}
I_{\text{total}} = I_1 + I_2 + 2\sqrt{I_1 \times I_2} \times \cos(\Delta\phi)
\end{align*}

With $\Delta \phi \approx \pi$ (out of phase) $\to \cos (\Delta \phi)=-1$:
\begin{align*}
I_{\text{total}} &= I_A + I_B - 2\sqrt{I_A \times I_B} \\
&= 3.98 \times 10^{-6} + 5.31 \times 10^{-7} - 2\sqrt{3.98 \times 10^{-6} \times 5.31 \times 10^{-7}} \\
&= 3.98 \times 10^{-6} + 5.31 \times 10^{-7} - 2 \times 1.455 \times 10^{-6} \\
&= 4.51 \times 10^{-6} - 2.91 \times 10^{-6} \\
&= 1.60 \times 10^{-6} , \text{W/m}^2
\end{align*}

The combined sound intensity level:
\begin{align*}
\beta_{\text{total}} &= 10 \log\left(\frac{I_{\text{total}}}{I_0}\right) \\
&= 10 \log\left(\frac{1.60 \times 10^{-6}}{10^{-12}}\right) \\
&= 10 \log(1.60 \times 10^6) \\
&= 10 \times 6.20 \\
&\approx 62.0 , \text{dB}
\end{align*}

Note that due to destructive interference, the combined intensity level (62.0 dB) is less than the intensity level from speaker $A$ alone (66 dB)
}
\wb
\pagebreak

\pbm{}{
Horseshoe bats emit sounds from their nostrils and then listen to the frequency of the sound reflected from their prey to determine the prey’s speed. If a horseshoe bat flying at speed $v_{\text{bat}}$ emits sound of frequency $f_{\text{bat}}$, the sound it hears reflected from an insect flying toward the bat has a higher frequency $f_{\text{refl}}$. What is the speed of the insect?
}
\wb
\sol{}{
\wb
\udia{0.8}{rdf113}
In this scenario, we have a double Doppler effect:
\begin{enumerate}
  \item The bat (moving source) emits sound that the insect (moving observer) receives\\
  \item  The insect (now a moving source) reflects the sound, which the bat (moving observer) receives\\
\end{enumerate}

For a sound wave with frequency $f_{\text{bat}}$ emitted by a bat moving with speed $v_{\text{bat}}$, the frequency received by an insect moving toward the bat with speed $v_{\text{insect}}$, is:

\begin{align*}
f_{\text{insect}} &= f_{\text{bat}} \times \frac{v + v_{\text{insect}}}{v - v_{\text{bat}}}
\end{align*}

where $v$ is the speed of sound.

After reflection, the frequency heard by the bat is:
\begin{align*}
f_{\text{refl}} &= f_{\text{insect}} \times \frac{v + v_{\text{bat}}}{v - v_{\text{insect}}}
\end{align*}

Substituting the first equation into the second:
\begin{align*}
f_{\text{refl}} &= f_{\text{bat}} \times \frac{v + v_{\text{insect}}}{v - v_{\text{bat}}} \times \frac{v + v_{\text{bat}}}{v - v_{\text{insect}}}
\end{align*}

Hence,
\begin{align*}
  \frac{f_{\text{refl}}\left(v - v_{\text{bat}}\right)}{f_{\text{bat}}\left(v + v_{\text{bat}}\right)} &= \frac{v + v_{\text{insect}}}{v - v_{\text{insect}}} \\
    \frac{f_{\text{refl}}\left(v - v_{\text{bat}}\right) + f_{\text{bat}}\left(v + v_{\text{bat}}\right)}{f_{\text{refl}}\left(v - v_{\text{bat}}\right) - f_{\text{refl}}\left(v + v_{\text{bat}}\right)} &= \frac{2v}{2v_{\text{insect}}} &&
    \lt[\because \frac{N_1}{D_1} = \frac{N_2}{D_2} \Rightarrow \frac{N_1 + D_1}{N_1 - D_1} = \frac{N_2 + D_2}{N_2 - D_2}\rt] \\
    \imps \quad v_{\text{insect}} &= \frac{v \left( f_{\text{refl}}(v - v_{\text{bat}}) - f_{\text{refl}}(v + v_{\text{bat}}) \right)}{f_{\text{refl}}(v - v_{\text{bat}}) + f_{\text{refl}}(v + v_{\text{bat}})}
\end{align*}

Let $\dfrac{f_{\text{refl}}+f_{\text{bat}}}{f_{\text{refl}}-f_{\text{bat}}}=\mu$, then
\begin{align*}
  \boxed{v_{\text{insect}} = \frac{v(v - \mu v_{\text{bat}})}{\mu v - v_{\text{bat}}}\h}
\end{align*}
This formula gives the speed of the insect flying toward the bat. We can verify this makes sense by checking a special case: if $v_{\text{bat}}=0$ (stationary bat), then:

\begin{align*}
v_{\text{insect}} &= \frac{v^2}{\mu v} = \frac{v}{\mu}
\end{align*}
Which matches the standard formula for the Doppler effect from a moving reflector
}
\wb
\pagebreak

\end{document}
