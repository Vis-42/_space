\documentclass{report}

\input{latex-templates/preamble}
\newcommand{\eps}{\epsilon}
\newcommand{\veps}{\varepsilon}
\newcommand{\Qed}{\begin{flushright}\qed\end{flushright}}

\newcommand{\parinn}{\setlength{\parindent}{1cm}}
\newcommand{\parinf}{\setlength{\parindent}{0cm}}

% \newcommand{\norm}{\|\cdot\|}
\newcommand{\inorm}{\norm_{\infty}}
\newcommand{\opensets}{\{V_{\alpha}\}_{\alpha\in I}}
\newcommand{\oset}{V_{\alpha}}
\newcommand{\opset}[1]{V_{\alpha_{#1}}}
\newcommand{\lub}{\text{lub}}
\newcommand{\del}[2]{\frac{\partial #1}{\partial #2}}
\newcommand{\Del}[3]{\frac{\partial^{#1} #2}{\partial^{#1} #3}}
\newcommand{\deld}[2]{\dfrac{\partial #1}{\partial #2}}
\newcommand{\Deld}[3]{\dfrac{\partial^{#1} #2}{\partial^{#1} #3}}
\newcommand{\der}[2]{\frac{\mathrm{d} #1}{\mathrm{d} #2}}
% \newcommand{\ddd}[3]{\frac{\mathrm{d}^{#3} #1}{\mathrm{d}^{#3} #2}}
\newcommand{\lm}{\lambda}
\newcommand{\uin}{\mathbin{\rotatebox[origin=c]{90}{$\in$}}}
\newcommand{\usubset}{\mathbin{\rotatebox[origin=c]{90}{$\subset$}}}
\newcommand{\lt}{\left}
\newcommand{\rt}{\right}
\newcommand{\bs}[1]{\boldsymbol{#1}}
\newcommand{\exs}{\exists}
\newcommand{\st}{\strut}
\newcommand{\dps}[1]{\displaystyle{#1}}
\newcommand{\id}{\text{id}}
\newcommand{\imps}{\quad \Rightarrow \quad}
\newcommand{\cimps}{\quad \Leftrightarrow \quad}
\newcommand{\kyuki}[1]{\quad \quad \bqty{\because \eqref{#1}}}
\newcommand{\kyukifir}[2]{\quad \quad \bqty{\because \eqref{#1} \& \eqref{#2}}}
\newcommand{\boxdia}[2]{\begin{wrapfigure}{r}{#1\textwidth}
		\fbox{\includegraphics[width=\linewidth]{Figures/#2.png}}
	\end{wrapfigure}}
\newcommand{\dia}[2]{\begin{wrapfigure}{r}{#1\textwidth}
		\includegraphics[width=\linewidth]{Figures/#2.png}
	\end{wrapfigure}}
\newcommand{\boxudia}[2]{\begin{figure}[H]
		\centering
		\fbox{\includegraphics[width=#1\textwidth]{Figures/#2.png}}
		\end{figure}}
\newcommand{\udia}[2]{\begin{figure}[H]
		\centering
		\includegraphics[width=#1\textwidth]{Figures/#2.png}
	\end{figure}}
\newcommand{\su}[2]{\textcolor{my#1}{#2}}
\newcommand{\shs}[1]{\\ \textbf{{\Large #1}}\\}
\newcommand{\sss}[1]{\vspace*{-1cm} \subsubsection*{#1}}
\newcommand{\unt}[1]{\text{#1}}
\newcommand{\wa}{
	\noindent\rule{\textwidth}{0.4pt} 
	\vspace{0.5cm}}
\newcommand{\wb}{\noindent\rule{\textwidth}{0.4pt}}
\newcommand{\qmi}{\int_{-\infty}^{\infty}}
\newcommand{\qmk}{|\psi(x,0)|^{2}}
\newcommand{\qml}{\exp{-\frac{(x - x_0)^2}{4\sigma_0^2} + \frac{i}{\hbar}p_0 x}}
\newcommand{\qmls}{\exp{-\frac{(x - x_0)^2}{4\sigma_0^2} - \frac{i}{\hbar}p_0 x}}
\newcommand{\e}[1]{\exp\lt(#1\rt)}
\newcommand\prm[2][^n]{\prescript{#1\mkern-2.5mu}{}P_{#2}}
\newcommand\cmb[2][^n]{\prescript{#1\mkern-0.5mu}{}C_{#2}}
\newcommand{\ki}[1]{\lt[\therefore #1\rt]}
\newcommand{\h}{\underset{\rotatebox{135}{\#}}{}}
\newcommand{\f}{\frac{1}{2}}


%\newcommand{\sol}[1]{\vspace{0.5cm} 
%\setlength{\parindent}{0cm} \textcolor{mytheoremfr}{\textbf{\underline{Solution:}}} \textcolor{mytheoremfr}{#1}}
\newcommand{\solve}[1]{\setlength{\parindent}{0cm}\textbf{\textit{Solution: }}\setlength{\parindent}{1cm}#1 \Qed}

\input{latex-templates/letterfonts}
\usepackage{multicol}
\usepackage{physics}
\usepackage{float}
\usepackage{hyperref}
\usepackage{wrapfig}
\usepackage{pgfplots}

\setlength{\fboxsep}{1pt} % Space between image and border
\setlength{\fboxrule}{0.5pt} % Border thickness

\setlength{\columnsep}{20pt} % Adjust space between columns
\setlength{\columnseprule}{1pt}% Thickness of vertical line

\title{\Huge{PC2032 Classical Mechanics 1}\\Homework Assignment 11}
\author{\huge{Parth Bhargava}\\ AO310667E}
\date{}

\begin{document}
	\maketitle
	

\pbm{}{
Imagine a hole is drilled through the centre of the Earth to the other side.
\begin{enumerate}
	\item[a.] Write Newton's law of gravitation for an object at the distance $r$ from the centre of the Earth, and show that the force is of Hooke's law form $F = -kr$, where the effective force constant is $k = \frac{4}{3} \pi \rho Gm$. Here $\rho$ is the density of the Earth assumed uniform and $G$ is the gravitational constant.
	\item[b.] Show that a sack of mail dropped into the hole will execute simple harmonic motion if it moves without friction. When will it arrive at the other side of the Earth?
\end{enumerate}
}
\sol{}{
\sss{a.}
\udia{0.5}{rdf84}
Assuming a uniform density $\rho$ for the Earth, the mass enclosed within a sphere of radius $r$ is given by $M(r) = \rho \times \frac{4}{3} \pi r^3$. According to Newton's law of gravitation, the force on an object of mass $m$ at a distance $r$ from the center of the Earth is:
\[
F = -G \frac{m M(r)}{r^2} = -G \frac{m (\rho \frac{4}{3} \pi r^3)}{r^2} = - \left( \frac{4}{3} \pi \rho Gm \right) r\h
\]
This is of the form $F = -kr$, where the effective force constant is $k = \frac{4}{3} \pi \rho Gm$.\\
\wa
\sss{b.}
\udia{0.5}{rdf85}
The equation of motion for the sack of mail is given by $m \ddot{r} = F = -kr$, where $k = \frac{4}{3} \pi \rho Gm$. This can be written as:
\[
\ddot{r} + \frac{k}{m} r = 0 \imps \ddot{r} + \omega^2 r = 0
\]
where $\omega^2 = \dfrac{k}{m} = \dfrac{4}{3} \pi \rho G$. This is the equation for simple harmonic motion with angular frequency $\omega = \sqrt{\dfrac{4}{3} \pi \rho G}$.

The period of this motion is $T = \dfrac{2\pi}{\omega} = 2 \sqrt{\dfrac{3\pi}{4\rho G}}$. The time it takes to reach the other side of the Earth is half the period, so:
\[
t = \frac{T}{2} = \sqrt{\frac{3\pi}{4 \rho G}}\h
\]
}
\wa
\pagebreak
\pbm{}{
A spring with $k = 2.00 \, \text{N/m}$ and an attached bob oscillates in a viscous medium. The first maximum, of $+5.00 \, \text{cm}$ from the equilibrium position, is observed at $t = 2.00 \, \text{s}$ and the next maximum, of $+4.90 \, \text{cm}$, occurs at $t = 3.00 \, \text{s}$. What will the position of the bob be at $3.50 \, \text{s}$ and at $4.20 \, \text{s}$? What was its position at $t = 0.00 \, \text{s}$?
}
\sol{
The amplitude of a damped oscillation decreases exponentially with time: $A(t) = A_0 e^{-\gamma t}$, where $\gamma$ is the damping constant. We have two maxima:
\[
A(2.00) = A_0 e^{-2\gamma} = 5.00 \, \text{cm}
\]
\[
A(3.00) = A_0 e^{-3\gamma} = 4.90 \, \text{cm}
\]
Dividing the first equation by the second:
\[
\frac{5.00}{4.90} = \frac{A_0 e^{-2\gamma}}{A_0 e^{-3\gamma}} = e^{\gamma} \imps \gamma = \ln\left(\frac{5.00}{4.90}\right) \approx 0.0202 \, \text{s}^{-1}
\]
Now we can find $A_0$:
\[
A_0 = 5.00 \, \text{cm} \times e^{2\gamma} = 5.00 \times e^{2 \times 0.0202} \approx 5.206 \, \text{cm}
\]
The time difference between two consecutive maxima is the period $T = 3.00 \, \text{s} - 2.00 \, \text{s} = 1.00 \, \text{s}$.\\ The angular frequency is $$\omega = \frac{2\pi}{T} = 2\pi \, \text{rad/s}$$
The position of the bob can be written as $x(t) = A(t) \cos(\omega t + \phi) = A_0 e^{-\gamma t} \cos(\omega t + \phi)$\\
Assuming the maxima occur at the cosine peak, we can take $\omega t + \phi = 2n\pi$. \\For the first maximum at $t=2.00 \, \text{s}$, let's assume $$\omega(2.00) + \phi = 2\pi \imps 4\pi + \phi = 2\pi \imps \phi = -2\pi$$
Then $x(t) = 5.206 e^{-0.0202 t} \cos(2\pi t - 2\pi) = 5.206 e^{-0.0202 t} \cos(2\pi t)$
\udia{0.4}{rdf86}
\udia{1}{rdf90}
\sss{Hence,}
At $t = 3.50 \, \text{s}$:
\begin{align*}
  x(3.50) &= 5.206 e^{-0.0202 \times 3.50} \cos(2\pi \times 3.50) = 5.206 e^{-0.0707} \cos(7\pi) \\ &\approx 5.206 \times 0.9318 \times (-1) \\ \imps x(3.50) &\approx -4.85\, \text{cm, (3 s.f.)}\h
\end{align*}
At $t = 4.20 \, \text{s}$:
\begin{align*}
	x(4.20) &= 5.206 e^{-0.0202 \times 4.20} \cos(2\pi \times 4.20) = 5.205 e^{-0.08484} \cos(8.4\pi) \\ 
	&\approx 5.206 \times 0.9186 \times \cos(0.4\pi) \\
	&\approx 5.206 \times 0.9186 \times 0.3090 \\
	\imps x(4.20) &\approx 1.48 \, \text{cm, (3 s.f.)}\h
\end{align*}
At $t = 0.00 \, \text{s}$:
\begin{align*}
	x(0.00) &= 5.206 e^{0} \cos(0) = 5.21 \, \text{cm, (3 s.f.)}\h
\end{align*}
}
\wa
\pagebreak
\pbm{}{
In class we have obtained the equation of motion for a damped oscillator using Newton's Laws. We will now find the equation of motion using Lagrangian mechanics.\\
Consider a block of mass $m$ attached to a spring of stiffness $k$ sitting on a horizontal table.
\begin{enumerate}
  \item[a.] Write down the lagrangain $\mcL$ of the system
  \item[b.] The block experiences a resistive force, which can be expressed as a potential $U = -\f \gamma \dot{x}^2$, where $\gamma$ is the damping coefficient.his dissipative force can be
incorporated into the E-L equations as follows:
    $$\dv{t}\lt( \pdv{\mcL}{\dot{x}} \rt) - \pdv{\mcL}{x} = \pdv{U}{\dot{x}} $$
\end{enumerate}
Using the Euler-Lagrange equation, find the equations of motion.
}
\sol{
\sss{a.}
The Lagrangian of the system is given as:
\begin{align*}
\mcL &= T - V 
   = \f m \dot{x}^2 - \f k x^2 
\end{align*}
\sss{b.}
The Euler-Lagrange Equation:
\begin{align*}
\dv{t}\lt( \pdv{\mcL}{\dot{x}} \rt) - \pdv{\mcL}{x} &= \pdv{U}{\dot{x}}  
   \imps \dv{t} \lt( m \dot{x} \rt) - (-kx) = -\gamma \dot{x} 
   \imps m \ddot{x} + \gamma \dot{x} + kx = 0\h
\end{align*}
This is the equation of motion for the damped oscillator.\\
\udia{0.5}{rdf87}
}
\wa
\pagebreak
\pbm{}{
Consider two particles of masses $m_1 = m_2 = m$ confined to move along the x-axis, with positions $x_1, x_2$. These masses are connected by a spring, whose relaxed length is $l$.
\begin{enumerate}
	\item[a.] Write down the Lagrangian in terms of $x_1, x_2$.
	\item[b.] Rewrite the Lagrangian in terms of $X = \f (x_1 + x_2)$ and $x = x_2 - x_1 - l$
	\item[c.] Find the equations of motion for $X$ and $x$ using $m_1 = m_2 = 2 \, \text{kg}$ and $k = 2 \, \text{N/m}$
\end{enumerate}
}
\sol{
\udia{0.8}{rdf88}
\sss{a.}
The kinetic energy of the system is $$T = \f m_1 \dot{x}_1^2 + \f m_2 \dot{x}_2^2 = \f m (\dot{x}_1^2 + \dot{x}_2^2)$$
The potential energy of the spring is $$V = \f k (x_2 - x_1 - l)^2$$
The Lagrangian is $$L = T - V = \f m (\dot{x}_1^2 + \dot{x}_2^2) - \f k (x_2 - x_1 - l)^2$$
\wa
\sss{b.}
From the definitions, we have $x_1 = X - \f(x+l)$ and $x_2 = X + \f(x+l)$.
Then $\dot{x}_1 = \dot{X} - \f\dot{x}$ and $\dot{x}_2 = \dot{X} + \f\dot{x}$.
Substituting these into the kinetic energy:
\begin{align*}
T &= \f m \left( \left(\dot{X} - \f\dot{x}\right)^2 + \left(\dot{X} + \f\dot{x}\right)^2 \right) \\
&= \f m \left( \dot{X}^2 - \dot{X}\dot{x} + \frac{1}{4}\dot{x}^2 + \dot{X}^2 + \dot{X}\dot{x} + \frac{1}{4}\dot{x}^2 \right) \\
&= \f m \left( 2\dot{X}^2 + \f\dot{x}^2 \right) = m \dot{X}^2 + \frac{1}{4} m \dot{x}^2
\end{align*}
The potential energy is $V = \f k x^2$.
The Lagrangian in terms of $X$ and $x$ is:
\[
L = m \dot{X}^2 + \frac{1}{4} m \dot{x}^2 - \f k x^2
\]
\wa
\sss{c.}
Given $m = m_1 = m_2 = 2 \, \text{kg}$ and $k = 2 \, \text{N/m}$\\
The Lagrangian is $$L = 2 \dot{X}^2 + \f \dot{x}^2 - x^2$$
Applying the Euler-Lagrange Equation for $X$:
\begin{align*}
	\dv{t} \lt(\pdv{L}{\dot{X}}\rt) &= \pdv{L}{X} \imps \dv{t} (4 \dot{X}) = 0 \imps \ddot{X} = 0 \h
\end{align*}
Similarily for $x$:
\begin{align*}
	\dv{t} \lt(\pdv{L}{\dot{x}}\rt) &= \pdv{L}{x} \imps \dv{t} (\dot{x}) = -2x \imps \ddot{x} = -2x \h
\end{align*}
The equation of motion for $x$ is $\ddot{x} - (-2x) = 0 \imps \ddot{x} + 2x = 0$.\\
}
\wa
\pagebreak
\pbm{Bonus}{
For a ball thrown up in the air, guess a solution of the form $y(t) = a_2 t^2 + a_1 t + a_0$, given that $y(0) = y(T) = 0$.
\begin{enumerate}
	\item[a.] Using this form of $y$, write down the kinetic and potential energies of the ball.
	\item[b.] Hence, write down the Lagrangian and find the action $S$ over the given time duration.
	\item[c.] For $a_2 = -g/2$, show that the value of the action is a minima.
\end{enumerate}
}
\sol{
\sss{a.}
Given 
$$y(t) = a_2 t^2 + a_1 t + a_0$$
From $y(0) = 0$, we get 
$$a_0 = 0$$
From $y(T) = 0$, we get 
$$a_2 T^2 + a_1 T = 0 \imps a_1 = -a_2 T$$
So, 
$$y(t) = a_2 t^2 - a_2 T t = a_2 t (t - T) \imps \dot{y}(t) = 2 a_2 t - a_2 T = a_2 (2t - T)$$\\
The kinetic and potential energies of the ball are 
$$K = \f m \dot{y}^2 = \f m a_2^2 (2t - T)^2 \quad \quad V = m g y = m g a_2 t (t - T)$$
\wa
\sss{b.}
The Lagrangian is $$L = T - V = \f m a_2^2 (2t - T)^2 - m g a_2 t (t - T)$$
The action $S$ is thus given by,
\begin{align*}
	S &= \int_0^T L dt = \int_0^T \left[ \f m a_2^2 (2t - T)^2 - m g a_2 t (t - T) \right] dt \\
	&= \f m a_2^2 \int_0^T (4t^2 - 4tT + T^2) dt - m g a_2 \int_0^T (t^2 - Tt) dt \\
	&= \f m a_2^2 \left[ \frac{4}{3} t^3 - 2t^2 T + T^2 t \right]_0^T - m g a_2 \left[ \frac{1}{3} t^3 - \f T t^2 \right]_0^T \\
	&= \f m a_2^2 \left( \frac{4}{3} T^3 - 2T^3 + T^3 \right) - m g a_2 \left( \frac{1}{3} T^3 - \f T^3 \right) \\
	&= \f m a_2^2 \left( \frac{4 - 6 + 3}{3} T^3 \right) - m g a_2 \left( \frac{2 - 3}{6} T^3 \right) \\
	&= \frac{1}{6} m a_2^2 T^3 + \frac{1}{6} m g a_2 T^3\\
	S &= \frac{mT^3}{6}\lt(a_2^2 + ga_2\rt)\h
\end{align*}
\wa
\sss{c.}
To extremize action,
\begin{align*}
	\dv{S}{ta_2} &= 0 \imps \dv{t} \lt(\frac{mT^3}{6}\lt(a_2^2 + ga_2\rt)\rt) = 0 \imps \frac{mT^3}{6}\lt(2a_2 + g\rt) =0 \imps 2a_2 + g = 0 \imps a_2=-\frac{g}{2}\h
\end{align*}
Thus, for $a_2=-\dfrac{g}{2}$, the action is indeed a minimum.\\
}
	\wa
\end{document}
