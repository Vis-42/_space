\documentclass{report}

\input{latex-templates/preamble}
\newcommand{\eps}{\epsilon}
\newcommand{\veps}{\varepsilon}
\newcommand{\Qed}{\begin{flushright}\qed\end{flushright}}

\newcommand{\parinn}{\setlength{\parindent}{1cm}}
\newcommand{\parinf}{\setlength{\parindent}{0cm}}

% \newcommand{\norm}{\|\cdot\|}
\newcommand{\inorm}{\norm_{\infty}}
\newcommand{\opensets}{\{V_{\alpha}\}_{\alpha\in I}}
\newcommand{\oset}{V_{\alpha}}
\newcommand{\opset}[1]{V_{\alpha_{#1}}}
\newcommand{\lub}{\text{lub}}
\newcommand{\del}[2]{\frac{\partial #1}{\partial #2}}
\newcommand{\Del}[3]{\frac{\partial^{#1} #2}{\partial^{#1} #3}}
\newcommand{\deld}[2]{\dfrac{\partial #1}{\partial #2}}
\newcommand{\Deld}[3]{\dfrac{\partial^{#1} #2}{\partial^{#1} #3}}
\newcommand{\der}[2]{\frac{\mathrm{d} #1}{\mathrm{d} #2}}
% \newcommand{\ddd}[3]{\frac{\mathrm{d}^{#3} #1}{\mathrm{d}^{#3} #2}}
\newcommand{\lm}{\lambda}
\newcommand{\uin}{\mathbin{\rotatebox[origin=c]{90}{$\in$}}}
\newcommand{\usubset}{\mathbin{\rotatebox[origin=c]{90}{$\subset$}}}
\newcommand{\lt}{\left}
\newcommand{\rt}{\right}
\newcommand{\bs}[1]{\boldsymbol{#1}}
\newcommand{\exs}{\exists}
\newcommand{\st}{\strut}
\newcommand{\dps}[1]{\displaystyle{#1}}
\newcommand{\id}{\text{id}}
\newcommand{\imps}{\quad \Rightarrow \quad}
\newcommand{\cimps}{\quad \Leftrightarrow \quad}
\newcommand{\kyuki}[1]{\quad \quad \bqty{\because \eqref{#1}}}
\newcommand{\kyukifir}[2]{\quad \quad \bqty{\because \eqref{#1} \& \eqref{#2}}}
\newcommand{\boxdia}[2]{\begin{wrapfigure}{r}{#1\textwidth}
		\fbox{\includegraphics[width=\linewidth]{Figures/#2.png}}
	\end{wrapfigure}}
\newcommand{\dia}[2]{\begin{wrapfigure}{r}{#1\textwidth}
		\includegraphics[width=\linewidth]{Figures/#2.png}
	\end{wrapfigure}}
\newcommand{\boxudia}[2]{\begin{figure}[H]
		\centering
		\fbox{\includegraphics[width=#1\textwidth]{Figures/#2.png}}
		\end{figure}}
\newcommand{\udia}[2]{\begin{figure}[H]
		\centering
		\includegraphics[width=#1\textwidth]{Figures/#2.png}
	\end{figure}}
\newcommand{\su}[2]{\textcolor{my#1}{#2}}
\newcommand{\shs}[1]{\\ \textbf{{\Large #1}}\\}
\newcommand{\sss}[1]{\vspace*{-1cm} \subsubsection*{#1}}
\newcommand{\unt}[1]{\text{#1}}
\newcommand{\wa}{
	\noindent\rule{\textwidth}{0.4pt} 
	\vspace{0.5cm}}
\newcommand{\wb}{\noindent\rule{\textwidth}{0.4pt}}
\newcommand{\qmi}{\int_{-\infty}^{\infty}}
\newcommand{\qmk}{|\psi(x,0)|^{2}}
\newcommand{\qml}{\exp{-\frac{(x - x_0)^2}{4\sigma_0^2} + \frac{i}{\hbar}p_0 x}}
\newcommand{\qmls}{\exp{-\frac{(x - x_0)^2}{4\sigma_0^2} - \frac{i}{\hbar}p_0 x}}
\newcommand{\e}[1]{\exp\lt(#1\rt)}
\newcommand\prm[2][^n]{\prescript{#1\mkern-2.5mu}{}P_{#2}}
\newcommand\cmb[2][^n]{\prescript{#1\mkern-0.5mu}{}C_{#2}}
\newcommand{\ki}[1]{\lt[\therefore #1\rt]}
\newcommand{\h}{\underset{\rotatebox{135}{\#}}{}}
\newcommand{\f}{\frac{1}{2}}


%\newcommand{\sol}[1]{\vspace{0.5cm} 
%\setlength{\parindent}{0cm} \textcolor{mytheoremfr}{\textbf{\underline{Solution:}}} \textcolor{mytheoremfr}{#1}}
\newcommand{\solve}[1]{\setlength{\parindent}{0cm}\textbf{\textit{Solution: }}\setlength{\parindent}{1cm}#1 \Qed}

\input{latex-templates/letterfonts}
\usepackage{physics}
\usepackage{lipsum}
\usepackage{float}
\usepackage{hyperref}
\usepackage{wrapfig}
\usepackage{enumitem}
\setlength{\fboxsep}{1pt} % Space between image and border
\setlength{\fboxrule}{0.5pt} % Border thickness

\title{\Huge{PC2031 Electricity and Magnetism 1}\\Tutorials}
\author{\huge{Parth Bhargava}}
\date{\today}

\begin{document}
	\maketitle
	\tableofcontents
	
	\chapter{Vector Calculus}
	\qs{}{
	Given that,
	
	$$\vb{r}=x\vu*{x}+y\vu*{y}+z\vu*{z},\quad r=|\vb{r}|=\sqrt{x^2+y^2+z^2}$$
	
	and
	
	$$\vb{r^{\prime}}=x^{\prime}\vu*{x}+y^{\prime}\vu*{y}+z^{\prime}\vu*{z},\quad r^{\prime}=|\vb{r^{\prime}}|=\sqrt{x^{\prime2}+y^{\prime2}+z^{\prime2}}$$
	
	show that
	
	$$\grad\pqty{\frac1{|\vb{r}-\vb{r^{\prime}}|}}=-\frac{\vb{r}-\vb{r^{\prime}}}{|\vb{r}-\vb{r^{\prime}}|^3}$$
	
	
	Hence, deduce that
	
	$$\grad\times\pqty{\frac{\vb{r}-\vb{r^{\prime}}}{|\vb{r}-\vb{r^{\prime}}|^3}}=0$$
	}
	\sol{}{
	Here,
	\begin{align*}
		\frac{1}{\abs{\vb{r} - \vb{r}'}} &= \left[ (x - x')^2 + (y - y')^2 + (z - z')^2 \right]^{-1/2} \\
		\grad \pqty{\frac{1}{\abs{\vb{r} - \vb{r}'}}} &= \vu*{x} \lt( -\frac{1}{2} \rt) \left[ (x - x')^2 + (y - y')^2 + (z - z')^2 \right]^{-3/2} 2 (x - x') \\
		&\quad + \vu*{y} \lt( -\frac{1}{2} \rt) \left[ (x - x')^2 + (y - y')^2 + (z - z')^2 \right]^{-3/2} 2 (y - y') \\
		&\quad + \vu*{z} \lt( -\frac{1}{2} \rt) \left[ (x - x')^2 + (y - y')^2 + (z - z')^2 \right]^{-3/2} 2 (z - z') \\
		&= -\vu*{x} \frac{(x - x')}{\abs{\vb{r} - \vb{r}'}^3} - \vu*{y} \frac{(y - y')}{\abs{\vb{r} - \vb{r}'}^3} - \vu*{z} \frac{(z - z')}{\abs{\vb{r} - \vb{r}'}^3} = -\frac{\vb{r} - \vb{r}'}{\abs{\vb{r} - \vb{r}'}^3} \\
		\therefore \grad \lt( \frac{1}{\abs{\vb{r} - \vb{r}'}} \rt) &= -\frac{\vb{r} - \vb{r}'}{\abs{\vb{r} - \vb{r}'}^3}
	\end{align*}
	\qed\\
	Hence,
	$$\curl \lt( \frac{\vb{r} - \vb{r}'}{\abs{\vb{r} - \vb{r}'}^3} \rt)= -\curl \lt[ \grad \lt( \frac{1}{\abs{\vb{r} - \vb{r}'}} \rt) \rt] = 0$$
	\qed
	}
	
	\qs{}{
	Calculate the line integral of the function $\vb{v}=x^2\vu*{x}+2yz\vu*{y}+y^2\vu*{z}$ from the
	origin to the point $(1, 1, 1)$ by three different routes:\\
	\begin{enumerate}
		\item[a.] $(0, 0, 0) \to (1, 0, 0) \to (1, 1, 0) \to (1, 1, 1)$\\
		\item[b.] $(0, 0, 0) \to (0, 0, 1) \to (0, 1, 1) \to (1, 1, 1)$\\
		\item[c.] The direct straight line.\\
		\item[d.] What is the line integral around the closed loop that goes out along path (a) and then goes back along path (b)?\\
	\end{enumerate}
	}
	\sol{}{
	$$
	\vb{v} = x^2 \vu{x} + 2yz \vu{y} + y^2 \vu{z}, \quad \dd{\vb{\ell}} = \dd{x} \vu{x} + \dd{y} \vu{y} + \dd{z} \vu{z}\\
	$$
	\wa

	\sss{a.} $(0,0,0) \to (1,0,0)$, along $x$-axis
	
	$$
	\int \vb{v} \cdot \dd{\vb{\ell}} = \int_0^1 v_x \dd{x} = \int_0^1 x^2 \dd{x} = \frac{1}{3}
	$$
	
	$(1,0,0) \to (1,1,0)$, along $y$-axis
	
	$$
	\int \vb{v} \cdot \dd{\vb{\ell}} = \int_0^1 v_y \dd{y} = \int_0^1 2yz \dd{y} = 0, \quad \because z=0 \text{ along the line from } (1,0,0) \to (1,1,0)
	$$
	
	$(1,1,0) \to (1,1,1)$, along $z$-axis
	
	$$
	\int \vb{v} \cdot \dd{\vb{\ell}} = \int_0^1 v_z \dd{z} = \int_0^1 y^2 \dd{z} = y^2 \int_0^1 \dd{z} = 1, \quad \because y=1 \text{ along the line from } (1,1,0) \to (1,1,1)
	$$
	
	$\therefore$ total $\int \vb{v} \cdot \dd{\vb{\ell}} = \frac{1}{3} + 0 + 1 = \frac{4}{3}\\$
	\wa
	
	\sss{b.} $(0,0,0) \to (0,0,1)$, along $z$-axis
	
	$$
	\int \vb{v} \cdot \dd{\vb{\ell}} = \int_0^1 v_z \dd{z} = \int_0^1 y^2 \dd{z} = 0, \quad \because y=0 \text{ along the line from } (0,0,0) \to (0,0,1)
	$$
	
	$(0,0,1) \to (0,1,1)$, along $y$-axis
	
	$$
	\int \vb{v} \cdot \dd{\vb{\ell}} = \int_0^1 v_y \dd{y} = \int_0^1 2yz \dd{y} = 2z \int_0^1 y \dd{y} = 2 \left( \frac{1}{2} \right) = 1, \quad \because z=1 \text{ along the line from } (0,0,1) \to (0,1,1)
	$$
	
	$(0,1,1) \to (1,1,1)$, along $x$-axis
	
	$$
	\int \vb{v} \cdot \dd{\vb{\ell}} = \int_0^1 v_x \dd{x} = \int_0^1 x^2 \dd{x} = \frac{1}{3}
	$$
	
	$\therefore$ total $\int \vb{v} \cdot \dd{\vb{\ell}} = 0 + 1 + \frac{1}{3} = \frac{4}{3}\\$
	\wa
	
	\sss{c.} The direct straight line from $(0,0,0) \to (1,1,1)$
	
	$$
	\int \vb{v} \cdot \dd{\vb{\ell}} = \int \left( x^2 \vu{x} + 2yz \vu{y} + y^2 \vu{z} \right) \cdot \left( \dd{x} \vu{x} + \dd{y} \vu{y} + \dd{z} \vu{z} \right) = \int \left( x^2 \dd{x} + 2yz \dd{y} + y^2 \dd{z} \right)
	$$
	
	At any point on the line from $(0,0,0) \to (1,1,1)$, $x=y=z$. Thus, $\dd{x} = \dd{y} = \dd{z}$
	
	$$
	\therefore \int \vb{v} \cdot \dd{\vb{\ell}} = \int \left( x^2 \dd{x} + 2yz \dd{y} + y^2 \dd{z} \right) = \int_0^1 \left( 4x^2 \dd{x} \right) = \frac{4}{3}\\
	$$
	\wa
	
	\sss{d.} along path (a) and back along reversed path (b)
	
	$$
	\int \vb{v} \cdot \dd{\vb{\ell}} = \underbrace{\int \vb{v} \cdot \dd{\vb{\ell}}}_{\text{along path (a)}} + \underbrace{\int \vb{v} \cdot \dd{\vb{\ell}}}_{\text{reversed path (b)}} = \frac{4}{3} - \frac{4}{3} = 0\\
	$$
	\wa
	}
	
	\qs{}{
	\begin{enumerate}
		\item[a.] Show that 
		$$\int_S\grad T\times\dd\vb{a}=-\oint_CT\dd\vb{\ell}$$
		Hint: Let $\vb{A}=\vb{c}T$ in Stokes’ theorem\\
		\item[b.] Hence, show that
		$$\oint(\vb{c}\cdot \vb{r})\operatorname{d\vb{\ell}}=\vb{a}\times \vb{c},$$
		
		
		for any constant vector $\vb{c}.$ \\
		Hint: Let $T= \vb{c}\cdot \vb{r}$
	\end{enumerate}
	
	}
	\sol{}{
	
	}
	
	\qs{}{
	Calculate the divergence and curl of the following vector functions:
	\begin{enumerate}
		\item[a.] $v_a=x^2\vu*{\vb{x}}+3xz^2\vu*{\vb{y}}-2xz\vu*{\vb{z}}$\\
		\item[b.] $\vb{v}_b= y^2\vu*{\vb{x}} + ( 2xy+ z^2) \vu*{\vb{y}} + 2yz\vu*{\vb{z}}$\\
		\item[c.] Which of the vectors can be expressed as the gradient of a scalar? Find a scalar function that does the job.\\
		\item[d.] Which vector can be expressed as the curl of a vector? Find such a vector.\\
	\end{enumerate}
	}
	\sol{}{
	
	}
	
	\chapter{Electrostatics}
	\qs{}{
	Consider a hemispherical shell of radius $R$ and uniform surface charge density $\sigma$ in the region $z\geq0.$ The center of the hemispherical shell is the origin.
	\udia{0.3}{rdf8}
	Find the electric field at the center of the hemispherical shell. Repeat the calculations for a solid hemisphere of radius $R$ and uniform volume charge density $\rho$.
	}
	\sol{}{
		
	}
	
	\qs{}{
	If the electric field in some region is given (in spherical coordinates) by the expression
	$$\vb{E}\lt(\vb{r}\rt)=\frac kr\lt(3\vb{\vu*{r}}+2\sin\theta\cos\theta\sin\vb{\phi}\vb{\vu*{\theta}}+\sin\theta\cos\vb{\phi}\vb{\vu*{\phi}}\rt),$$
	
	for some constant $k$, what is the charge density?
	}
	\sol{}{
		
	}
	
	\qs{}{
	Consider a semi-infinite hollow cylindrical shell with radius $R$ and uniform surface charge density $\sigma$ extending from $z=-\infty$ to $z=0$ along the $z$-axis. The cylindrical axis of the cylindrical shell is the $z$-axis. Find the electric field at an arbitrary point along the $+z$-axis. Repeat the calculations for a semi-infınite solid cylinder with radius $R$ and uniform volume charge density $\rho$.
	}
	\sol{}{
		
	}
	
	\qs{}{
	A long coaxial cable carries a uniform volume charge density $\rho,\rho>0$, on the inner cylinder of radius $a$ and a uniform surface charge density on the outer cylindrical shell of radius $b,b>a.$
	\udia{0.3}{rdf9}
	This surface charge density is negative and is of just right magnitude that
	the cable as a whole is electrically neutral. Find $\vb{E}(\vb{r})$ everywhere.
	}
	\sol{}{
		
	}
	
	\chapter{Magnetostatics}
	\qs{}{
	Starting from the divergence theorem, show that
	$$\iiint_V\curl \vb{A} \ \dd v=\oiint\dd \vb{a}\times \vb{A}$$
	}
	\sol{}{
		
	}
	
	\qs{}{
	A square wire loop of size $2a\times2a$ lies in the $xy$ plane with its center at the origin and sides parallel to the $x$ and $y$ axes. The current $I$ flows counter- clockwise around the loop as viewed from the $+z$-axis.
	\udia{0.3}{rdf10}
	Find the magnetic field at an arbitrary point on the z-axis.		
	}
	\sol{}{
		
	}
	
	\qs{}{
	A circular cylindrical solenoid of finite length $L$ and radius $a$ is coaxial with the $z$-axis extending from $z=-L/2$ to $z=L/2$. The solenoid has $N$ tight wound turns and each turn carries current $I$.
	\udia{0.4}{rdf11}
	The current flows counter-clockwise as viewed from the $z$-axis. Find the magnetic field at an arbitrary point along the $z$-axis.
	}
	\sol{}{
		
	}
	
	\qs{}{
	A long coaxial cable, coaxial with the $z$-axis, consists of a central conductor of radius $a$ and an outer conductor of inner radius $b$ and outer radius $c$.
	\udia{0.2}{rdf12}
	The central conductor carries a current $I$ in the $+z$ direction and the outer conductor carries the return current $I$ in the $-z$ direction. The currents are uniformly distributed in the conductors. There is an insulating material between the conductors. Find the magnetic field everywhere.
	}
	\sol{}{
		
	}
	
	\qs{}{
		A plane wire loop of irregular shape is situated so that part of it is in a uniform magnetic field $B$. In the figure below, the field occupies the shaded region, and points perpendicular to the plane of the loop.
		\udia{0.3}{rdf13}
		The loop carries a current $I$. Show that the net magnetic force on the loop is $F=IBw$, where $w$ is the chord subtended. What is the direction of the force?
	}
	\sol{}{
		
	}
	
	\chapter{Potentials}
	\qs{}{
	An inverted hemispherical bowl of radius $R$ carries a uniform surface charge density $\sigma$. Find the electrostatic scalar potential difference between the “north pole” and the center.	
	}
	\sol{}{
		
	}
	
	\qs{}{
	A sphere of radius $R$ carries a volume charge density $\rho\pqty{\vb{r}}=\alpha r$ where $\alpha$ is a positive constant. Find the electrostatic total energy of the configuration.	
	}
	\sol{}{
		
	}
	
	\qs{}{
	A disk of radius $a$ carries a uniform surface charge density $\sigma$.
	\begin{enumerate}
		\item[a.] Find the electrostatic scalar potential at any point on the symmetry axis of the disk.
		\item[b.] Find the electrostatic scalar potential at any point on the rim of the disk.
		\nt{$$\int_{\phi=0}^{2\pi}\int_{w=0}^1\frac{w}{\sqrt{1+w^2-2w\cos\phi}}\dd w\dd \phi=4$$}
		\item[c.] Find the electrostatic total energy of the disk.
	\end{enumerate}
	}
	\sol{}{
		
	}
	
	\qs{}{
	Find the magnetic vector potential of an infinite uniform surface current $\vb{K}\pqty{\vb{r}}=K\vu*{x}$ flowing over the $xy$ plane.
	\udia{0.3}{rdf14}
	}
	\sol{}{
		
	}
	
	\qs{}{
	What current density would produce the vector potential,  $\vb{A}=k\vu*{\phi}$(where $k$ is a constant), in cylindrical coordinates?
	}
	\sol{}{
		
	}
	
	\chapter{Electrodynamics}
	\qs{}{
	A long solenoid with radius $a$ and $n$ turns per unit length carries a time-dependent current $I(t)$ in the $\vu*{\phi}$ direction. Find the electric field at a distance $s$ from the axis in the quasistatic approximation.	
	}
	\sol{}{
		
	}
	
	\qs{}{
	An alternating current $I(t)= I_0 \cos(\omega t)$ flows along a long straight wire in the $+z$ direction and returns along a coaxial conducting tube of radius $a$.
	\begin{enumerate}
		\item Assuming that the field goes to zero as $s \to \infty$, find $\vb{E}(s,t)$ in the
		quasistatic approximation.\\
		\item Find the displacement current density $J_d(\vb{r},t)$ and the total displacement current $I_d(t)$.\\
	\end{enumerate}
	}
	\sol{}{
		
	}
	
	\qs{}{
	An infinite cylinder of radius $R$ carries a uniform surface charge $\sigma$. We propose to set it spinning about its axis, at a final angular velocity $\omega_f$. How much work will this take, per unit length? Do it two ways, and compare your answers:
	\begin{enumerate}
		\item Find the magnetic field and the induced electric field (in the quasistatic approximation), inside and outside the cylinder, in terms of $\omega , \dot{\omega}$ and $s$ (the distance from the axis). Calculate the torque you must exert, and from that obtain the work done per unit length $W=\int N \dd \phi$\\
		\item Use $$\frac{1}{2\mu_0}\int B^2 \dd v$$ to determine the energy stored in the resulting magnetic field.\\
	\end{enumerate}
	}
	\sol{}{
		
	}
	
	\qs{}{
	A perfectly conducting spherical shell of radius $a$ rotates about the z axis with angular velocity $\omega$, in a uniform magnetic field $\vb{B}=B_0\vu*{z}$. Calculate the emf developed between the “north pole” and the equator.	
	}
	\sol{}{
		
	}
	
	\chapter{Multipole Expansions}
	\qs{}{
	A solid sphere, of radius $R$, is centered at the origin. The “northern” hemisphere carries a uniform charge density $\rho_0$ and the “southern” hemisphere a uniform charge density $-\rho_0$. Find the asymptotic electrostatic field $\vb{E}(r,\theta)$ where $r \gg R$.	
	}
	\sol{}{
		
	}
	
	\qs{}{
	An electric dipole $p_1=p_0\vu*{z}$ is fixed at the origin. Find the work required to bring another dipole $p_2=p_0\vu*{z}$ at $(0,0,z_0)$.
	\udia{0.4}{rdf15}
	What is the force on the dipole $p_2$?
	}
	\sol{}{
		
	}
	
	\qs{}{
	A uniform volume current density $\vb{J}=J_0\vu*{z}$ fills a slab straddling the $yz$ plane from $x=-a$ to $x=+a$.\\
	\udia{0.3}{rdf16}
	A magnetic dipole $\vb{m}=m_0\vu*{x}$ is situated at the origin. Find the force on the dipole. Repeat the calculation for a dipole pointing in the $y$-direction: $\vb{m}=m_0\vu*{y}$.
	}
	\sol{}{
		
	}
	
	 
	
\end{document}