\documentclass{report}

\input{latex-templates/preamble}
\newcommand{\eps}{\epsilon}
\newcommand{\veps}{\varepsilon}
\newcommand{\Qed}{\begin{flushright}\qed\end{flushright}}

\newcommand{\parinn}{\setlength{\parindent}{1cm}}
\newcommand{\parinf}{\setlength{\parindent}{0cm}}

% \newcommand{\norm}{\|\cdot\|}
\newcommand{\inorm}{\norm_{\infty}}
\newcommand{\opensets}{\{V_{\alpha}\}_{\alpha\in I}}
\newcommand{\oset}{V_{\alpha}}
\newcommand{\opset}[1]{V_{\alpha_{#1}}}
\newcommand{\lub}{\text{lub}}
\newcommand{\del}[2]{\frac{\partial #1}{\partial #2}}
\newcommand{\Del}[3]{\frac{\partial^{#1} #2}{\partial^{#1} #3}}
\newcommand{\deld}[2]{\dfrac{\partial #1}{\partial #2}}
\newcommand{\Deld}[3]{\dfrac{\partial^{#1} #2}{\partial^{#1} #3}}
\newcommand{\der}[2]{\frac{\mathrm{d} #1}{\mathrm{d} #2}}
% \newcommand{\ddd}[3]{\frac{\mathrm{d}^{#3} #1}{\mathrm{d}^{#3} #2}}
\newcommand{\lm}{\lambda}
\newcommand{\uin}{\mathbin{\rotatebox[origin=c]{90}{$\in$}}}
\newcommand{\usubset}{\mathbin{\rotatebox[origin=c]{90}{$\subset$}}}
\newcommand{\lt}{\left}
\newcommand{\rt}{\right}
\newcommand{\bs}[1]{\boldsymbol{#1}}
\newcommand{\exs}{\exists}
\newcommand{\st}{\strut}
\newcommand{\dps}[1]{\displaystyle{#1}}
\newcommand{\id}{\text{id}}
\newcommand{\imps}{\quad \Rightarrow \quad}
\newcommand{\cimps}{\quad \Leftrightarrow \quad}
\newcommand{\kyuki}[1]{\quad \quad \bqty{\because \eqref{#1}}}
\newcommand{\kyukifir}[2]{\quad \quad \bqty{\because \eqref{#1} \& \eqref{#2}}}
\newcommand{\boxdia}[2]{\begin{wrapfigure}{r}{#1\textwidth}
		\fbox{\includegraphics[width=\linewidth]{Figures/#2.png}}
	\end{wrapfigure}}
\newcommand{\dia}[2]{\begin{wrapfigure}{r}{#1\textwidth}
		\includegraphics[width=\linewidth]{Figures/#2.png}
	\end{wrapfigure}}
\newcommand{\boxudia}[2]{\begin{figure}[H]
		\centering
		\fbox{\includegraphics[width=#1\textwidth]{Figures/#2.png}}
		\end{figure}}
\newcommand{\udia}[2]{\begin{figure}[H]
		\centering
		\includegraphics[width=#1\textwidth]{Figures/#2.png}
	\end{figure}}
\newcommand{\su}[2]{\textcolor{my#1}{#2}}
\newcommand{\shs}[1]{\\ \textbf{{\Large #1}}\\}
\newcommand{\sss}[1]{\vspace*{-1cm} \subsubsection*{#1}}
\newcommand{\unt}[1]{\text{#1}}
\newcommand{\wa}{
	\noindent\rule{\textwidth}{0.4pt} 
	\vspace{0.5cm}}
\newcommand{\wb}{\noindent\rule{\textwidth}{0.4pt}}
\newcommand{\qmi}{\int_{-\infty}^{\infty}}
\newcommand{\qmk}{|\psi(x,0)|^{2}}
\newcommand{\qml}{\exp{-\frac{(x - x_0)^2}{4\sigma_0^2} + \frac{i}{\hbar}p_0 x}}
\newcommand{\qmls}{\exp{-\frac{(x - x_0)^2}{4\sigma_0^2} - \frac{i}{\hbar}p_0 x}}
\newcommand{\e}[1]{\exp\lt(#1\rt)}
\newcommand\prm[2][^n]{\prescript{#1\mkern-2.5mu}{}P_{#2}}
\newcommand\cmb[2][^n]{\prescript{#1\mkern-0.5mu}{}C_{#2}}
\newcommand{\ki}[1]{\lt[\therefore #1\rt]}
\newcommand{\h}{\underset{\rotatebox{135}{\#}}{}}
\newcommand{\f}{\frac{1}{2}}


%\newcommand{\sol}[1]{\vspace{0.5cm} 
%\setlength{\parindent}{0cm} \textcolor{mytheoremfr}{\textbf{\underline{Solution:}}} \textcolor{mytheoremfr}{#1}}
\newcommand{\solve}[1]{\setlength{\parindent}{0cm}\textbf{\textit{Solution: }}\setlength{\parindent}{1cm}#1 \Qed}

\input{latex-templates/letterfonts}
\setlength{\parindent}{0pt}
\usepackage{physics, siunitx}
\usepackage{float}
\usepackage{hyperref}
\usepackage{wrapfig}
\usepackage{pgfplots}
\setlength{\fboxsep}{4pt} % Space between image and border
\setlength{\fboxrule}{0.5pt} % Border thickness

\title{\Huge{\textbf{PC2032}}\\ \su{g}{Classical Mechanics I} \\ {\huge \su{r}{Homework 12}}}
\author{\huge{Parth Bhargava}\\ } %A0310667E
\date{\today}

\begin{document}

\maketitle

\pbm{}{
A pendulum consists of a mass m and a massless stick of length l (see Figure). The pendulum support oscillates horizontally with a position given by 
$$x(t) = A \cos(\omega t)$$ 
What is the general solution for the angle of the pendulum as a function of time? \\
\textit{You may leave the constants that depend on initial conditions unspecified.}
\udia{0.2}{rdf110}
}
\wb
\sol{}{
\wa
\dia{0.4}{rdf111}
Let \( x_m \) and \( y_m \) be the \( x \) and \( y \) positions of the mass \( m \), respectively.
$$
\begin{cases}
    x_m = x(t) + \ell \sin\theta &\imps \dot{x}_m = \dot{x} + \ell \dot{\theta} \cos\theta \\
    y_m = -\ell \cos\theta &\imps \dot{y}_m = \ell \dot{\theta} \sin\theta\\
\end{cases}
$$
Here,
\begin{align*}
 v_m &= \sqrt{\dot{x}_m^2 + \dot{y}_m^2} \\
    \imps  v_m^2 &= \dot{x}_m^2 + \dot{y}_m^2 \\
    &= \dot{x}^2 + 2\ell \dot{x} \dot{\theta} \cos\theta + \ell^2 \dot{\theta}^2 \cos^2\theta + \ell^2 \dot{\theta}^2 \sin^2\theta \\
    &= \dot{x}^2 + 2\ell \dot{x} \dot{\theta} \cos\theta + \ell^2 \dot{\theta}^2
\end{align*}

The Lagrangian of the system is given by,

\begin{align*}
\mathcal{L} &= T - U \\
&= \lt(\frac{1}{2} m v_m^2\rt) - \lt(-mg\ell \cos\theta\rt)\\
&= \frac{1}{2} m \dot{x}^2 + m\ell \dot{x} \dot{\theta} \cos\theta + \frac{1}{2} m \ell^2 \dot{\theta}^2 + mg\ell \cos\theta\\
\end{align*}

The Euler-Lagrange Equations in terms of $\theta$ give us,
\begin{align*}
\dv{t} \left( \pdv{\mathcal{L}}{\dot{\theta}} \right) = \pdv{\mathcal{L}}{\theta}\imps&
\dv{t} \left( m\ell \dot{x} \cos\theta + m\ell^2 \dot{\theta} \right) = -m\ell \dot{x} \dot{\theta} \sin\theta - mg\ell \sin\theta \\
\imps& m\ell \cos\theta \ddot{x} - m\ell \dot{x} \dot{\theta} \sin\theta  + m\ell^2 \ddot{\theta} = -m\ell \dot{x} \dot{\theta} \sin\theta - mg\ell \sin\theta \\
\imps& g\sin\theta + \ddot{x} \cos\theta + \ell \ddot{\theta} = 0
\end{align*}

Since \( x(t) = A \cos(\omega t) \Rightarrow \ddot{x} = -A\omega^2 \cos(\omega t) \), 
\[
g \sin\theta - A\omega^2 \cos(\omega t) \cos\theta + \ell \ddot{\theta} = 0
\]

For small angles, $\cos\theta \approx 1$ and $\sin\theta \approx \theta$, which gives,
\begin{align*}
\ell \ddot{\theta} + g\theta &= A\omega^2 \cos(\omega t) \imps \ddot{\theta} + \frac{g}{\ell} \theta = \frac{A}{\ell} \omega^2 \cos(\omega t)
\end{align*}

This is the equation of a driven oscillator.

\[
\therefore \theta(t) = \frac{A}{\ell} \cdot \frac{\omega^2}{\frac{g}{\ell} - \omega^2} \cos(\omega t) + D \cos\left( \sqrt{\frac{g}{\ell}} t + \varphi \right)\h
\]

Where \( D \) and \( \varphi \) are constants to be determined.
}
\wb
\pagebreak

\pbm{}{
Two masses $m_1$ and $m_2$ are connected by a massless rope in which a massless spring (with spring constant $k$) is inserted. The rope passes over a massless, smooth pulley. The mass $m_1$ slides on a frictionless horizontal surface, while the mass $m_2$ hangs vertically. When the spring is in its relaxed state, the total length of the spring-rope system is $l$.
Find suitable generalized coordinates to describe the motion of the two masses,allowing for elongation or compression of the spring.
Construct the Lagrangian and derive the appropriate Euler-Lagrange equations.
\udia{0.4}{rdf109}
}
\wb
\sol{}{
\wa
\dia{0.4}{rdf91}
For the entire system,
\begin{align*}
	T &= \f m_1 \dot{x}^2 + \f m_2 \dot{y}^2 \\
	V &= \f k (x + y - l)^2 - m_2 g y \\
	\mcL &= \f \lt[ m_1 \dot{x}^2 + m_2 \dot{y}^2 - k(x + y - l)^2 + 2m_2 g y \right]
\end{align*}
The Euler-Lagrange Equations in terms of $x$ give us,
\begin{align*}
	\dv{t} \lt( \pdv{\mcL}{\dot{x}} \right) &= \pdv{\mcL}{x}
	\imps \f \dv{t} \lt( 2 m_1 \dot{x} \right) 
	= \f \lt[ -2k(x + y - l) \right]
	\imps m_1 \ddot{x} = -k(x + y - l)\h
\end{align*}
Similarily, the Euler-Lagrange Equations in terms of $y$ give us,
\begin{align*}
	\dv{t} \lt( \pdv{\mcL}{\dot{y}} \right) &= \pdv{\mcL}{y}
	\imps \f \dv{t} \lt( 2 m_2 \dot{y} \right) 
	= \f \lt[ -2k(x + y - l) + 2 m_2 g \right]
	\imps m_2 \ddot{y} = -k(x + y - l) + m_2 g\h
\end{align*}
}
\wb
\pagebreak

\pbm{}{
A uniform rod of length $2L$ and mass $M$ is suspended from a fixed horizontal table by a length $l$, inextensible rope (whose mass is negligible), as shown in the lt figure. A constant horizontal force $F$ is applied at the free end of the rod, causing the system to move as depicted in the right figure. Let’s choose $\theta$ and $\varphi$ as the generalized coordinates.
\udia{0.7}{rdf92}
\begin{enumerate}
	\item[1.] Write down the coordinates of the CM of the rod using $\theta$ and $\varphi$. What is the velocity of the CM?
	\item[2.] Recall that the kinetic energy of the rod can be written as the sum of the translational kinetic energy of the center of mass $T_{\text{CM}}$ and the rotational kinetic energy $T_{\text{rot}}$  Write down the total kinetic energy.
	\item[3.] Since the force F is constant and acts along the same direction, you can regard it as a potential similar to gravity in the horizontal direction. Write down the total potential energy.
	\item[4.] Write down the Lagrangian of this system and derive its equations of motion.
\end{enumerate}
}
\wb
\sol{}{
\wb
\udia{0.9}{rdf106}
The total kinetic eneregy is given as,
\begin{align*}
T &= T_{\text{rot}} + T_{\text{CM}} = \f \lt( \frac{ML^2}{3} \rt) \dot{\varphi}^2 + \f M \vb{\dot{Q}}^2_{CM} \imps T = \lt( \frac{ML^2}{6} \rt) \dot{\varphi}^2 + \f M \lt[l^2\dot{\theta}^2 +L^2\dot{\varphi}^2 +2lL\dot{\theta}\dot{\varphi}\cos (\theta-\varphi)\rt]
\end{align*}

The total potential energy is given as,
\begin{align*}
V= - Mg \left( l \cos\theta + L \cos\varphi \right) - F \left( l \sin\theta + 2L \sin\varphi \right)
\end{align*}

Hence, the lagrangian of the system is given as follows,
\begin{align*}
	\mcL &= T-V\\
	\imps \mcL &= \lt( \frac{ML^2}{6} \rt) \dot{\varphi}^2 + \f M \lt[l^2\dot{\theta}^2 +L^2\dot{\varphi}^2 +2lL\dot{\theta}\dot{\varphi}\cos (\theta-\varphi)\rt] + Mg \left( l \cos\theta + L \cos\varphi \right) + F \left( l \sin\theta + 2L \sin\varphi \right)
\end{align*}

For each generalized coordinate \( q \), the Euler-Lagrange equation is:
\[
\dv{t}\left(\pdv{\mcL}{\dot{q}}\right) - \pdv{\mcL}{q} = 0
\]

Thus, the Euler-Lagrange Equations in terms of $\varphi$ give us,
  \begin{align*}
    &\dv{t} \left( \pdv{\mcL}{\dot{\varphi}} \right) = \pdv{\mcL}{\varphi} \\ \\
    \imps& \dv{t} \left[ \frac{4ML^2}{3}\dot{\varphi} + MlL\dot{\theta}\cos(\theta-\varphi) \right] = MlL\dot{\theta}\dot{\varphi}\sin(\theta-\varphi) - MgL\sin\varphi + 2FL\cos\varphi \\
    \imps& \frac{4ML^2}{3}\ddot{\varphi} + MlL\ddot{\theta}\cos(\theta-\varphi) -MlL\dot{\theta}^2\sin(\theta-\varphi)
    = - MgL\sin\varphi + 2FL\cos\varphi \\
    \imps& \ddot{\varphi} = \frac{3}{4L}\left[-l\ddot{\theta}\cos(\theta-\varphi) + l\dot{\theta}^2\sin(\theta-\varphi) - g\sin\varphi + \frac{2F}{M}\cos\varphi\right] \h \tag{1} \label{01} 
  \end{align*}
  Similarily, the Euler-Lagrange Equations in terms of $\theta$ give us,
  \begin{align*}
    & \dv{t} \left( \pdv{\mcL}{\dot{\theta}} \right) = \pdv{\mcL}{\theta} \\ \\
    \imps& \dv{t} \left[ Ml^2\dot{\theta} + MlL\dot{\varphi}\cos(\theta-\varphi) \right] = -MlL\dot{\theta}\dot{\varphi}\sin(\theta-\varphi) - Mgl\sin\theta + Fl\cos\theta \\
    \imps& Ml^2\ddot{\theta} + MlL\ddot{\varphi}\cos(\theta-\varphi) + MlL\dot{\varphi}^2\sin(\theta-\varphi) =  - Mgl\sin\theta + Fl\cos\theta \\
    \imps& \ddot{\theta} = \frac{1}{l }\left[-L\ddot{\varphi}\cos(\theta-\varphi) - L\dot{\varphi}^2\sin(\theta-\varphi) - g\sin\theta + \frac{F}{M}\cos\theta\right] \h \tag{2} \label{02}
  \end{align*}
Therefore, \eqref{01} and \eqref{02} give us the equations of motion for the given system. \\
}
\wb
\pagebreak

\end{document}
