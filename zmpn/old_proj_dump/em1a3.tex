\documentclass{report}

\input{latex-templates/preamble}
\newcommand{\eps}{\epsilon}
\newcommand{\veps}{\varepsilon}
\newcommand{\Qed}{\begin{flushright}\qed\end{flushright}}

\newcommand{\parinn}{\setlength{\parindent}{1cm}}
\newcommand{\parinf}{\setlength{\parindent}{0cm}}

% \newcommand{\norm}{\|\cdot\|}
\newcommand{\inorm}{\norm_{\infty}}
\newcommand{\opensets}{\{V_{\alpha}\}_{\alpha\in I}}
\newcommand{\oset}{V_{\alpha}}
\newcommand{\opset}[1]{V_{\alpha_{#1}}}
\newcommand{\lub}{\text{lub}}
\newcommand{\del}[2]{\frac{\partial #1}{\partial #2}}
\newcommand{\Del}[3]{\frac{\partial^{#1} #2}{\partial^{#1} #3}}
\newcommand{\deld}[2]{\dfrac{\partial #1}{\partial #2}}
\newcommand{\Deld}[3]{\dfrac{\partial^{#1} #2}{\partial^{#1} #3}}
\newcommand{\der}[2]{\frac{\mathrm{d} #1}{\mathrm{d} #2}}
% \newcommand{\ddd}[3]{\frac{\mathrm{d}^{#3} #1}{\mathrm{d}^{#3} #2}}
\newcommand{\lm}{\lambda}
\newcommand{\uin}{\mathbin{\rotatebox[origin=c]{90}{$\in$}}}
\newcommand{\usubset}{\mathbin{\rotatebox[origin=c]{90}{$\subset$}}}
\newcommand{\lt}{\left}
\newcommand{\rt}{\right}
\newcommand{\bs}[1]{\boldsymbol{#1}}
\newcommand{\exs}{\exists}
\newcommand{\st}{\strut}
\newcommand{\dps}[1]{\displaystyle{#1}}
\newcommand{\id}{\text{id}}
\newcommand{\imps}{\quad \Rightarrow \quad}
\newcommand{\cimps}{\quad \Leftrightarrow \quad}
\newcommand{\kyuki}[1]{\quad \quad \bqty{\because \eqref{#1}}}
\newcommand{\kyukifir}[2]{\quad \quad \bqty{\because \eqref{#1} \& \eqref{#2}}}
\newcommand{\boxdia}[2]{\begin{wrapfigure}{r}{#1\textwidth}
		\fbox{\includegraphics[width=\linewidth]{Figures/#2.png}}
	\end{wrapfigure}}
\newcommand{\dia}[2]{\begin{wrapfigure}{r}{#1\textwidth}
		\includegraphics[width=\linewidth]{Figures/#2.png}
	\end{wrapfigure}}
\newcommand{\boxudia}[2]{\begin{figure}[H]
		\centering
		\fbox{\includegraphics[width=#1\textwidth]{Figures/#2.png}}
		\end{figure}}
\newcommand{\udia}[2]{\begin{figure}[H]
		\centering
		\includegraphics[width=#1\textwidth]{Figures/#2.png}
	\end{figure}}
\newcommand{\su}[2]{\textcolor{my#1}{#2}}
\newcommand{\shs}[1]{\\ \textbf{{\Large #1}}\\}
\newcommand{\sss}[1]{\vspace*{-1cm} \subsubsection*{#1}}
\newcommand{\unt}[1]{\text{#1}}
\newcommand{\wa}{
	\noindent\rule{\textwidth}{0.4pt} 
	\vspace{0.5cm}}
\newcommand{\wb}{\noindent\rule{\textwidth}{0.4pt}}
\newcommand{\qmi}{\int_{-\infty}^{\infty}}
\newcommand{\qmk}{|\psi(x,0)|^{2}}
\newcommand{\qml}{\exp{-\frac{(x - x_0)^2}{4\sigma_0^2} + \frac{i}{\hbar}p_0 x}}
\newcommand{\qmls}{\exp{-\frac{(x - x_0)^2}{4\sigma_0^2} - \frac{i}{\hbar}p_0 x}}
\newcommand{\e}[1]{\exp\lt(#1\rt)}
\newcommand\prm[2][^n]{\prescript{#1\mkern-2.5mu}{}P_{#2}}
\newcommand\cmb[2][^n]{\prescript{#1\mkern-0.5mu}{}C_{#2}}
\newcommand{\ki}[1]{\lt[\therefore #1\rt]}
\newcommand{\h}{\underset{\rotatebox{135}{\#}}{}}
\newcommand{\f}{\frac{1}{2}}


%\newcommand{\sol}[1]{\vspace{0.5cm} 
%\setlength{\parindent}{0cm} \textcolor{mytheoremfr}{\textbf{\underline{Solution:}}} \textcolor{mytheoremfr}{#1}}
\newcommand{\solve}[1]{\setlength{\parindent}{0cm}\textbf{\textit{Solution: }}\setlength{\parindent}{1cm}#1 \Qed}

\input{latex-templates/letterfonts}
\setlength{\parindent}{0pt}
\usepackage{physics, siunitx}
\usepackage{float}
\usepackage{hyperref}
\usepackage{wrapfig}
\usepackage{pgfplots}
\setlength{\fboxsep}{4pt} % Space between image and border
\setlength{\fboxrule}{0.5pt} % Border thickness

\title{\Huge{\textbf{PC2031}}\\ \su{g}{Electricity and Magnetism 1} \\ {\huge \su{r}{Assignment 3}}}
\author{\huge{Parth Bhargava}\\ } %A0310667E
\date{\today}

\begin{document}
	\maketitle
	
	\pbm{}{
	A long wire carries a current $I = 100$ A. A rectangular loop, with dimensions $l = 10$ cm and $w = 8$ cm, is moving at a speed $v = 5$ m/s
  \udia{0.7}{rdf104}
  \begin{enumerate}
    \item[a.] At the instant moment when the loop is $d = 15$ cm away from the wire, determine the emf induced in the moving loop. Determine also
the direction of the induced current.
    \item[b.] The induced current itself generates an opposing emf as the loop moves away from the wire. What resistance value should the loop have to minimize the impact of this induced current? Determine the \textit{lower bound} for the resistance, providing a rough yet reasonable estimate.
  \end{enumerate}
  }
  \wb
  \sol{}{
    \wa
    \sss{a.}\\
    The magnetic field due to a long straight wire at distance $r$ is given by:
    \begin{align*}
      \vec{B}(r) &= \frac{\mu_0 I}{2\pi r}
    \end{align*}
    Where $\mu_0 = 4\pi \times 10^{-7}$ H/m is the permeability of free space.
    To calculate the magnetic flux through the loop at any instant, we need to integrate the field over the loop area. Since the field varies with distance from the wire, we must integrate:
    \begin{align*}
      \Phi &= \int \vec{B} \cdot d\vec{A} \\
      &= \int_{d}^{d+l} \frac{\mu_0 I}{2\pi r} w \, dr \\
      &= \frac{\mu_0 I w}{2\pi} \ln\left(\frac{d+l}{d}\right)
    \end{align*}
    The induced EMF is:
    \begin{align*}
      \mathcal{E} &= -\dv{\Phi}{t} \\
      &= -\dv{\Phi}{d}\cdot v\\
      &= -v \cdot \dv{d} \left[\frac{\mu_0 I w}{2\pi} \ln\left(\frac{d+l}{d}\right)\right]\\
      &= -v \cdot \frac{\mu_0 I w}{2\pi} \cdot \dv{d} \left[\ln\left(\frac{d+l}{d}\right)\right] \\
      &= -v \cdot \frac{\mu_0 I w}{2\pi} \cdot \lt(\frac{d}{d+l}\rt)\lt(-\frac{l}{d^2}\rt) \\
      &= \frac{\mu_0 I w lv}{2\pi d(d+l)}\h
    \end{align*}
    At the given moment, $d = 15$ cm and $d+l = 25$ cm. Therefore, the induced EMF is:
    \begin{align*}
      \mathcal{E} &= \frac{\mu_0 I l w v}{2\pi d(d+l)} \\
      &= \frac{4\pi \times 10^{-7} \times 100 \times 10 \times 8 \times 5 \times 10^{-4}}{2\pi \times 15 \times 25 \times 10^{-4}} \\
      &= \frac{2 \times 10^{-7} \times 4 \times 8 \times 10}{3} \\
      &= 21.\bar{3} \times 10^{-6} \text{ V} \\
      &= 21.\bar{3} \, \mu\text{V }\h
    \end{align*}
    The positive sign indicates the direction of the induced EMF. According to Lenz's law, the induced current will create a magnetic field that opposes the decrease in flux as the loop moves away from the wire. Therefore, the induced current will flow in a clockwise direction when viewed from above the diagram.\\
    \wa
    \sss{b.}\\ 
In this part, we are concerned with the magnetic field generated by the induced current circulating within the moving loop. As the loop recedes from the infinitely long wire, the magnitude of this induced current varies, which in turn alters the magnetic field it produces. According to Faraday's law of electromagnetic induction, any change in magnetic flux through the loop induces an additional emf. Our goal is to ensure that this secondary emf—arising from the loop’s own induced current—is negligible in comparison to the primary emf calculated in part (a).

In essence, we require that:
induced emf $\gg$ emf due to the induced current’s own magnetic field.

We have,
\begin{align*}
    I_{\text{ind}}= \frac{\mcE}{R} = \frac{\mu_0 I wlv}{2 \pi d(d+l)R}
\end{align*}
Let $\phi_{\text{ind}} = K I_{\text{ind}} \text{ since } \left[ \phi_{\text{ind}} \propto I_{\text{ind}} \right] $,
\begin{align*}
  \mcE_{\text{ind}} = \dv{\phi_{\text{ind}}}{t} &= K \dv{I_{\text{ind}}}{t} \\
    &= K \dv{t} \left( \frac{\mu_0 I w l v}{2\pi d(d + \ell) R} \right) \\
    &= \frac{K \mu_0 I w l v}{2\pi R} \cdot \left( \frac{(2d + \ell)v}{(d^2(d+\ell)^2)} \right) \\
\end{align*}
Now,
\begin{align*}
    \mcE  &\gg \frac{K \mu_0 I w l v}{2\pi R} \cdot \left( \frac{(2d + \ell)v}{(d^2(d+\ell)^2)} \right) \\
    \mcE &\gg \frac{K \mcE (2d + \ell)v}{Rd(d+\ell)} \\
    R &\gg \frac{K (2d + \ell)v}{d(d+\ell)}\\
    R &\gtrsim \frac{100K (2d + \ell)v}{d(d+\ell)}
\end{align*}

}
\wb	
	\pbm{}{
    Three electric dipoles ($A$, $B$ and $C$ ), each with an electric dipole moment $\vb{p}$, are placed at equal distances along a straight line.
  \udia{0.7}{rdf105}
      Calculate the total force $\vb{F}$ acting on dipole $C$ due to the presence of dipoles $A$ and $B$. Express your result in terms of the magnitude of $\vb{F}$ and its orientation (angle) with respect to the horizontal axis.
    }
  \wb
  \sol{}{
    \wb\\
The interaction energy between two electric dipoles separated by a displacement vector $\vec{r}$ (from the first dipole to the second) is given by:

\begin{align*}
U &= \frac{1}{4\pi\varepsilon_0}\left[\frac{\vec{p}_1 \cdot \vec{p}_2}{r^3} - \frac{3(\vec{p}_1 \cdot \vec{r})(\vec{p}_2 \cdot \vec{r})}{r^5}\right]
\end{align*}

The force on the second dipole due to the first can be found as:

\begin{align*}
\vec{F} &= -\vec{\nabla}U
\end{align*}

\subsection*{Force Due to Dipole A}

For dipoles $A$ and $C$:

\begin{align*}
  \vec{p}_A \cdot \vec{p}_C &= p\hat{y} \cdot p\hat{x} = 0 \\
  \vec{p}_A \cdot \vec{r}_{AC} &= p\hat{y} \cdot 2b\hat{x} = 0 \\
  \vec{p}_C \cdot \vec{r}_{AC} &= p\hat{x} \cdot 2b\hat{x} = 2pb
\end{align*}

Now we can compute the interaction energy:

\begin{align*}
  U_{AC} &= \frac{1}{4\pi\varepsilon_0}\left[\frac{\vec{p}_A \cdot \vec{p}_C}{r_{AC}^3} - \frac{3(\vec{p}_A \cdot \vec{r}_{AC})(\vec{p}_C \cdot \vec{r}_{AC})}{r_{AC}^5}\right] \\
  &= \frac{1}{4\pi\varepsilon_0}\left[\frac{0}{(2b)^3} - \frac{3(0)(2pb)}{(2b)^5}\right] \\
  &= 0
\end{align*}

Since the interaction energy is zero, the force on dipole $C$ due to dipole $A$ is also zero:

\begin{align*}
\vec{F}_{AC} &= -\vec{\nabla}U_{AC} = -\vec{\nabla}(0) = \vec{0}
\end{align*}

This result makes physical sense because dipole $A$ is oriented perpendicular to both the line connecting it to dipole $C$ and to dipole $C$'s orientation.

\subsection*{Force Due to Dipole B}

For dipoles $B$ and $C$:

\begin{align*}
  \vec{p}_B \cdot \vec{p}_C &= p\hat{x} \cdot p\hat{x} = p^2 \\
  \vec{p}_B \cdot \vec{r}_{BC} &= p\hat{x} \cdot b\hat{x} = pb \\
  \vec{p_}_C \cdot \vec{r}_{BC} &= p\hat{x} \cdot b\hat{x} = pb
\end{align*}

Now we can compute the interaction energy:

\begin{align*}
  U_{BC} &= \frac{1}{4\pi\varepsilon_0}\left[\frac{\vec{p}_B \cdot \vec{p}_C}{r_{BC}^3} - \frac{3(\vec{p}_B \cdot \vec{r}_{BC})(\vec{p}_C \cdot \vec{r}_{BC})}{r_{BC}^5}\right] \\
  &= \frac{1}{4\pi\varepsilon_0}\left[\frac{p^2}{b^3} - \frac{3(pb)(pb)}{b^5}\right] \\
  &= \frac{p^2}{4\pi\varepsilon_0 b^3}[1 - 3] \\
  &= -\frac{2p^2}{4\pi\varepsilon_0 b^3}
\end{align*}

To find the force, we take the negative gradient of the energy:

\begin{align*}
  \vec{F}_{BC} &= -\vec{\nabla}U_{BC} \\
  &= -\frac{\partial}{\partial r_{BC}}\left(-\frac{2p^2}{4\pi\varepsilon_0 r_{BC}^3}\right)\hat{r}_{BC} \\
  &= \frac{2p^2}{4\pi\varepsilon_0}\frac{\partial}{\partial r_{BC}}\left(\frac{1}{r_{BC}^3}\right)\hat{r}_{BC} \\
  &= \frac{2p^2}{4\pi\varepsilon_0}\left(-\frac{3}{r_{BC}^4}\right)\hat{r}_{BC} \\
  &= -\frac{6p^2}{4\pi\varepsilon_0 r_{BC}^4}\hat{r}_{BC}\\
  \vec{F}_{BC} &= -\frac{6p^2}{4\pi\varepsilon_0 b^4}\hat{x}
\end{align*}

The negative sign indicates that the force is in the direction opposite to $\hat{r}_{BC}$, meaning it's a repulsive force pushing dipole $C$ away from dipole $B$.

\subsection*{Total Force on Dipole C}

The total force on dipole $C$ is the vector sum of the forces due to dipoles $A$ and $B$:

\begin{align*}
\vec{F} &= \vec{F}_{AC} + \vec{F}_{BC} \\
&= \vec{0} + \left(-\frac{6p^2}{4\pi\varepsilon_0 b^4}\hat{x}\right) \\
&= -\frac{6p^2}{4\pi\varepsilon_0 b^4}\hat{x}
\end{align*}

\subsection*{Final Answer}

The magnitude of the total force on dipole $C$ is:

\begin{align*}
|\vec{F}| &= \frac{6p^2}{4\pi\varepsilon_0 b^4}\h
\end{align*}

The orientation of the force is 180° with respect to the horizontal axis (pointing in the negative x-direction).

This result makes physical sense because:
\begin{enumerate}
  \item Dipole $A$ exerts no force on $C$ due to their perpendicular orientations

  \item Dipoles $B$ and $C$ are parallel and aligned along the same axis, creating a attractive force between them
\end{enumerate}
 
  }
	\wa
\end{document}
