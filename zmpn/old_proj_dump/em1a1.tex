\documentclass{report}

\input{latex-templates/preamble}
\newcommand{\eps}{\epsilon}
\newcommand{\veps}{\varepsilon}
\newcommand{\Qed}{\begin{flushright}\qed\end{flushright}}

\newcommand{\parinn}{\setlength{\parindent}{1cm}}
\newcommand{\parinf}{\setlength{\parindent}{0cm}}

% \newcommand{\norm}{\|\cdot\|}
\newcommand{\inorm}{\norm_{\infty}}
\newcommand{\opensets}{\{V_{\alpha}\}_{\alpha\in I}}
\newcommand{\oset}{V_{\alpha}}
\newcommand{\opset}[1]{V_{\alpha_{#1}}}
\newcommand{\lub}{\text{lub}}
\newcommand{\del}[2]{\frac{\partial #1}{\partial #2}}
\newcommand{\Del}[3]{\frac{\partial^{#1} #2}{\partial^{#1} #3}}
\newcommand{\deld}[2]{\dfrac{\partial #1}{\partial #2}}
\newcommand{\Deld}[3]{\dfrac{\partial^{#1} #2}{\partial^{#1} #3}}
\newcommand{\der}[2]{\frac{\mathrm{d} #1}{\mathrm{d} #2}}
% \newcommand{\ddd}[3]{\frac{\mathrm{d}^{#3} #1}{\mathrm{d}^{#3} #2}}
\newcommand{\lm}{\lambda}
\newcommand{\uin}{\mathbin{\rotatebox[origin=c]{90}{$\in$}}}
\newcommand{\usubset}{\mathbin{\rotatebox[origin=c]{90}{$\subset$}}}
\newcommand{\lt}{\left}
\newcommand{\rt}{\right}
\newcommand{\bs}[1]{\boldsymbol{#1}}
\newcommand{\exs}{\exists}
\newcommand{\st}{\strut}
\newcommand{\dps}[1]{\displaystyle{#1}}
\newcommand{\id}{\text{id}}
\newcommand{\imps}{\quad \Rightarrow \quad}
\newcommand{\cimps}{\quad \Leftrightarrow \quad}
\newcommand{\kyuki}[1]{\quad \quad \bqty{\because \eqref{#1}}}
\newcommand{\kyukifir}[2]{\quad \quad \bqty{\because \eqref{#1} \& \eqref{#2}}}
\newcommand{\boxdia}[2]{\begin{wrapfigure}{r}{#1\textwidth}
		\fbox{\includegraphics[width=\linewidth]{Figures/#2.png}}
	\end{wrapfigure}}
\newcommand{\dia}[2]{\begin{wrapfigure}{r}{#1\textwidth}
		\includegraphics[width=\linewidth]{Figures/#2.png}
	\end{wrapfigure}}
\newcommand{\boxudia}[2]{\begin{figure}[H]
		\centering
		\fbox{\includegraphics[width=#1\textwidth]{Figures/#2.png}}
		\end{figure}}
\newcommand{\udia}[2]{\begin{figure}[H]
		\centering
		\includegraphics[width=#1\textwidth]{Figures/#2.png}
	\end{figure}}
\newcommand{\su}[2]{\textcolor{my#1}{#2}}
\newcommand{\shs}[1]{\\ \textbf{{\Large #1}}\\}
\newcommand{\sss}[1]{\vspace*{-1cm} \subsubsection*{#1}}
\newcommand{\unt}[1]{\text{#1}}
\newcommand{\wa}{
	\noindent\rule{\textwidth}{0.4pt} 
	\vspace{0.5cm}}
\newcommand{\wb}{\noindent\rule{\textwidth}{0.4pt}}
\newcommand{\qmi}{\int_{-\infty}^{\infty}}
\newcommand{\qmk}{|\psi(x,0)|^{2}}
\newcommand{\qml}{\exp{-\frac{(x - x_0)^2}{4\sigma_0^2} + \frac{i}{\hbar}p_0 x}}
\newcommand{\qmls}{\exp{-\frac{(x - x_0)^2}{4\sigma_0^2} - \frac{i}{\hbar}p_0 x}}
\newcommand{\e}[1]{\exp\lt(#1\rt)}
\newcommand\prm[2][^n]{\prescript{#1\mkern-2.5mu}{}P_{#2}}
\newcommand\cmb[2][^n]{\prescript{#1\mkern-0.5mu}{}C_{#2}}
\newcommand{\ki}[1]{\lt[\therefore #1\rt]}
\newcommand{\h}{\underset{\rotatebox{135}{\#}}{}}
\newcommand{\f}{\frac{1}{2}}


%\newcommand{\sol}[1]{\vspace{0.5cm} 
%\setlength{\parindent}{0cm} \textcolor{mytheoremfr}{\textbf{\underline{Solution:}}} \textcolor{mytheoremfr}{#1}}
\newcommand{\solve}[1]{\setlength{\parindent}{0cm}\textbf{\textit{Solution: }}\setlength{\parindent}{1cm}#1 \Qed}

\input{latex-templates/letterfonts}
\usepackage{physics}
\usepackage{float}
\usepackage{hyperref}
\usepackage{wrapfig}
\usepackage{pgfplots}
\setlength{\fboxsep}{1pt} % Space between image and border
\setlength{\fboxrule}{0.5pt} % Border thickness

\title{\Huge{PC2031 Electricity and Magnetism 1}\\Assignment 1}
\author{\huge{Parth Bhargava}\\ AO310667E}
\date{}

\begin{document}
	\maketitle
	
	\pbm{}{
	Calculate the following integral
	$$\int \lt[ \vb{a} (\dot{\vb{b}} \cdot \vb{a} + \vb{b} \cdot \dot{\vb{a}}) + \dot{\vb{a}} (\vb{b} \cdot \vb{a}) - 2 (\dot{\vb{a}} \cdot \vb{a}) \vb{b} - \dot{\vb{b}} |\vb{a}|^2 \rt] \dd t$$
	where $\dot{\vb{a}}$ and $\dot{\vb{b}}$ are the derivatives of $\vb{a}$ and $\vb{b}$ with respect to $t$. Simplify your result and express it in the form of a vector triple product.\\
	}
	\sol{}{
	\begin{align*}
		\vb{a} (\dot{\vb{b}} \cdot \vb{a} + \vb{b} \cdot \dot{\vb{a}}) + \dot{\vb{a}} (\vb{b} \cdot \vb{a}) - 2 (\dot{\vb{a}} \cdot \vb{a}) \vb{b} - \dot{\vb{b}} |\vb{a}|^2 &=\vb{a} \dv{t} \pqty{\vb{b} \cdot \vb{a}} + \dv{\vb{a}}{t} \pqty{\vb{b} \cdot \vb{a}} - \lt[ \pqty{\dv{\vb{a}}{t} \cdot \vb{a}} + \pqty{\vb{a} \cdot \dv{\vb{a}}{t}} \rt] \vb{b} - \pqty{\vb{a} \cdot \vb{a}} \dv{\vb{b}}{t}\\
		&= \dv{t} \lt[ (\vb{b} \cdot \vb{a})\vb{a} \rt] - \lt[ \dv{\pqty{\vb{a} \cdot \vb{a}}}{t} \vb{b} + \pqty{\vb{a} \cdot \vb{a}} \dv{\vb{b}}{t} \rt]\\
		&= \dv{t} \lt[(\vb{b} \cdot \vb{a})\vb{a} - (\vb{a} \cdot \vb{a}) \vb{b} \rt]\\
		&=\dv{t} [\vb{a} \times (\vb{a} \times \vb{b} )]
	\end{align*}
	Hence,
	\begin{equation*}
		\int \lt[ \vb{a} (\dot{\vb{b}} \cdot \vb{a} + \vb{b} \cdot \dot{\vb{a}}) + \dot{\vb{a}} (\vb{b} \cdot \vb{a}) - 2 (\dot{\vb{a}} \cdot \vb{a}) \vb{b} - \dot{\vb{b}} |\vb{a}|^2 \rt] \dd t = \int \lt[ \dv{t} [\vb{a} \times (\vb{a} \times \vb{b} )] \rt] \dd t = \vb{a} \times (\vb{a} \times \vb{b} )
	\end{equation*}
	\\
	}
	\wa
	
	\pbm{}{
	Three curves, $C_1$, $C_2$ and $C_3$, all starting at point $P_0 = (0, 0)$ and ending at $P_1 = (1, 1)$, are shown below:
	\udia{0.3}{rdf32}
	$C_1$ is a straight line from $P_0$ to $P_1$; $C_2$ is a path consisting of 2 straight segments, and $C_3$ is a part of the parabola $x = y^2$. Suppose a vector field is given by
	$$\mathbf{F} = xy \vu*{x} + (y^2 + 1) \vu*{y}$$
	Calculate the line integral $\int_{C_i} \mathbf{F} \cdot \dd \mathbf{r}$ for each of the 3 curves from $P_0$ to $P_1$.
	}
	\pagebreak
	\sol{}{
	\sss{Along $C_1$,} 
	\begin{align*}
		\int_{C_1} \vb{F} \cdot \dd{\vb{r}} &= \int_{C_1} (xy \vu*{x} + (y^2 + 1) \vu*{y}) \cdot (\dd{x} \vu*{x} + \dd{y} \vu*{y}) \quad [\because \dd{\vb{r}} = \dd{x} \vu*{x} + \dd{y}\vu*{y}]\\
		&= \int_{C_1} (xy \dd{x} + (y^2 + 1) \dd{y})
	\end{align*}
	At any point on $C_1$, $x = y \Rightarrow \dd{x} = \dd{y}$,
	\begin{align*}
		\therefore \int_{C_1} \vb{F} \cdot \dd{\vb{r}} &= \int_{C_1} (x^2 \dd{x} + (x^2 + 1) \dd{x})\\
		&= \int_0^1 (2x^2 + 1) \dd{x}\\
		&= \lt[ \frac{2}{3} x^3 + x \rt]_0^1\\
		\imps \int_{C_1} \vb{F} \cdot \dd{\vb{r}} &= \frac{5}{3}
	\end{align*}
	\wa
	\sss{Along $C_2$,}
	\begin{align*}
		\int_{C_2} \vb{F} \cdot \dd{\vb{r}} &= \int_0^1 (F_x)_{y=0} \dd{x} + \int_0^1 (F_y)_{x=1} \dd{y}\\
		&= \int_0^1 (0) \dd{x} + \int_0^1 (y^2 + 1) \dd{y}\\
		&= \int_0^1 (y^2 + 1) \dd{y}\\
		&= \lt[ \frac{y^3}{3} + y \rt]_0^1\\
		\Rightarrow \int_{C_2} \vb{F} \cdot \dd{\vb{r}} &= \frac{4}{3}
	\end{align*}
	\wa
	\sss{Along $C_3$,} 
	
	\begin{align*}
		\int_{C_3} \vb{F} \cdot \dd{\vb{r}} &= \int_{C_3} (xy \vu*{x} + (y^2 + 1) \vu*{y}) \cdot (\dd{x} \vu*{x} + \dd{y} \vu*{y}) \quad [\because \dd{\vb{r}} = \dd{x} \vu*{x} + \dd{y} \vu*{y}]\\
		&= \int_{C_3} (xy \dd{x} + (y^2 + 1) \dd{y})
	\end{align*}
	
	At any point on $C_3$, $x = y^2 \Rightarrow \dd{x} = 2y \dd{y}$,
	\begin{align*}
		\therefore \int_{C_3} \vb{F} \cdot \dd{\vb{r}} &= \int_{C_3} (y^2 \cdot y \cdot 2y \dd{y} + (y^2 + 1) \dd{y})\\
		&= \int_0^1 (2y^4 + y^2 + 1) \dd{y}\\
		&= \lt[ \frac{2y^5}{5} + \frac{y^3}{3} + y \rt]_0^1\\
		&= \frac{2}{5} + \frac{1}{3} + 1\\
		\int_{C_3} \vb{F} \cdot \dd{\vb{r}} &= \frac{26}{15}
	\end{align*}
	}
	\wa
	
	\pbm{}{
	A vector field $\mathbf{F}$ is given by
	$$\mathbf{F}(x, y, z) = 15z \vu*{x} - 10 \vu*{y} + 5y \vu*{z}$$
	A surface $S$ defined by
	$$4x + 5y + 8z = 20$$
	is located inside the region $0 \leq x \leq \infty$, $0 \leq y \leq \infty$, $0 \leq z \leq \infty$ i.e within the first octant.
	Calculate the flux of $\mathbf{F}$ over this surface i.e $\int_S \mathbf{F}(x, y, z) \cdot \dd \mathbf{a}$.
	\\
	}
	\sol{}{
	\sss{Method 1}
	We are given the vector field $\mathbf{F}(x, y, z) = 15z \vu*{x} - 10 \vu*{y} + 5y \vu*{z}$ and the surface $S$ defined by $4x + 5y + 8z = 20$ within the first octant. We need to calculate the flux of $\mathbf{F}$ over this surface, i.e., $\int_S \mathbf{F} \cdot \dd\mathbf{a}$.\\
	Here,
	\begin{align*}
		4x + 5y + 8z &= 20 \imps 
		\begin{cases}
			x = \dfrac{20 - 5y - 8z}{4} = 5 - \dfrac{5}{4}y - 2z, \\ \\
			y = \dfrac{20 - 4x - 8z}{5} = 4 - \dfrac{4}{5}x - \dfrac{8}{5}z.
		\end{cases}
	\end{align*}
	Hence, 
	\begin{align*}
		\int_S \mathbf{F}(x, y, z) \cdot \dd \mathbf{a} &= \int_{z=0}^{\frac{5}{2}} \int_{y=0}^{4-\frac{8}{5}z} 15z \, \dd y \, \dd z + \int_{z=0}^{\frac{5}{2}} \int_{x=0}^{5-2z} -10 \, \dd x \, \dd z + \int_{y=0}^{4} \int_{x=0}^{5-\frac{5}{4}y} 5y \, \dd x \, \dd y.
		\\
		&= \int_{z=0}^{\frac{5}{2}} \lt[ 15zy \rt]_{y=0}^{4-\frac{8}{5}z} \dd z + \int_{z=0}^{\frac{5}{2}} \lt[ -10x \rt]_{x=0}^{5-2z} \dd z + \int_{y=0}^{4} \lt[ 5yx \rt]_{x=0}^{5-\frac{5}{4}y} \dd y.
		\\
		&= \int_{z=0}^{\frac{5}{2}} \lt( 60z - 24z^2 \rt) \dd z + \int_{z=0}^{\frac{5}{2}} \lt( -50 + 20z \rt) \dd z + \int_{y=0}^{4} \lt( 25y - \frac{25}{4}y^2 \rt) \dd y.
		\\
		&= \lt[ 30z^2 - 8z^3 \rt]_{z=0}^{\frac{5}{2}} + \lt[ 10z^2 - 50z \rt]_{z=0}^{\frac{5}{2}} + \lt[ \frac{25}{2}y^2 - \frac{25}{12}y^3 \rt]_{y=0}^{4}.
		\\
		\imps \int_S \mathbf{F} \cdot \dd\mathbf{a}&= \frac{200}{3}
	\end{align*}
	\wa
	\udia{0.8}{rdf33}
	\wa
	\sss{Method 2}
	We are given the vector field $\mathbf{F}(x, y, z) = 15z \vu*{x} - 10 \vu*{y} + 5y \vu*{z}$ and the surface $S$ defined by $4x + 5y + 8z = 20$ within the first octant. We need to calculate the flux of $\mathbf{F}$ over this surface, i.e., $\int_S \mathbf{F} \cdot \dd\mathbf{a}$.\\
	Here,
	\begin{align*}
		\dd\mathbf{a} = \lt( \pdv{\mathbf{r}}{x} \times \pdv{\mathbf{r}}{y} \rt) \dd x \dd y.
	\end{align*}
	The position vector $\mathbf{r}(x, y)$ is:
	$$ \mathbf{r}(x, y) = x \vu*{x} + y \vu*{y} + \lt( \frac{20 - 4x - 5y}{8} \rt) \vu*{z}. $$
	Compute the partial derivatives:
	$$ \pdv{\mathbf{r}}{x} = \vu*{x} - \frac{4}{8} \vu*{z} = \vu*{x} - \frac{1}{2} \vu*{z} \qc \pdv{\mathbf{r}}{y} = \vu*{y} - \frac{5}{8} \vu*{z}. $$
	Now compute the cross product:
	\begin{align*}
		\pdv{\mathbf{r}}{x} \times \pdv{\mathbf{r}}{y} &= \begin{vmatrix}
			\vu*{x} & \vu*{y} & \vu*{z} \\
			1 & 0 & -\frac{1}{2} \\
			0 & 1 & -\frac{5}{8}
		\end{vmatrix}\\
		&= \vu*{x} \lt( 0 - \lt(-\frac{1}{2}\rt) \rt) - \vu*{y} \lt( 1 \cdot \lt(-\frac{5}{8}\rt) - 0 \rt) + \vu*{z} \lt( 1 \cdot 1 - 0 \rt)\\
		&= \frac{1}{2} \vu*{x} + \frac{5}{8} \vu*{y} + \vu*{z}\\
		\imps \dd\mathbf{a} &= \lt( \frac{5}{8} \vu*{x} + \frac{1}{2} \vu*{y} + \vu*{z} \rt) \dd x \dd y
	\end{align*}
	Hence,
	\begin{align*}
		\int_S \mathbf{F} \cdot \dd\mathbf{a} &= \iint_S (15z \vu*{x} - 10 \vu*{y} + 5y \vu*{z}) \cdot \lt( \frac{1}{2} \vu*{x} + \frac{5}{8} \vu*{y} + \vu*{z} \rt) \dd x \dd y \\
		&= \iint_S \lt( 15z \cdot \frac{1}{2} + (-10) \cdot \frac{5}{8} + 5y \cdot 1 \rt) \dd x \dd y \\
		&= \iint_S \lt( \frac{15z}{2} - \frac{50}{8} + 5y \rt) \dd x \dd y \\
		&= \iint_S \lt( \frac{60z - 50 + 40y}{8} \rt) \dd x \dd y\\
		&= \iint_S \lt( \frac{60 \lt( \frac{20 - 4x - 5y}{8} \rt) - 50 + 40y}{8} \rt) \dd x \dd y  \quad \lt[ \because z = \frac{20 - 4x - 5y}{8} \rt]\\
		&= \iint_S \lt( \frac{60(20 - 4x - 5y) - 400 + 320y}{64} \rt) \dd x \dd y \\
		&= \iint_S \lt( \frac{1200 - 240x - 300y - 400 + 320y}{64} \rt) \dd x \dd y \\
		&= \iint_S \lt( \frac{800 - 240x + 20y}{64} \rt) \dd x \dd y\\
		&=\int_0^5 \int_0^{4 - \frac{4x}{5}} \lt( \frac{800 - 240x + 20y}{64} \rt) \dd y \dd x \quad \lt[\because \text{The limits for $x$ and $y$ are determined by the plane} \atop \text{intersecting the coordinate axes: $4x + 5y \leq 20$.}\rt]\\
		&=\int_0^5 \frac{1}{64} \lt(\int_0^{4 - \frac{4x}{5}} (800 - 240x + 20y) \dd y\rt) \dd x\\
		&=\int_0^5 \frac{1}{64} \lt[ (800 - 240x)y + \frac{20y^2}{2} \rt]_0^{4 - \frac{4x}{5}} \dd x\\
		&= \frac{1}{64} \int_0^5 \lt[ (800 - 240x)\lt(4 - \frac{4x}{5}\rt) + 10\lt(4 - \frac{4x}{5}\rt)^2 \rt] \dd x \\
		&= \frac{1}{64} \int_0^5 \lt[ \lt(3200 - 640x - 960x + \frac{960x^2}{5}\rt) + 10\lt(16 - \frac{32x}{5} + \frac{16x^2}{25}\rt) \rt] \dd x \\
		&= \frac{1}{64} \int_0^5 \lt[ \lt(3200 - 1600x + \frac{960x^2}{5}\rt) + \lt(160 - \frac{320x}{5} + \frac{160x^2}{25}\rt) \rt] \dd x \\
		&= \frac{1}{64} \int_0^5 \lt[ 3360 - 1600x - \frac{320x}{5} + \frac{4960x^2}{25} \rt] \dd x \\
		&= \int_0^5 \frac{105}{2} \dd x - \int_0^5 26x \dd x + \int_0^5 \frac{31x^2}{10} \dd x \\
		&=  \frac{105}{2} x \eval_0^5 -  13x^2 \eval_0^5 +  \frac{31x^3}{30} \eval_0^5 \\
		&=  \frac{525}{2} -325 + \frac{775}{6}\\
		\imps \int_S \mathbf{F} \cdot \dd\mathbf{a}&= \frac{200}{3}\\
	\end{align*}
	}
	\wa
	
\end{document}