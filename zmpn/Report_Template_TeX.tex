\documentclass[a4paper,11pt]{article}

% Packages
\usepackage[margin=2cm]{geometry}
\usepackage{amsmath}
\usepackage{amssymb}
\usepackage{graphicx}
\usepackage{caption}
\usepackage{subcaption}
\usepackage{float}
\usepackage{cite}
\usepackage{url}
\usepackage{setspace}
\usepackage{titlesec}
\usepackage{fancyhdr}
\usepackage{xcolor}

% Page setup
\pagestyle{fancy}
\fancyhf{}
\fancyhead[R]{\thepage}
\renewcommand{\headrulewidth}{0pt}

% Section formatting
\titleformat{\section}{\large\bfseries}{\thesection.}{0.5em}{}
\titleformat{\subsection}{\normalsize\bfseries}{\thesubsection}{0.5em}{}

% Title page information
\title{\textbf{Title of Experiment}}
\author{Submitter's Name (Matric No.)}
\date{Experiment A, DD-MMM-YYYY}

\begin{document}

\maketitle

% Guidelines box
\begin{center}
\fbox{\begin{minipage}{0.9\textwidth}
\textbf{\underline{Guidelines and Reminders}}
\begin{itemize}
\item \textit{Use normal margins (2.0-2.5 cm); standard font type with readable size (e.g. Times New Roman 12 pt).}
\item \textit{The total report length (sections 2-6) should be approximately 800-1500 words.}
\item \textit{The report should include 2-6 Figures. Each figure may have multiple sub-panels and/or multiple curves.}
\item \textit{Number all pages, sections, equations, figures, and tables.}
\item \textit{First page: include the title of experiment, submitter's name (date), submission details (e.g. expt A, date)}
\item \textit{Write only what you understand - avoid unsupported claims or interpretations copied without thought}
\item \textit{Extract quantitative results from every experiment, using physics-based fitting functions where possible}
\item \textit{Include uncertainties with all reported values; present your figures clearly and concisely}
\end{itemize}
\end{minipage}}
\end{center}

\section{Abstract}
The abstract should be a single paragraph (no more than 150 words). The abstract should provide:
\begin{itemize}
\item An opening sentence that states the question/problem addressed by the experiment
\item Enough background content to give context to the experiment
\item A brief statement of primary quantitative results with uncertainties
\item A short concluding sentence
\end{itemize}
\textbf{Note:} Written last but appears first; self-contained with no references to figures or equations.

\section{Background and Objectives}
The report should start with a brief introduction covering:
\begin{itemize}
\item Physical phenomenon being investigated and its significance
\item Theoretical background including relevant formulas and equations
\item Specific aims and objectives of your experiment
\item Context: why this measurement matters and how it fits into broader physics
\end{itemize}
Provide sufficient background to make the report understandable to technical readers. Avoid excessive historical descriptions and technological applications.

\section{Experimental Methods}
This section should provide sufficient information to allow replication of your results:
\begin{itemize}
\item \textbf{Experimental Design:} Describe objectives, design, and prespecified components
\item \textbf{Apparatus:} Include photograph of experimental setup, and, if useful, a labeled schematic
\item \textbf{Procedures:} Written in past tense, third person
\item \textbf{Equipment specifications:} Include instrument uncertainties and calibration details
\item \textbf{Safety considerations:} Note any special precautions taken
\end{itemize}
% Example figure placeholder
% \begin{figure}[H]
% \centering
% \includegraphics[width=0.7\textwidth]{experimental_setup.jpg}
% \caption{Experimental setup showing [describe key components]}
% \label{fig:setup}
% \end{figure}

\section{Results and Analysis}
This will be the main section of your report. Present your experimental findings in figures (and tables, if needed) with appropriate analysis, including fitting and uncertainties.
\subsection*{Data Presentation Requirements}
\begin{itemize}
\item Show raw data first, then analyzed results
\item All data must be presented either in main text or supplementary materials
\item \textbf{Error bars required on all key experimental results}
\item Consolidate related datasets as curves within single panel, or panels within single figures
\item Call out figures in numerical order (cannot reference Fig. 3 before Fig. 2)
\end{itemize}
\subsection*{Figure Requirements}
\begin{itemize}
\item Font size: 75-100\% of main text size
\item Data representation: Experimental data as dots/points with error bars; fitted curves as lines
\item Plot ranges: Use meaningful x,y ranges; employ semi-log or log-log scales when appropriate
\item Labels: Clear axis labels with units, legends for multiple datasets
\item Captions: Descriptive captions explaining what figures show
\end{itemize}
\subsection*{Content Organization}
\begin{itemize}
\item Divide into descriptive subsections for different experimental themes
\item Subheadings should be brief phrases ($\leq$10 words)
\item Include sample calculations where helpful
\item Justify fitting procedures and choice of fitting functions
\item Display equations should be numbered: (1), (2), etc.
\item Error analysis: include clear descriptions of statistical methods, uncertainty propagation calculations, identification of dominant error sources, sufficient detail for result verification
\end{itemize}

% Example equation
% The relationship between variables A and B is given by:
% \begin{equation}
% A = k \cdot B^n
% \label{eq:power_law}
% \end{equation}
% where $k$ is the proportionality constant and $n$ is the power law exponent.
% Example figure with data
% \begin{figure}[H]
% \centering
% \includegraphics[width=0.8\textwidth]{data_plot.pdf}
% \caption{Experimental data (points with error bars) and theoretical fit (solid line). The fit yields $k = (2.3 \pm 0.1) \times 10^{-3}$ and $n = 1.95 \pm 0.05$.}
% \label{fig:data}
% \end{figure}

\section{Discussion}
Interpret your results and evaluate their significance:
\begin{itemize}
\item \textbf{Physical interpretation:} What do your results mean physically?
\item \textbf{Comparison with theory:} Compare measured values with expectations
\item \textbf{Error analysis:} Identify, quantify, and rank uncertainty sources
\item \textbf{Explain discrepancies:} Provide quantitative explanations (avoid vague "human error")
\item \textbf{Limitations:} Discuss limitations of your results and interpretation
\item \textbf{Future improvements:} Steps for potential enhancements
\end{itemize}

\section{Conclusion}
Provide a concluding paragraph that:
\begin{itemize}
\item Restates objectives and how they were addressed
\item Summarizes key findings with measured values and uncertainties, final quantitative results
\item Suggests potential extensions or applications
\end{itemize}

\section{References}
Numbered list of all sources cited in the report.

\section{Annex (Optional)}
Include auxiliary information that:
\begin{itemize}
\item Contains control or supplemental experiments
\item Offers additional observations or hypotheses, description of artifacts etc.
\item Can be referenced in main report as ``see Annex''
\end{itemize}

\end{document}