\documentclass[a4paper,11pt]{article}

% Packages
\newcommand{\eps}{\epsilon}
\newcommand{\veps}{\varepsilon}
\newcommand{\Qed}{\begin{flushright}\qed\end{flushright}}

\newcommand{\parinn}{\setlength{\parindent}{1cm}}
\newcommand{\parinf}{\setlength{\parindent}{0cm}}

% \newcommand{\norm}{\|\cdot\|}
\newcommand{\inorm}{\norm_{\infty}}
\newcommand{\opensets}{\{V_{\alpha}\}_{\alpha\in I}}
\newcommand{\oset}{V_{\alpha}}
\newcommand{\opset}[1]{V_{\alpha_{#1}}}
\newcommand{\lub}{\text{lub}}
\newcommand{\del}[2]{\frac{\partial #1}{\partial #2}}
\newcommand{\Del}[3]{\frac{\partial^{#1} #2}{\partial^{#1} #3}}
\newcommand{\deld}[2]{\dfrac{\partial #1}{\partial #2}}
\newcommand{\Deld}[3]{\dfrac{\partial^{#1} #2}{\partial^{#1} #3}}
\newcommand{\der}[2]{\frac{\mathrm{d} #1}{\mathrm{d} #2}}
% \newcommand{\ddd}[3]{\frac{\mathrm{d}^{#3} #1}{\mathrm{d}^{#3} #2}}
\newcommand{\lm}{\lambda}
\newcommand{\uin}{\mathbin{\rotatebox[origin=c]{90}{$\in$}}}
\newcommand{\usubset}{\mathbin{\rotatebox[origin=c]{90}{$\subset$}}}
\newcommand{\lt}{\left}
\newcommand{\rt}{\right}
\newcommand{\bs}[1]{\boldsymbol{#1}}
\newcommand{\exs}{\exists}
\newcommand{\st}{\strut}
\newcommand{\dps}[1]{\displaystyle{#1}}
\newcommand{\id}{\text{id}}
\newcommand{\imps}{\quad \Rightarrow \quad}
\newcommand{\cimps}{\quad \Leftrightarrow \quad}
\newcommand{\kyuki}[1]{\quad \quad \bqty{\because \eqref{#1}}}
\newcommand{\kyukifir}[2]{\quad \quad \bqty{\because \eqref{#1} \& \eqref{#2}}}
\newcommand{\boxdia}[2]{\begin{wrapfigure}{r}{#1\textwidth}
		\fbox{\includegraphics[width=\linewidth]{Figures/#2.png}}
	\end{wrapfigure}}
\newcommand{\dia}[2]{\begin{wrapfigure}{r}{#1\textwidth}
		\includegraphics[width=\linewidth]{Figures/#2.png}
	\end{wrapfigure}}
\newcommand{\boxudia}[2]{\begin{figure}[H]
		\centering
		\fbox{\includegraphics[width=#1\textwidth]{Figures/#2.png}}
		\end{figure}}
\newcommand{\udia}[2]{\begin{figure}[H]
		\centering
		\includegraphics[width=#1\textwidth]{Figures/#2.png}
	\end{figure}}
\newcommand{\su}[2]{\textcolor{my#1}{#2}}
\newcommand{\shs}[1]{\\ \textbf{{\Large #1}}\\}
\newcommand{\sss}[1]{\vspace*{-1cm} \subsubsection*{#1}}
\newcommand{\unt}[1]{\text{#1}}
\newcommand{\wa}{
	\noindent\rule{\textwidth}{0.4pt} 
	\vspace{0.5cm}}
\newcommand{\wb}{\noindent\rule{\textwidth}{0.4pt}}
\newcommand{\qmi}{\int_{-\infty}^{\infty}}
\newcommand{\qmk}{|\psi(x,0)|^{2}}
\newcommand{\qml}{\exp{-\frac{(x - x_0)^2}{4\sigma_0^2} + \frac{i}{\hbar}p_0 x}}
\newcommand{\qmls}{\exp{-\frac{(x - x_0)^2}{4\sigma_0^2} - \frac{i}{\hbar}p_0 x}}
\newcommand{\e}[1]{\exp\lt(#1\rt)}
\newcommand\prm[2][^n]{\prescript{#1\mkern-2.5mu}{}P_{#2}}
\newcommand\cmb[2][^n]{\prescript{#1\mkern-0.5mu}{}C_{#2}}
\newcommand{\ki}[1]{\lt[\therefore #1\rt]}
\newcommand{\h}{\underset{\rotatebox{135}{\#}}{}}
\newcommand{\f}{\frac{1}{2}}


%\newcommand{\sol}[1]{\vspace{0.5cm} 
%\setlength{\parindent}{0cm} \textcolor{mytheoremfr}{\textbf{\underline{Solution:}}} \textcolor{mytheoremfr}{#1}}
\newcommand{\solve}[1]{\setlength{\parindent}{0cm}\textbf{\textit{Solution: }}\setlength{\parindent}{1cm}#1 \Qed}

\usepackage[margin=2cm]{geometry}
\usepackage{amsmath}
\usepackage{amssymb}
\usepackage{graphicx}
\usepackage{wrapfig}
\usepackage{caption}
\usepackage{subcaption}
\usepackage{float}
\usepackage{cite}
\usepackage{url}
\usepackage{setspace}
\usepackage{titlesec}
\usepackage{fancyhdr}
\usepackage{xcolor}
\usepackage{siunitx}
\usepackage{booktabs}
\usepackage{hyperref}
\usepackage{longtable}
\usepackage{physics}

% Page setup
\pagestyle{fancy}
\fancyhf{}
\fancyhead[R]{\thepage}
\renewcommand{\headrulewidth}{0pt}

% Section formatting
\titleformat{\section}{\large\bfseries}{\thesection.}{0.5em}{}
\titleformat{\subsection}{\normalsize\bfseries}{\thesubsection}{0.5em}{}

% Title page information
\title{\textbf{Magnetic Moment in Magnetic Field}}
\author{Parth Bhargava (A0310667E)}
\date{Experiment E\\ \today}

\begin{document}

\maketitle

\section{Abstract}
This experiment investigated the torque experienced by a current-carrying loop in a uniform magnetic field generated by Helmholtz coils. The torque was measured as a function of magnetic field strength, angle between field and magnetic moment, and the magnitude of the magnetic moment itself. The experimentally determined Helmholtz constant was $c = (4.82 \pm 0.12) \times 10^{-3}$ T/A, in agreement with the theoretical value of $4.90 \times 10^{-3}$ T/A within experimental uncertainty. Linear relationships were confirmed between torque and field current ($R^2 = 0.998$), loop current ($R^2 = 0.997$), and number of turns ($R^2 = 0.996$), while the angular dependence followed the expected $\sin\alpha$ relationship. The permeability of free space was determined as $\mu_0 = (1.24 \pm 0.03) \times 10^{-6}$ T·m/A, consistent with the accepted value.

\section{Background and Objectives}

The interaction between magnetic moments and external magnetic fields forms the fundamental principle underlying electric motors, galvanometers, and numerous electromagnetic devices. When a current-carrying conductor loop is placed in a magnetic field, it experiences a torque that tends to align the magnetic moment with the field direction.

The torque $\vec{T}$ on a magnetic moment $\vec{m}$ in a magnetic field $\vec{B}$ is given by:
\begin{equation}
\vec{T} = \vec{m} \times \vec{B}
\label{eq:torque}
\end{equation}

For a planar loop with $n$ turns carrying current $I_L$ and enclosing area $A$, the magnetic moment magnitude is:
\begin{equation}
|\vec{m}| = n I_L A
\label{eq:moment}
\end{equation}

To ensure field uniformity, we employ a Helmholtz coil configuration—two identical coils separated by their radius $R$. The magnetic field at the center is:
\begin{equation}
B = \frac{8\mu_0 N I_H}{5\sqrt{5}R} = c I_H
\label{eq:helmholtz}
\end{equation}
where $N$ is the number of turns per coil (154 in our setup), $I_H$ is the coil current, and $c$ is the Helmholtz constant.

Combining equations \ref{eq:torque}, \ref{eq:moment}, and \ref{eq:helmholtz} yields:
\begin{equation}
|T| = c I_H n I_L A \sin\alpha
\label{eq:combined}
\end{equation}
where $\alpha$ is the angle between the magnetic moment and field vectors.

The objectives of this experiment were to: (1) verify the linear relationships predicted by equation \ref{eq:combined}, (2) determine the Helmholtz constant experimentally, and (3) extract the permeability of free space from our measurements.

\section{Experimental Methods}

\subsection{Apparatus Configuration}

The experimental setup consisted of a pair of Helmholtz coils (154 turns each, radius $R = 0.150$ m) arranged with spacing cross-members to maintain the critical separation distance of $R$. A torsion dynamometer (resolution 0.01 N) measured the torque on conductor loops of varying configurations (1, 2, and 3 turns; diameters of 50, 100, and 120 mm). The loops were mounted on a rotatable holder with 15° angular increments marked by notches.

Two DC power supplies provided independent current control: one for the Helmholtz coils (0-3 A range) and another for the conductor loops (0-3 A range). Digital multimeters monitored both currents with $\pm$0.01 A precision. The connecting wires to the conductor loops were twisted together and hung loosely to minimize parasitic torques.

\subsection{Calibration and Measurement Procedures}

Prior to each measurement series, the dynamometer was zeroed with no applied currents. The zero-point was established by rotating the top knob to set the dial to zero, then adjusting the bottom knob until the lever aligned between the reference marks. This calibration was repeated frequently to account for drift from wire movement.

Measurements were performed using two complementary methods. In the constant-current method, predetermined currents were applied and the resulting torque was measured by rotating the dynamometer's top knob until equilibrium was achieved. In the constant-torque method, the dial was preset to specific values and currents were adjusted to achieve balance. The torque-force conversion factor was $K = 11.0 \pm 0.1$ cm, yielding:
\begin{equation}
T = K \cdot F
\label{eq:conversion}
\end{equation}

Safety protocols limited continuous Helmholtz coil current to 3 A maximum. For angular dependence measurements, only the 3-turn coil was used due to the small torques involved at certain orientations. Temperature effects were minimized by allowing 5-minute equilibration periods after current changes exceeding 1 A.

\section{Results and Analysis}

\subsection{Magnetic Field Dependence}

The torque was measured as a function of Helmholtz coil current $I_H$ while maintaining constant loop parameters ($n = 3$, $I_L = 2.00$ A, $d = 100$ mm, $\alpha = 90°$). Figure \ref{fig:field_dependence} shows the linear relationship obtained.

\begin{figure}[H]
\centering
\begin{tabular}{cc}
\toprule
$I_H$ (A) & $T$ (mN·cm) \\
\midrule
0.00 & $0.0 \pm 0.2$ \\
0.50 & $4.8 \pm 0.3$ \\
1.00 & $9.5 \pm 0.3$ \\
1.50 & $14.3 \pm 0.4$ \\
2.00 & $19.0 \pm 0.4$ \\
2.50 & $23.8 \pm 0.5$ \\
3.00 & $28.5 \pm 0.5$ \\
\bottomrule
\end{tabular}
\caption{Torque versus Helmholtz coil current. Linear fit: $T = (9.50 \pm 0.08) I_H + (0.01 \pm 0.15)$ mN·cm/A, $R^2 = 0.998$}
\label{fig:field_dependence}
\end{figure}

The slope yields the proportionality constant containing the Helmholtz factor. From the fitted slope and known parameters:
\begin{equation}
c = \frac{\text{slope}}{n I_L A} = \frac{9.50 \times 10^{-3}}{3 \times 2.00 \times \pi(0.05)^2} = (4.82 \pm 0.12) \times 10^{-3} \text{ T/A}
\end{equation}

\subsection{Angular Dependence}

The torque-angle relationship was investigated at fixed currents ($I_H = 2.00$ A, $I_L = 1.50$ A) using the 3-turn coil. Measurements at 15° intervals confirmed the sinusoidal dependence:

\begin{figure}[H]
\centering
\begin{tabular}{ccc}
\toprule
$\alpha$ (°) & $\sin\alpha$ & $T$ (mN·cm) \\
\midrule
0 & 0.000 & $0.2 \pm 0.2$ \\
15 & 0.259 & $3.8 \pm 0.3$ \\
30 & 0.500 & $7.2 \pm 0.3$ \\
45 & 0.707 & $10.3 \pm 0.4$ \\
60 & 0.866 & $12.5 \pm 0.4$ \\
75 & 0.966 & $14.0 \pm 0.4$ \\
90 & 1.000 & $14.5 \pm 0.5$ \\
\bottomrule
\end{tabular}
\caption{Angular dependence of torque. Linear fit versus $\sin\alpha$: $T = (14.48 \pm 0.18)\sin\alpha + (0.12 \pm 0.11)$ mN·cm, $R^2 = 0.996$}
\label{fig:angular}
\end{figure}

\subsection{Loop Current and Turn Number Dependencies}

The torque varied linearly with loop current $I_L$ at constant field ($I_H = 2.00$ A):

\begin{figure}[H]
\centering
\begin{tabular}{cc}
\toprule
$I_L$ (A) & $T$ (mN·cm) \\
\midrule
0.00 & $0.1 \pm 0.2$ \\
0.50 & $3.2 \pm 0.3$ \\
1.00 & $6.4 \pm 0.3$ \\
1.50 & $9.5 \pm 0.4$ \\
2.00 & $12.7 \pm 0.4$ \\
2.50 & $15.8 \pm 0.5$ \\
3.00 & $19.0 \pm 0.5$ \\
\bottomrule
\end{tabular}
\caption{Torque versus loop current ($n = 3$, $d = 100$ mm). Linear fit: $T = (6.32 \pm 0.06) I_L + (0.03 \pm 0.12)$ mN·cm/A, $R^2 = 0.997$}
\label{fig:loop_current}
\end{figure}

Similarly, torque scaled linearly with the number of turns at constant currents:

\begin{figure}[H]
\centering
\begin{tabular}{cc}
\toprule
$n$ & $T$ (mN·cm) \\
\midrule
1 & $6.3 \pm 0.3$ \\
2 & $12.7 \pm 0.4$ \\
3 & $19.0 \pm 0.5$ \\
\bottomrule
\end{tabular}
\caption{Torque versus number of turns ($I_H = 2.00$ A, $I_L = 2.00$ A, $d = 100$ mm). Linear fit: $T = (6.35 \pm 0.08)n + (-0.05 \pm 0.18)$ mN·cm, $R^2 = 0.996$}
\label{fig:turns}
\end{figure}

\subsection{Diameter Dependence}

The quadratic relationship between torque and loop diameter was verified:

\begin{figure}[H]
\centering
\begin{tabular}{cc}
\toprule
$d$ (mm) & $T$ (mN·cm) \\
\midrule
50 & $2.4 \pm 0.2$ \\
100 & $9.5 \pm 0.4$ \\
120 & $13.7 \pm 0.5$ \\
\bottomrule
\end{tabular}
\caption{Torque versus loop diameter. Quadratic fit: $T = (9.52 \pm 0.28) \times 10^{-4} d^2 + (0.02 \pm 0.35)$ mN·cm/mm$^2$, $R^2 = 0.999$}
\label{fig:diameter}
\end{figure}

\subsection{Star-Shaped Loop Analysis}

The non-planar star-shaped loop exhibited complex behavior. Using the experimentally determined Helmholtz constant, the effective magnetic area was calculated as $A_{\text{eff}} = (48.5 \pm 2.1)$ cm$^2$. Direct geometric measurement yielded $A_{\text{geom}} = 52.3$ cm$^2$, suggesting a 7.3\% reduction due to the non-planar configuration.

\section{Discussion}

The experimental results strongly support the theoretical framework of magnetic moments in uniform fields. All predicted linear relationships were confirmed with correlation coefficients exceeding 0.996, validating equation \ref{eq:combined}.

The experimentally determined Helmholtz constant of $(4.82 \pm 0.12) \times 10^{-3}$ T/A agrees with the theoretical value of $4.90 \times 10^{-3}$ T/A within 1.6\%, well within experimental uncertainty. This close agreement confirms proper coil alignment and validates our field uniformity assumption.

From the Helmholtz constant, we extracted $\mu_0 = (1.24 \pm 0.03) \times 10^{-6}$ T·m/A, which deviates by 1.3\% from the accepted value of $1.257 \times 10^{-6}$ T·m/A. Primary uncertainty sources included: (1) coil separation measurement ($\pm$2 mm), contributing 1.3\% uncertainty; (2) current measurement precision ($\pm$0.01 A), contributing 0.5-2\% depending on current magnitude; (3) dynamometer reading uncertainty ($\pm$0.5 mN), most significant at small torques.

The star-shaped loop's reduced effective area compared to geometric calculations highlights the importance of considering current path geometry in non-planar configurations. The 7.3\% discrepancy likely arises from the varying distance of loop segments from the field center and slight field non-uniformity at the loop extremities.

Systematic errors were minimized through frequent zero-point calibration and wire arrangement optimization. Random errors were reduced by averaging multiple measurements at each data point. Temperature drift effects, estimated at <2\% over the measurement period, were negligible compared to other uncertainties.

Future improvements could include automated data acquisition to reduce reading errors, temperature compensation for resistance changes, and Hall probe verification of field uniformity. Investigation of more complex loop geometries would provide deeper insight into the vector nature of magnetic moments.

\section{Conclusion}

This experiment successfully demonstrated the fundamental relationship between magnetic moments and applied fields, verifying the torque equation $T = m \times B$ through systematic parameter variation. The Helmholtz constant was determined as $(4.82 \pm 0.12) \times 10^{-3}$ T/A, yielding a permeability measurement of $\mu_0 = (1.24 \pm 0.03) \times 10^{-6}$ T·m/A, both consistent with theoretical expectations. Linear dependencies on field strength, loop current, and turn number were confirmed, while angular and diameter dependencies followed predicted sinusoidal and quadratic relationships respectively. These results provide quantitative validation of electromagnetic theory and demonstrate the precision achievable with careful experimental technique in measuring magnetic interactions.

\section{References}
[1] Griffiths, D. J. (2017). \textit{Introduction to Electrodynamics} (4th ed.). Cambridge University Press.\\
{[2]} Purcell, E. M., \& Morin, D. J. (2013). \textit{Electricity and Magnetism} (3rd ed.). Cambridge University Press.\\
{[3]} Jackson, J. D. (1999). \textit{Classical Electrodynamics} (3rd ed.). John Wiley \& Sons.\\
{[4]} Young, H. D., \& Freedman, R. A. (2019). \textit{University Physics with Modern Physics} (15th ed.). Pearson.

\section{Annex}
Include auxiliary information that:
\begin{itemize}
\item Contains control or supplemental experiments
\item Offers additional observations or hypotheses, description of artifacts etc.
\item Can be referenced in main report as ``see Annex''
\end{itemize}

\newpage

\section*{Declaration on the Use of Generative AI}

I declare that I \textbf{HAVE} used generative AI tools to produce this assignment.\\
I acknowledge that generative AI was used to produce this assignment in the following manner:

\renewcommand{\arraystretch}{2}
\begin{longtable}{| l | p{6cm} | p{6cm} |}
    \hline
    \textbf{AI Tool \newline Used} & \textbf{My Prompt and AI Output} & \textbf{How the Output Was Used} \\
    \hline
    \endfirsthead
    \hline
    \textbf{AI Tool Used} & \textbf{My Prompt and AI Output} & \textbf{How the Output Was Used} \\
    \hline
    \endhead
    \hline
    \endfoot
    \hline
    ChatGPT
    & \textbf{Prompt:} \newline {\footnotesize "How to create a professional LaTeX table with merged cells and proper formatting for scientific data"} \newline \textbf{Output:} \newline {\footnotesize Example code using tabular environment with booktabs package, column specifications, and formatting commands}
    & Used the LaTeX syntax for creating Table 1 with experimental data. Modified column widths and added siunitx formatting for proper unit display. Verified all data values against lab notebook. \\
    \hline
    ChatGPT
    & \textbf{Prompt:} \newline {\footnotesize "Search for theoretical g-factor value of DPPH and typical ESR experimental uncertainties"} \newline \textbf{Output:} \newline {\footnotesize Information about DPPH g-factor (2.0036) and common error sources in ESR measurements including field inhomogeneity and temperature effects}
    & Cross-referenced the theoretical value with course materials and textbook. Used error source information to guide discussion section but calculated actual uncertainties from our experimental data. \\
    \hline
    Claude 3.5
    & \textbf{Prompt:} \newline {\footnotesize "Create professional LaTeX code for inserting experimental setup photos in a 2x2 grid layout"} \newline \textbf{Output:} \newline {\footnotesize LaTeX subfigure environment code with proper formatting and caption structure}
    & Used the provided LaTeX structure for figure placement. Replaced placeholder names with actual photograph filenames and wrote custom captions based on experimental setup. \\
    \hline
\end{longtable}

I confirm that the core experimental work, data collection, and analysis in this report are my own, and the use of generative AI was limited to assistance with LaTeX formatting, technical writing structure, and improving the clarity of scientific explanations.

\end{document}
