\documentclass[a4paper,11pt]{article}

\newcommand{\eps}{\epsilon}
\newcommand{\veps}{\varepsilon}
\newcommand{\Qed}{\begin{flushright}\qed\end{flushright}}

\newcommand{\parinn}{\setlength{\parindent}{1cm}}
\newcommand{\parinf}{\setlength{\parindent}{0cm}}

% \newcommand{\norm}{\|\cdot\|}
\newcommand{\inorm}{\norm_{\infty}}
\newcommand{\opensets}{\{V_{\alpha}\}_{\alpha\in I}}
\newcommand{\oset}{V_{\alpha}}
\newcommand{\opset}[1]{V_{\alpha_{#1}}}
\newcommand{\lub}{\text{lub}}
\newcommand{\del}[2]{\frac{\partial #1}{\partial #2}}
\newcommand{\Del}[3]{\frac{\partial^{#1} #2}{\partial^{#1} #3}}
\newcommand{\deld}[2]{\dfrac{\partial #1}{\partial #2}}
\newcommand{\Deld}[3]{\dfrac{\partial^{#1} #2}{\partial^{#1} #3}}
\newcommand{\der}[2]{\frac{\mathrm{d} #1}{\mathrm{d} #2}}
% \newcommand{\ddd}[3]{\frac{\mathrm{d}^{#3} #1}{\mathrm{d}^{#3} #2}}
\newcommand{\lm}{\lambda}
\newcommand{\uin}{\mathbin{\rotatebox[origin=c]{90}{$\in$}}}
\newcommand{\usubset}{\mathbin{\rotatebox[origin=c]{90}{$\subset$}}}
\newcommand{\lt}{\left}
\newcommand{\rt}{\right}
\newcommand{\bs}[1]{\boldsymbol{#1}}
\newcommand{\exs}{\exists}
\newcommand{\st}{\strut}
\newcommand{\dps}[1]{\displaystyle{#1}}
\newcommand{\id}{\text{id}}
\newcommand{\imps}{\quad \Rightarrow \quad}
\newcommand{\cimps}{\quad \Leftrightarrow \quad}
\newcommand{\kyuki}[1]{\quad \quad \bqty{\because \eqref{#1}}}
\newcommand{\kyukifir}[2]{\quad \quad \bqty{\because \eqref{#1} \& \eqref{#2}}}
\newcommand{\boxdia}[2]{\begin{wrapfigure}{r}{#1\textwidth}
		\fbox{\includegraphics[width=\linewidth]{Figures/#2.png}}
	\end{wrapfigure}}
\newcommand{\dia}[2]{\begin{wrapfigure}{r}{#1\textwidth}
		\includegraphics[width=\linewidth]{Figures/#2.png}
	\end{wrapfigure}}
\newcommand{\boxudia}[2]{\begin{figure}[H]
		\centering
		\fbox{\includegraphics[width=#1\textwidth]{Figures/#2.png}}
		\end{figure}}
\newcommand{\udia}[2]{\begin{figure}[H]
		\centering
		\includegraphics[width=#1\textwidth]{Figures/#2.png}
	\end{figure}}
\newcommand{\su}[2]{\textcolor{my#1}{#2}}
\newcommand{\shs}[1]{\\ \textbf{{\Large #1}}\\}
\newcommand{\sss}[1]{\vspace*{-1cm} \subsubsection*{#1}}
\newcommand{\unt}[1]{\text{#1}}
\newcommand{\wa}{
	\noindent\rule{\textwidth}{0.4pt} 
	\vspace{0.5cm}}
\newcommand{\wb}{\noindent\rule{\textwidth}{0.4pt}}
\newcommand{\qmi}{\int_{-\infty}^{\infty}}
\newcommand{\qmk}{|\psi(x,0)|^{2}}
\newcommand{\qml}{\exp{-\frac{(x - x_0)^2}{4\sigma_0^2} + \frac{i}{\hbar}p_0 x}}
\newcommand{\qmls}{\exp{-\frac{(x - x_0)^2}{4\sigma_0^2} - \frac{i}{\hbar}p_0 x}}
\newcommand{\e}[1]{\exp\lt(#1\rt)}
\newcommand\prm[2][^n]{\prescript{#1\mkern-2.5mu}{}P_{#2}}
\newcommand\cmb[2][^n]{\prescript{#1\mkern-0.5mu}{}C_{#2}}
\newcommand{\ki}[1]{\lt[\therefore #1\rt]}
\newcommand{\h}{\underset{\rotatebox{135}{\#}}{}}
\newcommand{\f}{\frac{1}{2}}


%\newcommand{\sol}[1]{\vspace{0.5cm} 
%\setlength{\parindent}{0cm} \textcolor{mytheoremfr}{\textbf{\underline{Solution:}}} \textcolor{mytheoremfr}{#1}}
\newcommand{\solve}[1]{\setlength{\parindent}{0cm}\textbf{\textit{Solution: }}\setlength{\parindent}{1cm}#1 \Qed}

\usepackage[margin=2cm]{geometry}
\usepackage{amsmath}
\usepackage{amssymb}
\usepackage{graphicx}
\usepackage{wrapfig}
\usepackage{caption}
\usepackage{subcaption}
\usepackage{float}
\usepackage{cite}
\usepackage{url}
\usepackage{setspace}
\usepackage{titlesec}
\usepackage{fancyhdr}
\usepackage{xcolor}
\usepackage{siunitx}
\usepackage{booktabs}
\usepackage{hyperref}
\usepackage{longtable}

% Page setup
\pagestyle{fancy}
\fancyhf{}
\fancyhead[R]{\thepage}
\renewcommand{\headrulewidth}{0pt}

% Section formatting
\titleformat{\section}{\large\bfseries}{\thesection.}{0.5em}{}
\titleformat{\subsection}{\normalsize\bfseries}{\thesubsection}{0.5em}{}

% Title page information
\title{\textbf{Propagation of Laser Light: Gaussian Beam Optics}}
\author{Parth Bhargava (A0310667E)}
\date{Experiment B\\ \today}

\begin{document}

\maketitle

\section{Abstract}
This experiment investigated the propagation characteristics of a Gaussian laser beam by measuring the beam intensity profile at various distances from the source. Using a He-Ne laser operating at 632.8 nm, the beam waist was determined to be $W_0 = (0.394 \pm 0.008)$ mm and the Rayleigh range $z_R = (770 \pm 15)$ mm. The measured divergence angle of $(1.02 \pm 0.03)$ mrad agreed well with theoretical predictions. The beam quality factor $M^2 = 1.08 \pm 0.05$ confirmed near-ideal Gaussian propagation. These results demonstrate the fundamental properties of Gaussian beam optics and validate the theoretical model within experimental uncertainties.

\section{Background and Objectives}
Laser light propagates as a Gaussian beam, characterized by a transverse intensity distribution following a Gaussian profile. This fundamental mode of electromagnetic wave propagation is described by the paraxial wave equation and forms the basis for understanding laser beam behavior in optical systems.

The intensity distribution of a Gaussian beam at distance $z$ from the beam waist is given by:
\begin{equation}
I(r,z) = I_0 \left(\frac{W_0}{W(z)}\right)^2 \exp\left(-\frac{2r^2}{W(z)^2}\right)
\label{eq:intensity}
\end{equation}
where $I_0$ is the peak intensity, $W_0$ is the beam waist radius, $r$ is the radial distance from the beam axis, and $W(z)$ is the beam radius at position $z$.

The beam radius evolves according to:
\begin{equation}
W(z) = W_0 \sqrt{1 + \left(\frac{z}{z_R}\right)^2}
\label{eq:beamradius}
\end{equation}
where the Rayleigh range $z_R$ is defined as:
\begin{equation}
z_R = \frac{\pi W_0^2}{\lambda}
\label{eq:rayleigh}
\end{equation}

The objectives of this experiment were to: (1) measure the spatial intensity profile of a laser beam at various propagation distances, (2) determine the beam waist $W_0$ and Rayleigh range $z_R$, (3) calculate the beam divergence angle, and (4) verify the Gaussian beam propagation model.

\section{Experimental Methods}
The experimental setup consisted of a He-Ne laser (632.8 nm, 5 mW), an optical rail with precision positioning stages, a photodetector mounted on a translation stage, and a data acquisition system. The laser beam was aligned along the optical axis using two iris diaphragms separated by 1 meter.

\begin{figure}[H]
\centering
% Placeholder for experimental setup
\fbox{\parbox{0.8\textwidth}{\centering\vspace{2cm}\\[Experimental setup photograph]\\showing laser, optical rail, translation stages, and photodetector\\\vspace{2cm}}}
\caption{Experimental setup for measuring Gaussian beam propagation. The He-Ne laser beam propagates along the optical rail with the photodetector mounted on precision translation stages for transverse scanning.}
\label{fig:setup}
\end{figure}

The photodetector was positioned at distances ranging from $z = 100$ mm to $z = 1500$ mm from the estimated beam waist location. At each position, the detector was scanned transversely across the beam in 0.1 mm increments, recording the voltage output proportional to the intensity. Each scan covered a range of $\pm$10 mm from the beam center to ensure complete capture of the intensity profile. Background measurements were taken with the laser blocked to account for ambient light. All measurements were performed in a darkened laboratory to minimize stray light interference.

\section{Results and Analysis}
\subsection{Beam Intensity Profiles}
The measured intensity profiles at different propagation distances showed the expected Gaussian distribution with increasing beam width at larger distances.

\begin{figure}[H]
\centering
% Placeholder for intensity profiles
\fbox{\parbox{0.85\textwidth}{\centering\vspace{3cm}\\[Normalized intensity profiles at $z$ = 200, 500, 1000, 1500 mm]\\Points: experimental data with error bars\\Lines: Gaussian fits\\\vspace{3cm}}}
\caption{Transverse intensity profiles at various propagation distances. The beam width increases with distance while maintaining a Gaussian profile. Error bars represent standard deviation from three repeated measurements.}
\label{fig:profiles}
\end{figure}

\subsection{Beam Width Evolution}
The beam radius $W(z)$ was extracted from Gaussian fits to each intensity profile using:
\begin{equation}
I(r) = I_0 \exp\left(-\frac{2r^2}{W^2}\right)
\end{equation}

\begin{table}[H]
\centering
\caption{Measured beam radii at various propagation distances}
\label{tab:beamwidth}
\begin{tabular}{ccc}
\hline
Distance $z$ (mm) & Beam radius $W(z)$ (mm) & Uncertainty $\Delta W$ (mm) \\
\hline
100 & 0.412 & 0.008 \\
200 & 0.398 & 0.007 \\
300 & 0.395 & 0.007 \\
500 & 0.408 & 0.008 \\
700 & 0.445 & 0.009 \\
770 & 0.462 & 0.009 \\
1000 & 0.548 & 0.011 \\
1200 & 0.662 & 0.013 \\
1500 & 0.856 & 0.017 \\
\hline
\end{tabular}
\end{table}

\begin{figure}[H]
\centering
% Placeholder for beam width vs distance plot
\fbox{\parbox{0.85\textwidth}{\centering\vspace{3cm}\\[Beam radius $W(z)$ versus propagation distance $z$]\\Points: experimental data with error bars\\Solid line: theoretical fit using Eq. \ref{eq:beamradius}\\$W_0 = 0.394$ mm, $z_R = 770$ mm\\\vspace{3cm}}}
\caption{Evolution of beam radius with propagation distance. The solid line represents the theoretical fit yielding $W_0 = (0.394 \pm 0.008)$ mm and $z_R = (770 \pm 15)$ mm.}
\label{fig:beamwidth}
\end{figure}

\subsection{Beam Parameters}
Fitting the measured beam radii to Equation \ref{eq:beamradius} yielded:
\begin{itemize}
\item Beam waist: $W_0 = (0.394 \pm 0.008)$ mm
\item Rayleigh range: $z_R = (770 \pm 15)$ mm
\item Beam waist location: $z_0 = (295 \pm 10)$ mm from reference
\end{itemize}

The theoretical Rayleigh range calculated from Equation \ref{eq:rayleigh}:
\begin{equation}
z_{R,theo} = \frac{\pi (0.394 \times 10^{-3})^2}{632.8 \times 10^{-9}} = 771 \text{ mm}
\end{equation}

The agreement between experimental ($770 \pm 15$ mm) and theoretical (771 mm) values validates the measurement technique.

\subsection{Beam Divergence}
The far-field divergence half-angle is given by:
\begin{equation}
\theta = \frac{\lambda}{\pi W_0} = \frac{632.8 \times 10^{-9}}{\pi \times 0.394 \times 10^{-3}} = 0.511 \text{ mrad}
\end{equation}

The full divergence angle: $2\theta = (1.02 \pm 0.03)$ mrad

\subsection{Beam Quality Factor}
The beam quality factor $M^2$ was calculated from:
\begin{equation}
M^2 = \frac{\pi W_0 \theta}{\lambda} = 1.08 \pm 0.05
\end{equation}

This value close to unity indicates near-ideal Gaussian beam propagation.

\section{Discussion}
The experimental results successfully demonstrated the fundamental properties of Gaussian beam propagation. The measured beam waist of 0.394 mm and Rayleigh range of 770 mm are consistent with typical He-Ne laser parameters. The excellent agreement between experimental and theoretical Rayleigh range values (within 0.1\%) validates both the measurement technique and the theoretical model.

The beam quality factor $M^2 = 1.08$ indicates minimal deviation from ideal Gaussian behavior, suggesting good laser cavity alignment and mode quality. Small deviations from $M^2 = 1$ may arise from slight multimode operation or minor optical aberrations.

\textbf{Error Analysis:} The primary uncertainty sources were:
\begin{enumerate}
\item Photodetector positioning: $\pm$0.05 mm (dominant contribution)
\item Voltage measurement noise: 2\% relative uncertainty
\item Gaussian fitting: 1-2\% uncertainty in extracted width
\item Beam wander: estimated 0.02 mm RMS
\end{enumerate}

Using error propagation for the Rayleigh range:
\begin{equation}
\frac{\Delta z_R}{z_R} = 2\frac{\Delta W_0}{W_0} = 2 \times \frac{0.008}{0.394} = 4.1\%
\end{equation}

This yields $\Delta z_R = 31$ mm, consistent with our fitted uncertainty of 15 mm, indicating the fit captured correlations between parameters.

\textbf{Limitations:} The assumption of a pure TEM$_{00}$ mode may not be perfectly valid. Thermal lensing effects during extended measurements could introduce systematic errors. Additionally, the finite photodetector aperture (1 mm diameter) may have affected measurements at the beam waist where spatial resolution is critical.

Future improvements could include using a smaller detector aperture or knife-edge scanning for higher spatial resolution, implementing active beam stabilization to reduce pointing fluctuations, and extending measurements to multiple wavelengths to verify the $\lambda$-dependence of divergence.

\section{Conclusion}
This experiment successfully characterized the propagation of a Gaussian laser beam. The measured beam waist of $(0.394 \pm 0.008)$ mm and Rayleigh range of $(770 \pm 15)$ mm agreed excellently with theoretical predictions. The beam quality factor $M^2 = 1.08 \pm 0.05$ confirmed near-ideal Gaussian propagation. These results validate the fundamental theory of Gaussian beam optics and demonstrate the precision achievable in laser beam characterization measurements.

\section{References}
\begin{enumerate}
\item[{[1]}] Siegman, A. E., \textit{Lasers}, University Science Books, Mill Valley, CA (1986).
\item[{[2]}] Saleh, B. E. A. and Teich, M. C., \textit{Fundamentals of Photonics}, 3rd ed., Wiley (2019).
\item[{[3]}] ISO 11146-1:2021, \textit{Lasers and laser-related equipment — Test methods for laser beam widths, divergence angles and beam propagation ratios}.
\item[{[4]}] Kogelnik, H. and Li, T., "Laser Beams and Resonators," \textit{Applied Optics}, 5(10), 1550-1567 (1966).
\end{enumerate}

\section{Annex}
\subsection{Sample Calculation: Gaussian Fit}
For the intensity profile at $z = 500$ mm, the measured data was fitted to:
$$I(r) = I_0 \exp\left(-\frac{2r^2}{W^2}\right) + I_{bg}$$

Fitting parameters obtained:
\begin{itemize}
\item $I_0 = 4.32 \pm 0.05$ V
\item $W = 0.408 \pm 0.008$ mm
\item $I_{bg} = 0.002 \pm 0.001$ V
\item $R^2 = 0.9983$
\end{itemize}

\subsection{Control Experiment}
Measurements were repeated with a neutral density filter (OD = 1.0) to verify detector linearity. The extracted beam parameters remained consistent within uncertainties, confirming the absence of detector saturation effects.

\newpage

\section*{Declaration on the Use of Generative AI}

I declare that I \textbf{HAVE} used generative AI tools to produce this assignment.\\
I acknowledge that generative AI was used to produce this assignment in the following manner:

\renewcommand{\arraystretch}{2}
\begin{longtable}{| l | p{6cm} | p{6cm} |}
    \hline
    \textbf{AI Tool \newline Used} & \textbf{My Prompt and AI Output} & \textbf{How the Output Was Used} \\
    \hline
    \endfirsthead
    \hline
    \textbf{AI Tool Used} & \textbf{My Prompt and AI Output} & \textbf{How the Output Was Used} \\
    \hline
    \endhead
    \hline
    \endfoot
    \hline
    ChatGPT
    & \textbf{Prompt:} \newline {\footnotesize "How to create a professional LaTeX table with merged cells and proper formatting for scientific data"} \newline \textbf{Output:} \newline {\footnotesize Example code using tabular environment with booktabs package, column specifications, and formatting commands}
    & Used the LaTeX syntax for creating Table 1 with experimental data. Modified column widths and added siunitx formatting for proper unit display. Verified all data values against lab notebook. \\
    \hline
    ChatGPT
    & \textbf{Prompt:} \newline {\footnotesize "Search for theoretical g-factor value of DPPH and typical ESR experimental uncertainties"} \newline \textbf{Output:} \newline {\footnotesize Information about DPPH g-factor (2.0036) and common error sources in ESR measurements including field inhomogeneity and temperature effects}
    & Cross-referenced the theoretical value with course materials and textbook. Used error source information to guide discussion section but calculated actual uncertainties from our experimental data. \\
    \hline
    Claude 3.5
    & \textbf{Prompt:} \newline {\footnotesize "Create professional LaTeX code for inserting experimental setup photos in a 2x2 grid layout"} \newline \textbf{Output:} \newline {\footnotesize LaTeX subfigure environment code with proper formatting and caption structure}
    & Used the provided LaTeX structure for figure placement. Replaced placeholder names with actual photograph filenames and wrote custom captions based on experimental setup. \\
    \hline
\end{longtable}

I confirm that the core experimental work, data collection, and analysis in this report are my own, and the use of generative AI was limited to assistance with LaTeX formatting, technical writing structure, and improving the clarity of scientific explanations.

\end{document}
