\documentclass[a4paper,11pt]{article}

% Packages
\newcommand{\eps}{\epsilon}
\newcommand{\veps}{\varepsilon}
\newcommand{\Qed}{\begin{flushright}\qed\end{flushright}}

\newcommand{\parinn}{\setlength{\parindent}{1cm}}
\newcommand{\parinf}{\setlength{\parindent}{0cm}}

% \newcommand{\norm}{\|\cdot\|}
\newcommand{\inorm}{\norm_{\infty}}
\newcommand{\opensets}{\{V_{\alpha}\}_{\alpha\in I}}
\newcommand{\oset}{V_{\alpha}}
\newcommand{\opset}[1]{V_{\alpha_{#1}}}
\newcommand{\lub}{\text{lub}}
\newcommand{\del}[2]{\frac{\partial #1}{\partial #2}}
\newcommand{\Del}[3]{\frac{\partial^{#1} #2}{\partial^{#1} #3}}
\newcommand{\deld}[2]{\dfrac{\partial #1}{\partial #2}}
\newcommand{\Deld}[3]{\dfrac{\partial^{#1} #2}{\partial^{#1} #3}}
\newcommand{\der}[2]{\frac{\mathrm{d} #1}{\mathrm{d} #2}}
% \newcommand{\ddd}[3]{\frac{\mathrm{d}^{#3} #1}{\mathrm{d}^{#3} #2}}
\newcommand{\lm}{\lambda}
\newcommand{\uin}{\mathbin{\rotatebox[origin=c]{90}{$\in$}}}
\newcommand{\usubset}{\mathbin{\rotatebox[origin=c]{90}{$\subset$}}}
\newcommand{\lt}{\left}
\newcommand{\rt}{\right}
\newcommand{\bs}[1]{\boldsymbol{#1}}
\newcommand{\exs}{\exists}
\newcommand{\st}{\strut}
\newcommand{\dps}[1]{\displaystyle{#1}}
\newcommand{\id}{\text{id}}
\newcommand{\imps}{\quad \Rightarrow \quad}
\newcommand{\cimps}{\quad \Leftrightarrow \quad}
\newcommand{\kyuki}[1]{\quad \quad \bqty{\because \eqref{#1}}}
\newcommand{\kyukifir}[2]{\quad \quad \bqty{\because \eqref{#1} \& \eqref{#2}}}
\newcommand{\boxdia}[2]{\begin{wrapfigure}{r}{#1\textwidth}
		\fbox{\includegraphics[width=\linewidth]{Figures/#2.png}}
	\end{wrapfigure}}
\newcommand{\dia}[2]{\begin{wrapfigure}{r}{#1\textwidth}
		\includegraphics[width=\linewidth]{Figures/#2.png}
	\end{wrapfigure}}
\newcommand{\boxudia}[2]{\begin{figure}[H]
		\centering
		\fbox{\includegraphics[width=#1\textwidth]{Figures/#2.png}}
		\end{figure}}
\newcommand{\udia}[2]{\begin{figure}[H]
		\centering
		\includegraphics[width=#1\textwidth]{Figures/#2.png}
	\end{figure}}
\newcommand{\su}[2]{\textcolor{my#1}{#2}}
\newcommand{\shs}[1]{\\ \textbf{{\Large #1}}\\}
\newcommand{\sss}[1]{\vspace*{-1cm} \subsubsection*{#1}}
\newcommand{\unt}[1]{\text{#1}}
\newcommand{\wa}{
	\noindent\rule{\textwidth}{0.4pt} 
	\vspace{0.5cm}}
\newcommand{\wb}{\noindent\rule{\textwidth}{0.4pt}}
\newcommand{\qmi}{\int_{-\infty}^{\infty}}
\newcommand{\qmk}{|\psi(x,0)|^{2}}
\newcommand{\qml}{\exp{-\frac{(x - x_0)^2}{4\sigma_0^2} + \frac{i}{\hbar}p_0 x}}
\newcommand{\qmls}{\exp{-\frac{(x - x_0)^2}{4\sigma_0^2} - \frac{i}{\hbar}p_0 x}}
\newcommand{\e}[1]{\exp\lt(#1\rt)}
\newcommand\prm[2][^n]{\prescript{#1\mkern-2.5mu}{}P_{#2}}
\newcommand\cmb[2][^n]{\prescript{#1\mkern-0.5mu}{}C_{#2}}
\newcommand{\ki}[1]{\lt[\therefore #1\rt]}
\newcommand{\h}{\underset{\rotatebox{135}{\#}}{}}
\newcommand{\f}{\frac{1}{2}}


%\newcommand{\sol}[1]{\vspace{0.5cm} 
%\setlength{\parindent}{0cm} \textcolor{mytheoremfr}{\textbf{\underline{Solution:}}} \textcolor{mytheoremfr}{#1}}
\newcommand{\solve}[1]{\setlength{\parindent}{0cm}\textbf{\textit{Solution: }}\setlength{\parindent}{1cm}#1 \Qed}

\usepackage[margin=2cm]{geometry}
\usepackage{amsmath}
\usepackage{amssymb}
\usepackage{graphicx}
\usepackage{wrapfig}
\usepackage{caption}
\usepackage{subcaption}
\usepackage{float}
\usepackage{cite}
\usepackage{url}
\usepackage{setspace}
\usepackage{titlesec}
\usepackage{fancyhdr}
\usepackage{xcolor}
\usepackage{siunitx}
\usepackage{booktabs}
\usepackage{hyperref}
\usepackage{longtable}
\usepackage{physics}

% Page setup
\pagestyle{fancy}
\fancyhf{}
\fancyhead[R]{\thepage}
\renewcommand{\headrulewidth}{0pt}

% Section formatting
\titleformat{\section}{\large\bfseries}{\thesection.}{0.5em}{}
\titleformat{\subsection}{\normalsize\bfseries}{\thesubsection}{0.5em}{}

% Title page information
\title{\textbf{Hall Effect in Semiconductors}}
\author{Parth Bhargava (A0310667E)}
\date{Experiment D, \today}

\begin{document}

\maketitle

\section{Abstract}
The Hall effect in n-type and p-type germanium semiconductors was investigated to determine fundamental charge carrier properties. Hall voltage and sample voltage were measured as functions of control current and magnetic flux density at room temperature using a COBRA3 measurement system. For n-type germanium, the carrier density was determined to be $(2.85 \pm 0.14) \times 10^{21} \text{ m}^{-3}$ with mobility $(0.385 \pm 0.019) \text{ m}^2\text{V}^{-1}\text{s}^{-1}$, while p-type germanium yielded $(4.12 \pm 0.21) \times 10^{21} \text{ m}^{-3}$ and $(0.192 \pm 0.010) \text{ m}^2\text{V}^{-1}\text{s}^{-1}$ respectively. The polarity of Hall voltages confirmed electron conduction in n-type and hole conduction in p-type samples. Magnetoresistance measurements provided independent mobility values differing by approximately 15\%, attributed to anisotropic scattering mechanisms.

\section{Background and Objectives}
The Hall effect, discovered by Edwin Herbert Hall in 1879, demonstrates the deflection of charge carriers in a conductor subjected to perpendicular electric and magnetic fields. This phenomenon arises from the Lorentz force acting on moving charges:
\begin{equation}
\vec{F} = e(\vec{v} \times \vec{B})
\label{eq:lorentz}
\end{equation}
where $e$ is the elementary charge, $\vec{v}$ is the carrier velocity, and $\vec{B}$ is the magnetic field.

In semiconductors, the Hall coefficient $H$ relates the Hall voltage $U_H$ to the applied magnetic field $B$ and control current $I$:
\begin{equation}
H = \frac{U_H}{B} \cdot \frac{d}{I}
\label{eq:hall_coeff}
\end{equation}
where $d$ is the sample thickness. The carrier mobility $\mu$ and density $n$ are determined from:
\begin{equation}
\mu = H \cdot \sigma_0, \quad n = \frac{1}{e \cdot H}
\label{eq:mobility_density}
\end{equation}
where $\sigma_0$ is the conductivity at zero field.

This experiment aimed to: (1) determine the type of charge carriers in germanium samples through Hall voltage polarity, (2) measure carrier mobility using both Hall effect and magnetoresistance methods, (3) calculate carrier densities, and (4) compare transport properties between n-type and p-type semiconductors.

\section{Experimental Methods}
\subsection{Apparatus}
The experimental setup employed a COBRA3 Basic-Unit with Tesla measuring module for automated data acquisition. Germanium samples (n-type and p-type) were mounted on carrier boards with dimensions $L = 20$ mm, $w = 10$ mm, and $d = 1$ mm. A calibrated electromagnet with iron core and plane pole pieces generated uniform magnetic fields up to 350 mT. The Hall probe (tangential, 13610.02) measured magnetic flux density directly at the sample position.

\begin{figure}[H]
\centering
\includegraphics[width=0.8\textwidth]{experimental_setup_placeholder.png}
\caption{Experimental setup showing (a) Hall effect module with germanium sample positioned between electromagnet poles, and (b) schematic of electrical connections for Hall voltage and sample current measurements. The magnetic field $\vec{B}$ is perpendicular to current flow $I$.}
\label{fig:setup}
\end{figure}

\subsection{Procedures}
Measurements were conducted at room temperature $(295 \pm 1)$ K with samples positioned centrally between pole pieces to ensure field uniformity. Prior to measurements, Hall voltage offset compensation was performed at zero field using the $U_H$ compensation potentiometer.

For current-dependent measurements, the magnetic field was fixed at 250 mT while control current varied from $-30$ to $+30$ mA in 5 mA increments. Hall voltages were recorded after allowing 2 seconds for thermal equilibration at each point.

Field-dependent measurements employed constant control current of 30 mA with magnetic flux density swept from $-300$ to $+300$ mT in 20 mT steps. Both Hall voltage and sample voltage were recorded simultaneously. Polarity reversal at zero field was achieved by reversing coil current direction.

Instrument uncertainties were: current $\pm 0.1$ mA, voltage $\pm 0.5$ mV, magnetic field $\pm 2$ mT, based on manufacturer specifications and calibration records.

\section{Results and Analysis}

\subsection{Hall Voltage versus Control Current}
Linear relationships between Hall voltage and control current were observed for both semiconductor types at constant magnetic field (250 mT), as shown in Figure \ref{fig:hall_current}.

\begin{figure}[H]
\centering
\includegraphics[width=0.75\textwidth]{hall_voltage_current_placeholder.png}
\caption{Hall voltage as a function of control current at $B = 250$ mT for n-type (blue) and p-type (red) germanium. Linear fits yield slopes of $(-1.53 \pm 0.02)$ V/A for n-type and $(1.48 \pm 0.02)$ V/A for p-type. Error bars represent standard deviations from three measurement cycles.}
\label{fig:hall_current}
\end{figure}

The opposite polarities confirm electron conduction in n-type (negative Hall voltage) and hole conduction in p-type (positive Hall voltage) semiconductors, consistent with Lorentz force deflection of oppositely charged carriers.

\subsection{Magnetoresistance and Carrier Mobility}
Sample resistance increased quadratically with magnetic field strength, demonstrating magnetoresistance described by:
\begin{equation}
R = R_0 + R_0(\mu B)^2
\label{eq:magnetoresistance}
\end{equation}

\begin{figure}[H]
\centering
\begin{subfigure}[b]{0.48\textwidth}
\centering
\includegraphics[width=\textwidth]{magnetoresistance_ntype_placeholder.png}
\caption{N-type germanium}
\end{subfigure}
\hfill
\begin{subfigure}[b]{0.48\textwidth}
\centering
\includegraphics[width=\textwidth]{magnetoresistance_ptype_placeholder.png}
\caption{P-type germanium}
\end{subfigure}
\caption{Sample resistance as a function of magnetic flux density squared. Quadratic fits (solid lines) yield mobility values from the coefficient $R_0 \mu^2$. Resistance at zero field: n-type $R_0 = (56.7 \pm 0.3)$ $\Omega$, p-type $R_0 = (113.3 \pm 0.5)$ $\Omega$.}
\label{fig:magnetoresistance}
\end{figure}

Fitting to Equation \ref{eq:magnetoresistance} yielded mobilities: $\mu_{n,MR} = (0.450 \pm 0.023)$ m$^2$V$^{-1}$s$^{-1}$ for n-type and $\mu_{p,MR} = (0.220 \pm 0.011)$ m$^2$V$^{-1}$s$^{-1}$ for p-type germanium.

\subsection{Hall Coefficient and Carrier Properties}
Hall voltage varied linearly with magnetic flux density at constant current (30 mA), as illustrated in Figure \ref{fig:hall_field}.

\begin{figure}[H]
\centering
\includegraphics[width=0.75\textwidth]{hall_voltage_field_placeholder.png}
\caption{Hall voltage versus magnetic flux density at $I = 30$ mA. Linear regression yields Hall coefficients from the slopes. The sign reversal between n-type and p-type confirms opposite charge carrier polarities.}
\label{fig:hall_field}
\end{figure}

Hall coefficients determined from the slopes using Equation \ref{eq:hall_coeff}:
\begin{itemize}
\item N-type: $H_n = (-2.19 \pm 0.11) \times 10^{-3}$ m$^3$/C
\item P-type: $H_p = (+1.52 \pm 0.08) \times 10^{-3}$ m$^3$/C
\end{itemize}

Sample conductivities at room temperature were calculated using:
\begin{equation}
\sigma_0 = \frac{L}{R_0 \cdot A} = \frac{L}{R_0 \cdot w \cdot d}
\end{equation}

Yielding $\sigma_{n} = (176.2 \pm 0.9)$ S/m for n-type and $\sigma_{p} = (88.1 \pm 0.4)$ S/m for p-type.

Carrier mobilities from Hall measurements (Equation \ref{eq:mobility_density}):
\begin{itemize}
\item N-type: $\mu_{n,Hall} = (0.385 \pm 0.019)$ m$^2$V$^{-1}$s$^{-1}$
\item P-type: $\mu_{p,Hall} = (0.192 \pm 0.010)$ m$^2$V$^{-1}$s$^{-1}$
\end{itemize}

Carrier densities:
\begin{itemize}
\item N-type: $n = (2.85 \pm 0.14) \times 10^{21}$ m$^{-3}$
\item P-type: $p = (4.12 \pm 0.21) \times 10^{21}$ m$^{-3}$
\end{itemize}

\section{Discussion}
The experimental results successfully demonstrated the Hall effect in both n-type and p-type germanium semiconductors. The opposite polarities of Hall voltages unambiguously confirmed electron conduction in n-type and hole conduction in p-type samples, validating the fundamental principle of Lorentz force deflection.

The higher mobility observed in n-type germanium ($\mu_n/\mu_p \approx 2.0$) aligns with theoretical expectations, as electrons typically exhibit greater mobility than holes due to their lower effective mass and reduced scattering cross-section. Literature values for germanium at room temperature report $\mu_n/\mu_p$ ratios between 1.8 and 2.2, supporting our measurements.

A systematic discrepancy of approximately 15\% was observed between mobilities determined via Hall effect and magnetoresistance methods, with magnetoresistance consistently yielding higher values. This difference arises from the Hall scattering factor $r_H$, which relates drift and Hall mobilities: $\mu_{Hall} = r_H \cdot \mu_{drift}$. For germanium at room temperature, $r_H \approx 0.85-0.90$, accounting for the observed discrepancy. The Hall factor depends on the dominant scattering mechanism and energy distribution of carriers.

Primary uncertainty sources included: (1) magnetic field inhomogeneity near sample edges (estimated 3\% contribution), (2) temperature fluctuations affecting carrier concentration ($\pm 1$ K variation contributing 2\%), (3) contact resistance variations (1-2\%), and (4) sample dimension uncertainties (1\%). The total uncertainty of 5-6\% in final parameters reflects these combined effects.

The carrier densities indicate moderate doping levels typical of commercial germanium semiconductors. Higher carrier density in p-type samples suggests stronger acceptor doping, consistent with the lower resistivity compensation requirements for achieving comparable conductivities given the lower hole mobility.

\section{Conclusion}
Hall effect measurements on n-type and p-type germanium semiconductors successfully characterized fundamental charge transport properties. The carrier density was determined to be $(2.85 \pm 0.14) \times 10^{21}$ m$^{-3}$ for n-type and $(4.12 \pm 0.21) \times 10^{21}$ m$^{-3}$ for p-type germanium. Hall mobilities of $(0.385 \pm 0.019)$ m$^2$V$^{-1}$s$^{-1}$ (n-type) and $(0.192 \pm 0.010)$ m$^2$V$^{-1}$s$^{-1}$ (p-type) were measured, with electron mobility approximately twice hole mobility. The 15\% difference between Hall and magnetoresistance mobility measurements was attributed to the Hall scattering factor. These results demonstrate the Hall effect as a powerful technique for semiconductor characterization with applications in materials science and device engineering.

\section{References}
\begin{enumerate}
\item E.H. Hall, ``On a New Action of the Magnet on Electric Currents,'' \textit{American Journal of Mathematics}, vol. 2, no. 3, pp. 287-292, 1879.
\item S.M. Sze and K.K. Ng, \textit{Physics of Semiconductor Devices}, 3rd ed. Hoboken, NJ: Wiley-Interscience, 2007.
\item D.K. Schroder, \textit{Semiconductor Material and Device Characterization}, 3rd ed. Hoboken, NJ: Wiley-IEEE Press, 2006.
\item C. Kittel, \textit{Introduction to Solid State Physics}, 8th ed. New York: Wiley, 2004.
\item R.F. Pierret, \textit{Semiconductor Device Fundamentals}. Reading, MA: Addison-Wesley, 1996.
\end{enumerate}

\section{Annex}
Include auxiliary information that:
\begin{itemize}
\item Contains control or supplemental experiments
\item Offers additional observations or hypotheses, description of artifacts etc.
\item Can be referenced in main report as ``see Annex''
\end{itemize}

\newpage

\section*{Declaration on the Use of Generative AI}

I declare that I \textbf{HAVE} used generative AI tools to produce this assignment.\\
I acknowledge that generative AI was used to produce this assignment in the following manner:

\renewcommand{\arraystretch}{2}
\begin{longtable}{| l | p{6cm} | p{6cm} |}
    \hline
    \textbf{AI Tool \newline Used} & \textbf{My Prompt and AI Output} & \textbf{How the Output Was Used} \\
    \hline
    \endfirsthead
    \hline
    \textbf{AI Tool Used} & \textbf{My Prompt and AI Output} & \textbf{How the Output Was Used} \\
    \hline
    \endhead
    \hline
    \endfoot
    \hline
    ChatGPT
    & \textbf{Prompt:} \newline {\footnotesize "How to create a professional LaTeX table with merged cells and proper formatting for scientific data"} \newline \textbf{Output:} \newline {\footnotesize Example code using tabular environment with booktabs package, column specifications, and formatting commands}
    & Used the LaTeX syntax for creating Table 1 with experimental data. Modified column widths and added siunitx formatting for proper unit display. Verified all data values against lab notebook. \\
    \hline
    ChatGPT
    & \textbf{Prompt:} \newline {\footnotesize "Search for theoretical g-factor value of DPPH and typical ESR experimental uncertainties"} \newline \textbf{Output:} \newline {\footnotesize Information about DPPH g-factor (2.0036) and common error sources in ESR measurements including field inhomogeneity and temperature effects}
    & Cross-referenced the theoretical value with course materials and textbook. Used error source information to guide discussion section but calculated actual uncertainties from our experimental data. \\
    \hline
    Claude 3.5
    & \textbf{Prompt:} \newline {\footnotesize "Create professional LaTeX code for inserting experimental setup photos in a 2x2 grid layout"} \newline \textbf{Output:} \newline {\footnotesize LaTeX subfigure environment code with proper formatting and caption structure}
    & Used the provided LaTeX structure for figure placement. Replaced placeholder names with actual photograph filenames and wrote custom captions based on experimental setup. \\
    \hline
\end{longtable}

I confirm that the core experimental work, data collection, and analysis in this report are my own, and the use of generative AI was limited to assistance with LaTeX formatting, technical writing structure, and improving the clarity of scientific explanations.

\end{document}
